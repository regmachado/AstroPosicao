\documentclass[tikz, border={0 0 0 0}]{standalone}

\usepackage{preambletikz}

\begin{document}

% ascensão reta e declinação (graus)
\def\ra{30}
\def\dec{40}

% latitude e -tempo sideral (graus)
\def\lat{65}
\def\TS{-75}

% ângulo horário (graus)
\tikzmath{
\H = \TS - \ra;
}

\def\Az{22} % à mão, gambiarra
\def\zen{35} % à mão, gambiarra

% projeção da figura como um todo
\tdplotsetmaincoords{60}{100}

% coordenadas esféricas que serão usadas para as rotações
% ver https://texample.net/tikz/examples/the-3dplot-package/
\pgfmathsetmacro{\rvec}{1}
\pgfmathsetmacro{\thetavec}{-90+\lat}
\pgfmathsetmacro{\phivec}{90}

\begin{tikzpicture}[scale=8,tdplot_main_coords]
\path [use as bounding box] (-0.5 cm, 0.3 cm) rectangle (0.2 cm, 0.95 cm);



% origem
\coordinate (O) at (0,0,0);

% pontos cardeais no plano do horizonte
% \draw[thick] (0,0,0) -- ( 1, 0,0) node[below]{$O$}; % eixo  x
% \draw[thick] (0,0,0) -- (-1, 0,0) node[above]{$L$}; % eixo -x
% \draw[thick] (0,0,0) -- ( 0, 1,0) node[right]{$S$}; % eixo  y
% \draw[thick] (0,0,0) -- ( 0,-1,0) node[left ]{$N$}; % eixo -y

% zênite
% \draw[thick,-] (0,0,0) -- (0,0,1.0) node[above, yshift=3mm]{$Z$};

% plano do equador celeste
% \draw[fill, green, opacity=0.75] (0,0,0) circle (1);
% \draw[decoration={text along path, text={Horizonte}, text color=darkgreen, text align={left}}, decorate] (0.65,0.7) to [bend right=14] (0,1);

% inclinar por 90deg para poder plotar latitude
\tdplotsetthetaplanecoords{\phivec}

% plotar latitude
% \tdplotdrawarc[thick,gray,->,tdplot_rotated_coords]{(0,0,0)}{0.95}{-90}{-90+\lat}{left}{$\varphi$}
\tdplotdrawarc[color=black, tdplot_rotated_coords]{(1,0,0)}{-0.06}{88}{-20}{anchor=north, rotate=-30,color=black, xshift=-1mm}{$360^\circ-A$} %gambiarra


% inclinar o eixo z por -(90-DEC)
% rotacionar por TS para que o novo eixo-x seja o ponto vernal
\tdplotsetrotatedcoords{\phivec}{\thetavec}{\TS}

% mesma origem
\tdplotsetrotatedcoordsorigin{(O)}

% equador celeste
% \draw[fill, tdplot_rotated_coords, purple, opacity=0.75] (0,0,0) circle (1);
% \draw[decoration={text along path, text={Equador}, text color=darkpurple, text align={left}}, decorate] (1.45,-0.5) to [bend right=40] (-1,-0.1);

% pólo norte celeste
% \draw[thick,tdplot_rotated_coords,draw=none] (0,0,0) -- (0,0,1) node[above, yshift=2mm]{$PN$}; %eixo z'
\draw[tdplot_rotated_coords,black, fill=black] (0,0,1) circle (0.08mm) node[above] {$PN$};

% esfera celeste
% \shade[ball color=blue, opacity=0.5] (0,0,0) circle (1cm); 

% coordenadas cartesianas da estrela (nos eixos do sistema equatorial)
\tikzmath{
\xs = cos(\dec) * cos(\ra);
\ys = cos(\dec) * sin(\ra);
\zs = sin(\dec);
}

% estrela
\draw[black, draw=none, tdplot_rotated_coords] (\xs, \ys, \zs) node[]{\Large$\star$};

% ascensão reta
% \tdplotdrawarc[thick, Acolor,->,tdplot_rotated_coords]{(0,0,0)}{1}{0}{\ra}{anchor=north west,color=black}{\textcolor{Acolor}{$\alpha$}}
% \tdplotdrawarc[thick, Acolor,->,tdplot_rotated_coords]{(0,0,0)}{1}{-\TS}{\ra}{anchor=north west,color=black}{\textcolor{Acolor}{$H$}}
\tdplotdrawarc[color=black, tdplot_rotated_coords]{(0,0,1)}{0.15}{68}{28}{anchor=west,color=black, yshift=-1mm}{$H$} %gambiarra


% ponto vernal
% \draw[black, fill=black, tdplot_rotated_coords] (1,0,0) circle (0.08mm) node[below]{\Aries};

% pólo norte
\draw[black, fill=black, tdplot_rotated_coords] (0,0,1) circle (0.08mm);

% rotacionar o equador ao redor do eixo x, para poder plotar declinações
\tdplotsetrotatedthetaplanecoords{\ra}

% declinação
% \tdplotdrawarc[thick,-,tdplot_rotated_coords,color=zcolor]{(0,0,0)}{1}{90}{90-\dec}{left,color=black}{\textcolor{zcolor}{$\delta$}}
\tdplotdrawarc[thick,-,tdplot_rotated_coords,color=black]{(0,0,0)}{1}{0}{90-\dec}{below, rotate=-40, color=black}{$90^\circ-\delta$}

% meridiano
\tdplotsetthetaplanecoords{90}
% \draw[thick,white, solid, tdplot_rotated_coords] (1,0,0) arc (0:90:1);
% \draw[tdplot_rotated_coords, decoration={text along path, text={Meridiano}, text color=white, text align={center}}, decorate] (1,0) to [bend right=-44] (0,1);
\tdplotdrawarc[thick,-,tdplot_rotated_coords,color=black]{(0,0,0)}{1}{0}{-90+\lat}{anchor=south,color=black, rotate=8}{$90^\circ-\varphi$}
\tdplotsetthetaplanecoords{-90}

% azimute
% \tikzmath{
% \coszen = sin(\lat) * sin(\dec) + cos(\lat) * cos(\dec) * cos(\H);
% \zen = acos(\coszen);
% \cosAz = (cos(\lat) * sin(\dec) - sin(\lat) * cos(\dec) * cos(\H)) / sin(\zen);
% \AAz = acos(\cosAz);
% \sinA = -sin(\H) * cos(\dec) / sin(\zen);
% \Az = ifthenelse(\sinA<0, \AAz, 360-\AAz);
% }

% altura
\tdplotsetthetaplanecoords{\Az}
\tdplotdrawarc[thick,tdplot_rotated_coords,color=black]{(0,0,0)}{1}{0}{\zen+1.0}{anchor=north west,color=black, yshift=4mm}{\textcolor{black}{$z$}};
% \tdplotdrawarc[thick,tdplot_rotated_coords,color=white]{(0,0,0)}{1}{90}{\zen}{anchor=north west,color=black}{\textcolor{white}{$h$}};
\tdplotsetthetaplanecoords{\Az}

% zênite
\draw[black, fill=black] (0,0,1) circle (0.08mm) node[above] {$Z$};

% % sul
% \draw[black, fill=black] (0,0,1) circle (0.08mm);



\end{tikzpicture}

\end{document}


































