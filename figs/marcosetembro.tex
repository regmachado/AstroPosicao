\documentclass[border=0mm]{standalone}

\usepackage{preambletikz}

\begin{document}

\begin{tikzpicture}
 
\def\R{2.6}
\def\eps{23}
\tikzmath{
\xp = \R * cos(\eps);
\yp = \R * sin(\eps); 
}

\coordinate (O) at (0,0) {}; %center of the Earth

\coordinate (E1) at (\R,0) {}; %Equator (day side)
\coordinate (E2) at (\xp,\yp) {}; %cancer
\coordinate (E3) at (\xp,-\yp) {}; %capricorn

\coordinate (S1) at (3.5*\xp,0) {}; %Sun1
\coordinate (S2) at (3.5*\xp,\yp) {}; %Sun2
\coordinate (S3) at (3.5*\xp,-\yp) {}; %Sun2

% Sun rays
\draw[very thick, orange] (S1)--(E1);
\draw[very thick, orange] (S2)--(E2);
\draw[very thick, orange] (S3)--(E3);
% \draw[very thick, orange, dashed] (E1)--(0.0,0);

\def\rot{0}

% trópico de câncer
% \draw[dashed, thick, black, rotate=\rot] (\xp,\yp)--(-\xp,\yp) node[anchor=south west, xshift=1mm, yshift=-1mm, rotate=\rot] {Trópico de Câncer};

% trópico de capricórnio
% \draw[dashed, thick, black, rotate=\rot] (\xp,-\yp)--(-\xp,-\yp) node[anchor=south west, yshift=-1mm, rotate=\rot] {Trópico de Capricórnio};

% Earth
\draw[thick, black, rotate=\rot] (O) circle (\R);

% Poles
\draw[thick, black, rotate=\rot] (0,\R)--++(0,0.2) node[above, rotate=\rot] {$N$};
\draw[thick, black, rotate=\rot] (0,-\R)--++(0,-0.2) node[below, rotate=\rot] {$S$};

% Equator
\draw[thick, black, rotate=\rot] (-\R,0)--(\R,0);
\draw[draw=none, rotate=\rot] (\R,0)--(-\R,0) node[anchor=south west, yshift=-1mm, rotate=\rot] {Equador da Terra};

% person
\draw[draw=none, mygrass, fill=mygrass, rotate=\rot] (E1) node[rotate=-90, yshift=4mm] {\Strichmaxerl[3]};

% Angle: latitude sul
% \tkzMarkAngle[mark=none, draw=black, size=0.8](E1,O,E2);
% \tkzLabelAngle[dist=1](E1,O,E2){$\varphi$};

% Angle: latitude norte
% \tkzMarkAngle[mark=none, draw=black, size=0.8](E3,O,E1);
% \tkzLabelAngle[dist=1](E3,O,E1){$\varphi$};

\node[mygrass] at (6,3) [center] {{\bf Equinócio de março} ou };
\node[mygrass] at (6,2.5) [center] {{\bf Equinócio de setembro}};
% \node[mygrass] at (6,3) [center] {{\bf Solstício de dezembro}};
% \node[mygrass] at (6,3) [center] {{\bf Solstício de junho}};

\end{tikzpicture}
 
\end{document}
