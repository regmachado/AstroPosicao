\documentclass[tikz, border=1mm]{standalone}

\usepackage{preambletikz}

\begin{document}

\pagecolor{mygreen}

% ascensão reta e declinação (graus)
\def\ra{30}
\def\dec{40}

% latitude e -tempo sideral (graus)
\def\lat{65}
\def\TS{-75}

% ângulo horário (graus)
\tikzmath{
\H = \TS - \ra;
}

% projeção da figura como um todo
\tdplotsetmaincoords{70}{100}

% coordenadas esféricas que serão usadas para as rotações
% ver https://texample.net/tikz/examples/the-3dplot-package/
\pgfmathsetmacro{\rvec}{1}
\pgfmathsetmacro{\thetavec}{-90+\lat}
\pgfmathsetmacro{\phivec}{90}

\begin{tikzpicture}[scale=5,tdplot_main_coords]

% origem
\coordinate (O) at (0,0,0);

% pontos cardeais no plano do horizonte
\draw[thick, mybeige] (0,0,0) -- ( 1, 0,0) node[below]{$O$}; % eixo  x
\draw[thick, mybeige] (0,0,0) -- (-1, 0,0) node[above]{$L$}; % eixo -x
\draw[thick, mybeige] (0,0,0) -- ( 0, 1,0) node[right]{$S$}; % eixo  y
\draw[thick, mybeige] (0,0,0) -- ( 0,-1,0) node[left ]{$N$}; % eixo -y

% zênite
\draw[thick,-, mybeige] (0,0,0) -- (0,0,1.0) node[above, yshift=3mm]{$Z$};

% plano do equador celeste
\draw[thick, mybeige] (0,0,0) circle (1);

% inclinar por 90deg para poder plotar latitude
\tdplotsetthetaplanecoords{\phivec}

% plotar latitude
\tdplotdrawarc[thick,mybeige,->,tdplot_rotated_coords]{(0,0,0)}{0.8}{-90}{-90+\lat}{left}{$\varphi$}

% inclinar o eixo z por -(90-DEC)
% rotacionar por TS para que o novo eixo-x seja o ponto vernal
\tdplotsetrotatedcoords{\phivec}{\thetavec}{\TS}

% mesma origem
\tdplotsetrotatedcoordsorigin{(O)}

% equador celeste
\draw[thick, tdplot_rotated_coords, mybeige] (0,0,0) circle (1);

% pólo norte celeste
\draw[thick, mybeige,tdplot_rotated_coords,-] (0,0,0) -- (0,0,1) node[above, yshift=2mm]{$PN$}; %eixo z'

% esfera celeste
\draw[tdplot_screen_coords, mybeige, thick] (0,0) circle (\rvec); 

% coordenadas cartesianas da estrela (nos eixos do sistema equatorial)
\tikzmath{
\xs = cos(\dec) * cos(\ra);
\ys = cos(\dec) * sin(\ra);
\zs = sin(\dec);
}

% estrela
\draw[mybeige, draw=none, tdplot_rotated_coords] (\xs, \ys, \zs) node[anchor=south east, xshift=1.5mm, yshift=-1.5mm]{\Large$\star$};

% ascensão reta
\tdplotdrawarc[thick, mybeige,->,tdplot_rotated_coords]{(0,0,0)}{1}{0}{\ra}{anchor=north west,color=mybeige}{\textcolor{mybeige}{$\alpha$}}
\tdplotdrawarc[thick, mybeige,->,tdplot_rotated_coords]{(0,0,0)}{1}{-\TS}{\ra}{anchor=north west,color=mybeige}{\textcolor{mybeige}{$H$}}

% ponto vernal
\draw[mybeige, fill=mybeige, tdplot_rotated_coords] (1,0,0) circle (0.08mm) node[below]{\Aries};

% pólo norte
\draw[mybeige, fill=mybeige, tdplot_rotated_coords] (0,0,1) circle (0.08mm);

% rotacionar o equador ao redor do eixo x, para poder plotar declinações
\tdplotsetrotatedthetaplanecoords{\ra}

% declinação
\tdplotdrawarc[thick,->,tdplot_rotated_coords,color=mybeige]{(0,0,0)}{1}{90}{90-\dec}{anchor=south west,color=mybeige}{\textcolor{mybeige}{$\delta$}}

% meridiano
\tdplotsetthetaplanecoords{90}
\draw[thick,mybeige, solid, tdplot_rotated_coords] (1,0,0) arc (0:90:1);
\tdplotsetthetaplanecoords{-90}

% zênite
\draw[mybeige, fill=mybeige] (0,1,0) circle (0.08mm);

% sul
\draw[mybeige, fill=mybeige] (0,0,1) circle (0.08mm);

\end{tikzpicture}

\end{document}
