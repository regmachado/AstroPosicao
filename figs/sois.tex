\documentclass[tikz, border=1mm]{standalone}

\usepackage{preambletikz}

\begin{document}

% latitude e -tempo sideral (graus)
\def\lat{-23}

% projeção da figura como um todo
\tdplotsetmaincoords{70}{103}
% \tdplotsetmaincoords{90}{90}

% coordenadas esféricas que serão usadas para as rotações
% ver https://texample.net/tikz/examples/the-3dplot-package/
\pgfmathsetmacro{\rvec}{1}
\pgfmathsetmacro{\thetavec}{-90+\lat}
\pgfmathsetmacro{\phivec}{90}

\begin{tikzpicture}[scale=2.7,tdplot_main_coords]
\def\LX{2.3}
\def\LY{2.4}
\path [use as bounding box] (-0.5*\LX cm, -0.5*\LY cm) rectangle (0.5*\LX cm, 0.5*\LY cm);

% origem
\coordinate (O) at (0,0,0);

% pontos cardeais no plano do horizonte
\draw[thick] (0,0,0) -- ( 1, 0,0) node[below]{$O$}; % eixo  x
\draw[thick] (0,0,0) -- (-1, 0,0) node[above, xshift=1mm]{$L$}; % eixo -x
\draw[thick, draw=none] (0,0,0) -- ( 0, 1,0) node[right]{$S$}; % eixo  y
\draw[thick, draw=none] (0,0,0) -- ( 0,-1,0) node[left ]{$N$}; % eixo -y
\draw[thick, draw=none] (0,0,0) -- ( 0,0,1) node[above, yshift=2mm, xshift=1mm]{$Z$}; % eixo z


% inclinar o eixo
\tdplotsetrotatedcoords{90}{\thetavec}{270}

% mesma origem
\tdplotsetrotatedcoordsorigin{(O)}

% eixo
% \draw[very thick, -, tdplot_rotated_coords] (0,0,0) -- (0,0,1.2) node[below left]{$PN$};

% plano do horizonte
\draw[fill, greenhorizon, opacity=0.75] (0,0,0) circle (1);
\draw[decoration={text along path, text={Horizonte}, text color=mydarkgreen, text align={left}}, decorate] (0.7,0.63) to [bend right=14] (0,1);

% eixo 
% \draw[very thick, -, tdplot_rotated_coords] (0,0,0) -- (0,0,-1.2) node[above right]{$PS$};

\def\decA{0}
\def\decB{23}
\def\decC{-23}

\tikzmath{
\RA = cos(\decA);
\RB = cos(\decB);
\RC = cos(\decC);
\HA = sin(\decA);
\HB = sin(\decB);
\HC = sin(\decC);
}

% incorrect approximation:
\tikzmath{
\coseA = cos(\lat)^2 + sin(\lat)^2 * sqrt(1-(sin(\decA)/cos(\lat))^2);
\coseB = cos(\lat)^2 + sin(\lat)^2 * sqrt(1-(sin(\decB)/cos(\lat))^2);
\coseC = cos(\lat)^2 + sin(\lat)^2 * sqrt(1-(sin(\decC)/cos(\lat))^2);
\eA = acos( \coseA );
\eB = acos( \coseB );
\eC = acos( \coseC );
}

\tdplotdrawarc[thick, yellow,<-, tdplot_rotated_coords]{(0,0,\HA)}{\RA}{0-\eA}{180+\eA}{}{}
\tdplotdrawarc[thick, blue,<-, tdplot_rotated_coords]{(0,0,\HB)}{\RB}{0-\eB+20}{160+\eB}{}{}%fudge
\tdplotdrawarc[thick, red,<-, tdplot_rotated_coords]{(0,0,\HC)}{\RC}{0-\eC-1}{181+\eC}{}{} %fudge


% esfera celeste
\shade[ball color=bluesky, opacity=0.5] (0,0,0) circle (1cm); 

\end{tikzpicture}

\end{document}
