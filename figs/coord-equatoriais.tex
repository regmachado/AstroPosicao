\documentclass[tikz, border=0mm]{standalone}
 
\usepackage{preambletikz}
 
\begin{document}

\tdplotsetmaincoords{70}{100} 
\begin{tikzpicture}[tdplot_main_coords, scale=3.0]

% Estrela P
\def\ra{70}
\def\dec{60} 
\tikzmath{
\xs = cos(\dec) * cos(\ra);
\ys = cos(\dec) * sin(\ra);
\zs = sin(\dec);
\Xs = cos(\ra);
\Ys = sin(\ra);
}
\coordinate (P) at (\xs, \ys, \zs);

% equador celeste
\draw[fill, tdplot_rotated_coords, purpleequator] (0,0,0) circle (1);

% esfera celeste
\shade[ball color=bluesky, opacity=0.5] (0,0,0) circle (1cm);

% Pólos
\draw[thick, ->, tdplot_rotated_coords] ( 0, 0, 0) -- ( 0, 0, 1.2) node[above]{$PN$};
% \draw[tdplot_rotated_coords] (0,0,-1.06) node[below] {\phantom{Pólo Sul Celeste}};

% ra
\draw[black, -, thick] (0:0)--(1,0) node[below] {\Aries};
\draw[myyellow, ->, thick] (0:1) arc (0:\ra:1) node[midway, below] {$\alpha$};

% círculo horário da estrela
\tdplotsetthetaplanecoords{\ra}
\draw[myred, solid, thick] (0,0,0)--(\Xs, \Ys 0);
\draw[myred, dashed, thick, tdplot_rotated_coords] (0,0,0)--(P);

% dec
\draw[myred, ->, thick, tdplot_rotated_coords] (90:1) arc (90:90-\dec:1) node[midway, right]{$\delta$};

\draw[white, draw=none] (P) node[]{$\star$};
\draw[black, fill=black] (1,0,0) circle (0.1mm);

\end{tikzpicture}
 
\end{document}
