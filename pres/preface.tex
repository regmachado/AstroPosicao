\chapter*{Prefácio}
\addcontentsline{toc}{chapter}{Prefácio}

Esta apostila foi preparada para um curso de extensão a ser realizado na UTFPR Campus Curitiba em março de 2023. Não se trata de um curso completo de Astronomia de Posição: aqui estão presentes apenas alguns tópicos selecionados. A finalidade da apostila é servir como um resumo ou guia de estudos para os estudantes que forem acompanhar o curso. O público-alvo principal são estudantes de Física, em qualquer semestre da graduação.

O planejamento do curso prevê 8 aulas teóricas e, paralelamente, 4 atividades práticas. As aulas teóricas estão divididas em 3 partes. Na primeira parte (Capítulos 1 a 4), serão introduzidos os conceitos relativos à esfera celeste e os sistemas de coordenadas. Na segunda parte (Capítulos 5 e 6), estudaremos transformações de sistemas de coordenadas usando trigonometria esférica ou matrizes de rotação. Na terceira parte (Capítulos 7 e 8), veremos como escrever um programa computacional que calcula as transformações, usando a linguagem de programação Python. As atividades observacionais (Capítulos 9 a 12) consistem em fazer medidas --- do Sol durante o dia, e de estrelas durante a noite.

A principal referência usada para preparar este curso foi o livro do professor Roberto Boczko, \textit{Conceitos de Astronomia}, que recentemente foi disponibilizado na página do IAG/USP.\footnote{\url{https://www.iag.usp.br/astronomia/sites/default/files/conceitos_astronomia.pdf}} Outra referência muito útil e amplamente utilizada nesse tema são as notas de aula de \textit{Astronomia de Posição} do professor Gastão B.~Lima~Neto, que são atualizadas periodicamente e cuja versão mais recente pode ser encontrada em sua página.\footnote{\url{http://www.astro.iag.usp.br/~gastao/astroposicao.html}}

Mesmo quem se dedica à astronomia teórica ou computacional não pode ignorar o vocabulário das observações. Por isso, a astronomia esférica, com sua terminologia arcana, ainda é indispensável para qualquer ramo do nosso ofício, desde o sistema solar até a cosmologia. E estudar os movimentos na esfera celeste continua sendo uma boa porta de entrada para a astronomia como um todo.

Esta apostila está disponível no seguinte link, onde também podem ser encontrados os códigos-fonte em \LaTeX\ usados para gerar o texto e as figuras vetoriais aqui presentes:

\begin{center}
\url{https://github.com/regmachado/AstroPosicao/}
\end{center}

\begin{flushright}
\noindent Rubens E. G. Machado\\
Universidade Tecnológica Federal do Paraná\\
\texttt{rubensmachado@utfpr.edu.br}
\end{flushright}
