\chapter{Movimento anual do Sol}

Neste capítulo veremos como o movimento diário do Sol muda ao longo do ano.
 
\section{Solstícios e equinócios}

As datas de início das quatro estações do ano correspondem a momentos especiais, chamados de solstícios e equinócios. As datas aproximadas costumavam ser:

\begin{center}
\begin{tabular}{ll}
\hline
21 de março & equinócio de outono \\
21 de junho & solstício de inverno  \\
23 de setembro & equinócio de primavera \\
21 de dezembro & solstício de verão \\
\hline
\end{tabular}
\end{center}

\noindent para o hemisfério Sul. Já para as datas do hemisfério Norte, basta trocar verão$\leftrightarrow$inverno e primavera$\leftrightarrow$outono. O correto é sempre dizer explicitamente a que hemisfério se está referindo. No âmbito deste texto, quando faltar o complemento, fica subentendido que estamos falando na nossa realidade de habitantes do hemisfério Sul.

Veremos a seguir várias propriedades que variam ao longo das estações do ano, começando pela duração do dia. Para evitar confusão, às vezes é usado o termo `dia claro', significando o período em que o Sol está acima do horizonte. Os equinócio são os dois dias do ano em que a duração do dia claro é igual à duração da noite: 12 horas cada. De março a junho, as noites vão ficando progressivamente mais longas, até que no solstício de inverno temos a noite mais longa do ano. De junho até setembro, o dia claro volta a avançar, até que no equinócio de primavera temos novamente 12 horas de dia claro e 12 horas de noite. De setembro até dezembro, o dia claro continua avançando, até que no solstício de verão temos o mais longo dia claro do ano; é a noite mais curta. A partir daí, as noites passam a avançar, até chegar no equinócio de outono e recomeçar o ciclo.

Na prática, isso significa que os horários do nascer do Sol (\textbf{a aurora}) e do pôr do Sol (\textbf{o ocaso}) mudam ao longo do ano. Idealmente, poderíamos imaginar que nos dias dos equinócios, o Sol nasce às 6\,h e se põe às 18\,h. Já no verão (dias claros longos), o Sol nasce antes das 6\,h e se põe depois das 18\,h. No inverno (noites longas), nasce depois das 6\,h e se põe antes das 18\,h. Na realidade, os horários exatos da aurora e do ocaso não são simétricos como se poderia crer, mas são complicados por fatores que não abordaremos agora e que têm relação com a excentricidade da órbita da Terra. De qualquer forma, a variação na duração da noite é perceptível ao longo do ano, mesmo nas nossas latitudes moderadas de cerca de $20^{\circ}$. Em altas latitudes, a diferença é muito mais pronunciada.

É também interessante notar a relação entre as datas e o clima. Por exemplo, o dia 21 de junho é a noite mais curta do ano, ou seja, o dia com a mínima insolação. No entanto, 21 de junho não é necessariamente o dia mais \textit{frio} do ano. Da mesma forma, o dia 21 de dezembro --- máxima insolação --- não é o auge do verão, mas sim o primeiro dia do verão. No segundo dia do verão, o dia claro já começa a ceder lugar à noite, e assim por diante. Ocorre que a Terra não é simplesmente uma esfera uniforme recebendo luz solar. O clima de uma região é também regulado por diversos fatores locais como vegetação, relevo, regime de chuvas e ventos, proximidade do litoral, etc. Existe uma espécie de atraso térmico, já que as grandes massas d'água levam um tempo para conseguir se aquecer ou resfriar como um todo. Resulta que as datas de início das estações calham de funcionar bem, pois em muitos lugares esse atraso faz com que as altas temperaturas acabem aconteçendo por volta no meio do verão etc. Mas isso tudo depende bastante da região em questão. Em particular, os hemisférios da Terra são bastantes assimétricos nesse respeito, já que no hemisfério Sul a fração de área coberta por oceanos é bem maior.

É também curioso perceber que as quatro estações do ano não têm exatamente a mesma duração (conte os dias em um calendário). No hemisfério Sul, temos 186 dias de outono+inverno contra 179 dias de primavera+verão. No hemisfério Norte é o oposto. Ao longo da sua órbita elíptica ao redor do Sol, a velocidade orbital da Terra varia, sendo mais alta velocidade no periélio (janeiro), e mais baixa velocidade no afélio (julho). O periélio e o afélio não têm relação com os solstícios e equinócios.

Por fim, é sempre pertinente ressaltar que a existência das estações do ano é uma conseqüência da inclinação do eixo de rotação da Terra, e não da distância da Terra ao Sol. A excentricida da órbita da Terra é muito pequena. A distância média da Terra ao Sol (que é a Unidade Astronômica) vale aproximadamente 150 milhões de km. Ela varia de cerca de 0.98\,AU no periélio para cerca de 1.02\,AU no afélio.

\section{Trópicos de Câncer e Capricórnio}

\begin{figure}
\centering
\includestandalone{figs/dezembro}
\includestandalone{figs/marcosetembro}
\includestandalone{figs/junho}
\caption{Configuração dos raios de Sol ao meio-dia nos dias dos solstícios e equinócios.}
\label{fig:marcosetembro}
\end{figure}

Vamos considerar agora como observadores em diferentes latitudes recebem os raios de Sol nos quatro dias especiais do ano --- especificamente ao meio-dia verdadeiro. Para isso, precisamos momentaneamente sair do enquadramento da esfera celeste e enxergar a Terra como uma esfera em rotação na qual incidem raios de luz provenientes do Sol, pararelos entre si.

O primeiro painel da Fig.~\ref{fig:marcosetembro} mostra a configuração da Terra recebendo raios solares no dia do solstício de verão do hemisfério Sul. No nosso verão, naturalmente é o hemisfério Sul aquele que está mais voltado para o Sol, enquanto o hemisfério Norte recebe raios de Sol mais oblíquos. Um observador que estivesse no equador da Terra ao meio-dia não veria o Sol no seu zênite, mas sim com uma certa inclinação. Porém, no hemisfério Sul, existe uma latitude tal que a vertical do observador coincida com os raios de Sol neste momento. Essa latitude define o Trópico de Capricórnio e vale aproximadamente $-23^{\circ}$. Um observador ao Sul do Trópico de Capricórnio nunca verá o Sol no zênite.

No segundo painel da Fig.~\ref{fig:marcosetembro} está representada a configuração em qualquer um dos dois equinócios, seja primavera ou verão. Nesses dois dias de alta simetria, a Terra está igualmente iluminada nos dois hemisférios. Portanto, um observador que esteja sobre o equador terrestre, receberá os raios de Sol verticalmente ao meio-dia.

O teceiro painel da Fig.~\ref{fig:marcosetembro} representa o solstício de inverno do hemisfério Sul, quando é o Norte que está mais voltado para o Sol. Analogamente ao primeiro caso, existe uma latitude Norte onde o observador receberá o Sol ao longo do seu zênite ao meio-dia. Essa latitude vale $+23^{\circ}$ e define o Trópico de Câncer. Um observador ao Norte do Trópico de Câncer nunca verá o Sol no zênite. A região da terra contida entre os trópicos de Câncer e Capricórnio é chamda de zona tropical. É fácil se convencer geometricamente que este ângulo também é a inclinação entre o eixo de rotação da Terra e a linha perpendicular ao plano da órbita.

\section{Movimento diário do Sol}

\begin{wrapfigure}{r}{0pt}
\includestandalone{figs/sois}
\caption{Trajetórias do Sol no solstício de verão (vermelho), equinócios (amarelo) e solstício de inverno (azul).}
\label{fig:sois}
\end{wrapfigure}

Agora vamos analisar os solstícios e equinócios do ponto de vista de como um observador vê a trajetória diária do Sol. Na Fig.~\ref{fig:sois} está representada a esfera celeste de um observador localizado no Trópico de Capricórnio. O pólo celeste Sul não está mostrado, mas ele está $23^{\circ}$ acima do ponto cardeal Sul. A linha amarela corresponde à trajetória do Sol nos dias dos equinócios (primavera ou outono). Nessas datas, o Sol está no equador celeste. De fato, a linha amarela é o equador celeste. A linha vermelha é a trajetória do Sol no solstício de verão, quando o Sol está $23^{\circ}$ ao Sul do equador celeste. A linha azul é a trajetória do Sol no solstício de inverno, quando o Sol está $23^{\circ}$ ao Norte do equador celeste.

Note que os três arcos são paralelos entre si. Como o arco amarelo é o equador celeste, ele é um semi-círculo, o que significa que nos dias dos equinócios, o Sol fica 12\,h acima do horizonte. Já o arco vermelho tem mais que $180^{\circ}$, pois no verão o Sol fica mais que 12\,h acima do horizonte. O arco azul tem menos que $180^{\circ}$, pois no inverno o Sol fica menos que 12\,h acima do horizonte.

Apenas nos equinócios o Sol nasce exatamente no ponto cardeal Leste e se põe exatamente no ponto cardeal Oeste. No nosso verão, o Sol está aqui no Sul e por isso ele nasce e se põe em pontos mais ao Sul da linha Leste-Oeste. Já no nosso inverno, o Sol está ao Norte, e por isso ele nasce e se põe em pontos mais ao Norte da linha Leste-Oeste. A longo do ano as trajetórias do Sol na Fig.~\ref{fig:sois} ficam oscilando entre os extremos azul (inverno) e vermelho (verão). Indo do inverno para o verão, passa pelo equinócio de primavera; voltando do verão para o inverno, passa pelo equinócio de outono.

Consideremos agora as sombras projetadas por um gnômon ao meio-dia, lembrando que a Fig.~\ref{fig:sois} é um caso particular do observador que está na latitude $-23^{\circ}$. No solstício de verão, o Sol passa pelo zênite ao meio-dia --- é o único momento sem sombra do ano. Nos equinócios, o Sol do meio-dia passa a $23^{\circ}$ do zênite, projetando uma sombra que aponta para o Sul. No solstício de inverno, o Sol do meio-dia é o mais baixo do ano e projeta uma longa sombra em direção ao Sul.

Caso o observador estivesse em latitudes mais ao Sul do que o Trópico de Capricórnio, o pólo celeste estaria mais alto e conseqüentemente os três arcos da figura estariam menos inclinados com relação ao horizonte. Como resultado, o Sol do meio-dia nunca atingiria o zênite, nem mesmo no verão. Por outro lado, para um observador em algum lugar dentro da zona tropical, o Sol passa duas vezes pelo zênite: uma vez indo para o solstício de verão e outra voltando.

\section{Eclíptica e zodíaco}

Por fim, veremos como o Sol se comporta com relação às estrelas fixas. Para entender essa relação, vale a pena passar por um momento para a perspectiva heliocêntrica, isto é, pensar em termos da órbita da Terra ao redor do Sol. Estando na Terra, e olhando para o céu, digamos que num dado momento o Sol aparente estar na direção da constelação de Áries. Já no mês seguinte, a Terra terá avançado um pouco na sua órbita e agora o Sol aparentará estar na direção da constelação de Touro, e assim por diante ao longo dos 12 meses do ano, até retornar à constelação de Áries.

Voltando agora para a perspectiva geocêntrica da astronomia de posição, temos que o Sol percorre um caminho na esfera celeste ao longo de um ano. Esse percurso do Sol se chama \textbf{eclíptica} e é um grande círculo na esfera celeste, que está inclinado com relação ao equador por um ângulo de $\epsilon = 23^{\circ}$, chamado de \textbf{obliqüidade da eclíptica}. Fisicamente, esse ângulo pode ser entendido como a inclinação entre o equador terrestre e o plano da órbita da Terra. 

Lembramos que o desenho das constelações é fixo e gira como um todo uma vez por dia ao redor da Terra. Já o Sol não é fixo com relação às estrela; ele vai avançando gradualmente ao longo da eclíptica. Para dar uma volta completa na esfera celeste em um ano, o Sol percorre aproximadamente ${\sim}1^{\circ}$ por dia (isto é, $360^{\circ} / 365$ dias). Ao longo dos meses, esse deslocamente é notável. Mas na escala de tempo curta de um único dia, o movimento é pouco apreciável. Então é como se --- a cada dia --- o Sol se comportasse como uma dada estrela fixa.

\begin{wrapfigure}{R}{0pt}
\includestandalone{figs/ecliptica}
\caption{Relação entre a eclíptica e o equador celeste.}
\label{fig:ecliptica}
\end{wrapfigure}

Historicamente, as constelações serviram como referência para esse acompanhamento sistemático do caminho anual do Sol. Porém as constelações têm diferentes extensões angulares. Já na antigüidade, desde o tempo dos babilônios, 12 constelações acabaram sendo codificadas nos \textbf{signos do zodíaco}. Cada um dos 12 signos corresponde a um arco de $30^{\circ}$ ao longo da eclíptica. Assim, a circunferência completa fica divida em 12 partes iguais, que recebem os nomes de: Áries, Touro, Gêmeos, Câncer, Leão, Virgem, Libra, Escorpião, Sagitário, Capricórnio, Aquário e Peixes. Apesar da motivação original em termos das constelações, o fenômeno da precessão dos equinócios ao longo dos milênios faz com que os nomes dos signos estejam atualmente bastante defasados com relação às constelações. Mesmo assim, continuamos usando a nomenclatura dos signos do zodíaco na astronomia.

Podemos agora conectar as datas especiais do ano com as posições do Sol na eclíptica. Na Fig.~\ref{fig:ecliptica} vemos que a eclíptica intercepta o equador em dois pontos: estes são os equinócios. Em junho, o Sol está bem ao Norte, no que seria o signo de Câncer (\Cancer): este é o solstício de verão boreal. Em dezembro, o Sol está bem ao Sul, em Capricórnio (\Capricorn): é o solstício de inverno boreal. Quando o Sol intercepta o equador, indo do inverno boreal para o verão boreal, temos o equinócio de primaveral boreal: nesse momento, 21 de março, o Sol está no signo de Áries (\Aries), que era considerado o começo do ano em certos calendários da antigüidade. Veremos no próximo capítulo que esse ponto é uma referência importante para os sistemas de coordenadas. Esse ponto (\Aries) recebe os nomes de: \textbf{Primeiro Ponto de Áries}, ou \textbf{Ponto Vernal} (já que é a primavera do hemisfério Norte), ou ainda \textbf{Ponto Gama} (pois o símbolo do carneiro se assemelha à letra grega $\gamma$). Por fim, seis meses depois de Áries, quando o Sol cruza novamente o equador celeste, dessa vez descendo do Norte para o Sul, esse é o equinócio de outono boreal, em Libra (\Libra).
