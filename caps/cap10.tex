\chapter{Determinação da latitude geográfica}

\begin{figure}[ht]
\centering
\includestandalone{figs/equinocios}
\caption{Raios de Sol durante o equinócio para um observador no hemisfério Sul ao meio-dia verdadeiro.}
\label{fig:equinocios}
\end{figure}

Esta atividade prática consiste em determinar a latitude geográfica. Precisa ser realizada no dia do equinócio (primavera ou outono), ao meio-dia verdadeiro. Aqui precisaremos medir o tamanho do gnômon e o comprimento de sua sombra.

A Fig.~\ref{fig:equinocios} representa a configuração da Terra e dos raios solares ao meio-dia do equinócio. Nesse momento, o Sol está no equador celeste. Então um observador localizado no equador da Terra receberia os raios solares verticalmente. Considere um observador, neste mesmo momento, localizado em um ponto $P$, com uma latitude qualquer $\varphi$. Na Fig.~\ref{fig:equinocios}, é fácil perceber que o ângulo $\varphi$ é igual ao ângulo entre o Sol e o zênite desse observador. Em outras palavras: a distância zenital do Sol é a latitude geográfica (ou melhor, seu módulo).

\begin{wrapfigure}[18]{r}{0pt}
\includestandalone{figs/zenital}
\caption{Distância zenital do Sol.}
\label{fig:zenital}
\end{wrapfigure}

O procedimento prático é tão simples quanto o cálculo em si. Conforme a Fig.~\ref{fig:zenital}, basta medir o comprimento do gnômon ($l$) e o comprimento da sombra ($s$). A distância zenital do Sol será:
%
\begin{equation}
\tan z = \frac{s}{l}
\end{equation}
%
e, naquele momento, $\varphi = z$. Assim fica determinada a latitude geográfica. Embora não haja muita dificuldade prática em medir o comprimento de uma sombra, a sutileza deste método é saber o momento correto do meio-dia verdadeiro.

Para fins desta atividade, uma maneira prática de contornar tal dificuldade consistiria em tomar uma série de medidas. Um pouco antes do presumido meio-dia, começamos a tomar medidas da sombra do gnômon, separadas entre si por alguns minutos, digamos. Se continuarmos fazendo isso para além do meio-dia verdadeiro, ficará claro nos dados que uma das medições corresponde à mínima sombra. É aquela que deve ser usada.

Além disso, se a atividade do capítulo anterior tiver sido feita previamente, já teremos uma marcação da linha meridiana no chão. O instante de sombra mínima deve acontecer quando a sombra do gnômon estiver alinhada com a linha meridiana.

Mesmo assim, é interessante entender por quais motivos a mínima sobra ocorreu com alguns minutos de defasagem do meio-dia do relógio.





