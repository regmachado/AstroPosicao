\chapter{Determinação da linha meridiana}
\label{cap:meridiano}

\begin{wrapfigure}{r}{0pt}
\centering
\includestandalone{figs/linhaNS}
\caption{Determinação da linha Norte-Sul.}
\label{fig:linhaNS}
\end{wrapfigure}

O objetivo desta atividade prática é determinar o meridiano local, isto é, a direção Norte-Sul. O método consiste em fazer observações da direção da sombra do Sol ao redor do meio-dia. Este método é interessante pois prescinde de qualquer instrumento de medida ou qualquer acesso a informações externas. Não precisamos usar relógio, nem régua, nem bússola. Não é preciso consultar nenhuma informação em livros ou na internet. Não é preciso saber a latitude e nem sequer o dia do ano. Basta usar uma estaca vertical (gnômon) e um barbante.

Neste contexto, estamos entendendo por meio-dia o meio-dia verdadeiro, não o da hora civil. O instante de menor sombra do dia é o momento em que o Sol passa pelo meridiano local. De manhã, a sombra aponta para o Oeste; à tarde a sombra aponta para o Leste.

O procedimento é:

\begin{enumerate}

\item Num momento qualquer \textit{antes} do meio-dia, observe a sombra projetada no solo e faça uma marcação no chão de onde está a extremidade da sombra. É o ponto $A$ na Fig.~\ref{fig:linhaNS}.

\item Trace no chão um círculo de raio $IA$. Isso poderia ser feito estendendo um barbante desde a base do gnômon até a extremidade da sombra.

\item Em algum momento \textit{depois} do meio-dia, a extremidade da sombra interceptará o círculo desenhado. Quando isso acontecer, marque o ponto $A'$.

\item A bissetriz do ângulo $A\hat{I}A'$ é a direção Norte-Sul. Para saber qual direção é o Norte, basta notar que à tarde o Sol estará no lado Oeste.

\end{enumerate}

Uma maneira de melhorar a precisão seria marcar mais de um ponto no lado da manhã ($B$, $C$, etc), bem como suas contrapartidas à tarde ($B'$, $C'$, etc). A média dessas bissetrizes tende a ser mais precisa do que uma única medida.

Conceitualmente, seria suficiente passar um único traço no momento de menor sombra. Ocorre que, na prática, é difícil ter certeza de quando exatamente esse instante acontece. Inevitavelmente acabaríamos marcando algumas sombras antes e outras depois, para ter confiança de qual foi a mínima sombra. Então vale a pena lançar mão da simetria manhã-tarde.

Por fim, convém ressaltar que os momentos $A$ e $A'$ são simétricos com relaçao ao meio-dia verdadeiro, não o meio-dia da hora civil. Por exemplo, não serão horários como 11\,h\,00\,min e 13\,h\,00\,min, justamente porque o meio-dia verdadeiro não ocorrerá às 12\,h\,00\,min. Este método é engenhoso justamente porque evita ter que lidar com horários.

