\chapter{Esfera celeste}

Neste capítulo e no próximo estudaremos alguns conceitos da esfera celeste. Esfera celeste é o céu. Quando olhamos para cima numa noite estrelada em um campo aberto, temos a impressão de estarmos sob uma abóbada, como se fosse um teto hemisférico. A outra metade da esfera não é visível pois está abaixo do solo.

\section{Horizonte e zênite}

\begin{wrapfigure}{r}{0pt}
\includestandalone{figs/horizonte}
\caption{Esfera celeste mostrando o plano horizontal de um observador e seu zênite.}
\label{fig:horizonte}
\end{wrapfigure}

Começamos com a definição de alguns elementos locais, isto é, que são propriedades de cada observador.

O plano horizontal é essencialmente o `chão' do observador (o plano verde da Fig.~\ref{fig:horizonte}). Onde o céu aparenta tocar o solo, teríamos a linha do horizonte. Na prática, acidentes do relevo, vegetação ou construções geralmente bloqueiam a vista desse círculo que seria o horizonte. Sobre a linha do horizonte estão os quatro pontos cardeais: Norte, Sul, Leste e Oeste.

Se traçarmos uma linha vertical partindo do observador, ela interceptará o céu no ponto chamado de zênite. Em outras palavras, o zênite é simplesmente o ponto acima da cabeça do observador.

\section{Meridiano local}

\begin{wrapfigure}{r}{0pt}
\includestandalone{figs/planomeridiano}
\caption{O plano meridiano de um observador contém a linha Norte-Sul e o zênite.}
\label{fig:planomeridiano}
\end{wrapfigure}

Uma maneira de definir o plano meridiano é dizer que ele contém estes três pontos: o Norte, o Sul e o zênite. Naturalmente contém o observador também. A intersecção desse plano com a esfera celeste é um grande círculo chamado de meridiano local. Veremos que o meridiano local tem muita utilidade na descrição dos movimentos dos astros.

A primeira delas diz respeito ao movimento diário do Sol e sua relação com os pontos cardeais. Veremos mais adiante que o Sol só nasce no ponto cardeal Leste em certos dias especiais do ano. Idem para o pôr do Sol, que só se dá no ponto cardeal Oeste em dias especiais. No entanto, continua sendo verdade que o Sol nasce no \textit{lado Leste} e se põe no \textit{lado Oeste} do horizonte. Então, se precisássemos determinar os pontos cardeais em um dia qualquer, não seria útil tentar usar a direção exata do nascer e do pôr do Sol.

Mas existe uma propriedade que é útil o ano todo: o instante de menor sombra do dia. Esse momento é o meio-dia verdadeiro (ou meio-dia solar). Consideremos a sombra de uma haste vertical (um gnômon): de manhã, com o Sol vindo do Leste, a sombra apontará para o lado Oeste; já à tarde, com o Sol descendo para o Oeste, a sombra apontará para o Leste. Portanto, em um instante intermediário --- o meio-dia verdadeiro --- a sombra estará ao longo da direção Norte-Sul, também chamada de linha meridiana. Esta é a mínima sombra do dia. É verdade que, ao longo do ano, o comprimento da sombra do meio-dia muda, sendo mais curta no verão e mais longa no inverno. Mas independentemente do comprimento, a direção da mínima sombra do dia sempre define a direção da linha Norte-Sul. No Capítulo~\ref{cap:meridiano}, apresenta-se uma descrição mais detalhada de um método prático para essa determinação.

\section{Norte magnético}

É pertinente ressaltar que, na astronomia, o Norte que usamos é sempre o geográfico, não o magnético. O norte magnético, lido na prática em uma bússola, pode diferir consideravelmente do norte geográfico. A diferença entre os dois se chama declinação magnética. A correção não é trivial, pois a declinação magnética depende da localização do observador na superfície da Terra. Mas não se trata de um cálculo teórico simples que possa ser feito usando meramente a latitude. Isso ocorre porque o campo magnético do planeta é intrinsecamente assimétrico. Portanto, para fazer a correção é preciso consultar um mapa específico. Além disso, o mapa precisa ser recente, pois o campo magnético muda com o tempo, mesmo na escala de anos. Tipicamente, a declinação magnética pode variar da ordem de alguns graus por década, dependendo da região.

Atualmente, para um observador em Curitiba, a declinação magnética vem a ser de aproximadamente $-20^{\circ}$. O sinal negativo, neste caso, significa que a bússola aponta mais para o Oeste. Então, estando em Curitiba, alguém que faça uma leitura de uma bússola na prática precisa lembrar que o Norte verdadeiro está cerca de 20$^{\circ}$ \textit{para o Leste} do que indica a agulha. De acordo com modelos atuais, este valor tenderá a aumentar cerca de 1$^{\circ}$ por década.\footnote{\url{https://www.ngdc.noaa.gov/geomag/calculators/magcalc.shtml}}


\section{O que é a esfera celeste}

Considere uma paisagem cotidiana composta de elementos como pessoas, árvores, prédios e montanhas. Pode ser uma fotografia ou uma paisagem vista pela janela. A montanha é muito grande, mas talvez ocupe no seu campo visual o mesmo tamanho aparente que um prédio que, apesar de ser bem menor, está bem mais perto. Ou seja, nós geralmente conseguimos ter uma interpretação bastante adequada das diferentes distâncias. Isso funciona graças a alguns fatores: temos familiaridade com os tamanhos reais desses corpos quando vistos de perto; temos uma noção de perspectiva de como os corpos se apresentam quando vistos sob diferentes ângulos e sob diferentes condições de iluminação; como esses corpos têm extensão, o tamanho angular de cada um serve de referência para julgar as distâncias dos demais que compõem a paisagem. Em suma, nossa intuição visual é treinada pela experiência prévia que temos com o mundo a nossa volta, nas escalas de tamanho humanas. Nada disso funciona no caso do céu noturno.

Quando olhamos para o céu noturno, não temos como interpretar visualmente que as estrelas estão em diferentes distâncias. Cada estrela é praticamente uma fonte puntiforme de luz, sem tamanho discernível. São vistas contra um fundo escuro e sem referências relativas. Suas distâncias são estupendas e incompreensíveis para nossa intuição visual. Resulta que enxergamos esses pontos de luz como se estivessem todos à mesma distância. Por isso, a esfera celeste aparenta ser uma esfera, ainda que de raio indeterminado.

A imagem pode parecer um pouco incômoda: é como se vivêssemos no interior de uma grande uma casca esférica, centrada na Terra. No entanto, o geocentrismo dessa descrição não é motivo para preocupação, justamente por não passar disso --- uma descrição. Não é um geocentrismo em termos da estrutura física do Universo, mas apenas da escolha de onde é a origem. É meramente uma questão de referencial. E não existem referenciais certos ou errados; todos os referenciais são igualmente legítimos. Temos a liberdade de escolher qual referencial é o mais conveniente, dependendo do que se pretende estudar. Quase todas as observações astronômicas são feitas a partir da superfície da Terra. É, portanto, perfeitamente natural descrever as posições e movimentos dos astros a partir de um referencial centrado no observador --- no caso, centrado na Terra. Por isso, continua sendo conveniente adotar a descrição geocêntrica que sempre foi usada na astronomia de posição. Essa ainda é a maneira mais apropriada de lidar com dados observacionais.

Daí em diante, é sempre possível calcular mudanças de referencial, conforme a necessidade. Por exemplo, para estudar órbitas de planetas e cometas, será apropriado passar para um referencial centrado no Sol; para estudar a dinâmica das estrelas da Via Láctea, será apropriado passar para um referencial cuja origem é o centro da Galáxia, e assim por diante. Nesses casos, seria necessário também conhecer as distâncias. Métodos de determinação de distância são um tópico importantíssimo na astronomia.

\section{Pólos celestes e equador celeste}

\begin{wrapfigure}{r}{0pt}
\includestandalone{figs/equador}
\caption{Esfera celeste mostrando o equador celeste e os pólos celestes.}
\label{fig:equador}
\end{wrapfigure}

Devemos imaginar que a Terra está no centro da esfera celeste. A Terra tem seu eixo de rotação, que passa pelos pólos geográficos Norte e Sul. Se esse eixo for prolongado, ele interceptará a esfera celeste em dois pontos: o Pólo Norte Celeste e o Pólo Sul Celeste, conforme a Fig.~\ref{fig:equador}. A Terra tem seu equador geográfico, que é perpendicular ao eixo de rotação. Da mesma forma, se expandido, o plano do Equador da Terra interceptará a esfera celeste em um grande círculo chamado de Equador Celeste.

Na astronomia de posição, a Terra é estacionária, sem rotação. É a esfera celeste que gira, de Leste para Oeste, ao redor do eixo de rotação definido pelos pólos celestes. Ignorando efeitos de longo prazo, podemos considerar que as constelações são essencialmente fixas. Ou seja, todas as estrelas giram juntas, sem mudar as distâncias relativas entre elas. É como se o desenho das constelações fosse um padrão permanentemente estampado na face interna da esfera celeste. A esfera como um todo gira, mas o padrão não se deforma. É nesse sentido que usamos a expressão `estrelas fixas'. 

\section{Latitude geográfica}

Um observador no hemisfério Norte da Terra verá o pólo celeste Norte permanentemente acima de seu horizonte; equivalentemente, um observador do hemisfério Sul verá sempre o pólo celeste Sul. Para um dado observador, o pólo celeste visível é fixo e o céu inteiro aparenta girar ao redor do pólo. No caso do hemisfério Norte, por acaso existe uma estrela bem próxima da direção do pólo celeste Norte --- é a estrela $\alpha$ da constelação da Ursa Menor, conhecida como estrela Polar, ou Polaris. Na proximidade do pólo celeste Sul, não há uma estrela suficientemente brilhante para servir de referência. É comum usar os termos \textbf{boreal}, que significa do hemisfério Norte; e \textbf{austral}, que significa do hemisfério Sul.

\newpage

\begin{wrapfigure}{r}{0pt}
\includestandalone{figs/latitude}
\caption{Latitude geográfica.}
\label{fig:latitude}
\end{wrapfigure}

A altura do pólo celeste depende da latitude geográfica do observador. Para compreender essa configuração geométrica, precisamos considerar momentaneamente a extensão da Terra. Na Fig.~\ref{fig:latitude}, temos um observador no hemisfério Norte. O plano horizontal desse observador é tangente à superfície da Terra naquele ponto. A vertical do observador, prolongada até o centro da Terra, faz um ângulo $\varphi$ com o equador terrestre. Esse ângulo é a latitude geográfica, que varia no intervalo $-90^{\circ} < \varphi < +90^{\circ}$, sendo positiva no hemisfério Norte e negativa no hemisfério Sul. É fácil notar na Fig.~\ref{fig:latitude} que $\varphi$ também é o ângulo entre o horizonte e o pólo celeste Norte. A única sutileza aqui é reparar que parece haver duas linhas paralelas que apontam para o Norte: uma passando pelo centro da Terra e uma passando pelo observador. Ocorre que o raio da Terra é desprezível diante da esfera celeste, então ambas apontam efetivamente para a mesma direção. Resulta que (em módulo):

\begin{quote}
\textit{A latitude geográfica do observador é igual à altura do pólo celeste visível.}
\end{quote}

\begin{wrapfigure}[18]{r}{0pt}
\includestandalone{figs/ambas}
\caption{Esfera celeste mostrando o horizonte e o equador celeste.}
\label{fig:ambas}
\end{wrapfigure}

Voltando para a representação da esfera celeste na Fig.~\ref{fig:ambas}, podemos ver que o ângulo entre o zênite e o pólo celeste visível é $90^{\circ}-\varphi$. O ângulo entre o horizonte e equador celeste também é $90^{\circ}-\varphi$. O pólo celeste Norte fica acima do ponto cardeal Norte. Se fosse um observador no hemisfério Sul, veríamos o pólo celeste Sul acima do ponto cardeal Sul.

\begin{figure}
\includestandalone{figs/noequador}%
\includestandalone{figs/nopolonorte}\\
\includestandalone{figs/nonorte}%
\includestandalone{figs/nosul} 
\caption{As curvas vermelhas representam o movimento de algumas estrelas, para observadores localizados respectivamente: no equador, no pólo Norte, no hemisfério Norte, ou no hemisfério Sul da Terra.}
\label{fig:no}
\end{figure}

Agora estamos em condições de compreender exatamente como as estrelas se movem ao longo da noite, para observadores em qualquer latitude. Na Fig.~\ref{fig:no}, vamos considerar cada um dos casos, começando por um observador localizado no equador da Terra, isto é, com latitude geográfica $\varphi=0^{\circ}$. Nesse caso, a altura do pólo seria nula, então o eixo da esfera celeste encontra-se no plano do horizonte. Os pontos cardeais Norte e Sul coincidem com os pólos celestes Norte e Sul. A~esfera celeste como um todo gira ao redor desse eixo. O primeiro painel da Fig.~\ref{fig:no} mostra a trajetória de algumas estrelas para o observador que está no equador. Todas as estrelas da esfera celeste têm a oportunidade de nascer e se pôr ao longo do ano.

No segundo painel da Fig.~\ref{fig:no}, temos o caso extremo do observador que está no pólo Norte da Terra, isto é, com latitude geográfica $\varphi=+90^{\circ}$. O zênite desse observador coincide com o pólo celeste Norte. Neste caso peculiar, as estrelas não nascem nem se põem, mas continuam girando de Leste para Oeste. O hemisfério celeste Sul fica permanentemente abaixo do horizonte e nunca é visível.

No terceiro painel da Fig.~\ref{fig:no}, temos o caso do observador que está em alguma latitude intermediária qualquer do hemisfério Norte: $0^{\circ} < \varphi < 90^{\circ}$. O pólo celeste visível é o Norte e ele é visto acima do ponto cardeal Norte. As estrelas se movem de Leste para Oeste em arcos paralelos ao equador celeste e portanto inclinados com relação ao horizonte. Algumas estrelas próximas ao pólo celeste Norte nunca descem abaixo do horizonte --- são estrelas circumpolares. Algumas estrelas próximas ao pólo celeste Sul não são visíveis nunca para esse observador, pois ficam sempre abaixo do horizonte.

Finalmente, no quarto painel da Fig.~\ref{fig:no}, temos o caso do observador que está em alguma latitude intermediária qualquer do hemisfério Sul: $-90^{\circ} < \varphi < 0^{\circ}$. A situação é análoga à anterior, com a diferença de que o pólo visível é visto acima do ponto cardeal Sul.


