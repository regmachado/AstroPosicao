\chapter{Primeiros passos}

Pretendemos escrever programas em Python para automatizar alguns dos cálculos feitos anteriormente. Neste capítulo, o objetivo é essencialmente desenvolver um programa que calcule a solução do Exemplo \ref{ex:9}. Isto é, partindo de ascensão reta e declinação, obter altura e azimute da estrela. Para isso, é necessário saber a latitude geográfica e tempo sideral. Por enquanto, vamos supor que tempo sideral seja um dado. Com esse programa, ficará simples de re-executar as operações com dados de entrada diferentes.

\section{Graus e radianos}

Vamos começar com alguns exemplos simples de uso de Python. Um pequeno programa para converter um ângulo de graus para radianos seria assim:

\begin{lstlisting}[language=Python]
import numpy as np

angulo_em_graus = 90
angulo_em_radianos = angulo_em_graus * np.pi / 180
print(angulo_em_radianos)
\end{lstlisting}

\noindent\texttt{1.5707963267948966}\\

\noindent A biblioteca numpy foi importada no início apenas para podermos usar o valor de $\pi$. É trivial escrever um programa que faça a operação inversa, convertendo radianos para graus.

Em Python, existem muitas funções já prontas para operações que são freqüentemente necessárias. Por exemplo, já existe a função \texttt{np.radians()}, que converte de graus para radianos e pode ser usada assim, por exemplo:

\begin{lstlisting}[language=Python]
import numpy as np

x = np.radians(90)
\end{lstlisting}

\noindent Analogamente, a função \texttt{np.degrees()} converte de radianos para graus:

\begin{lstlisting}[language=Python]
import numpy as np

y = np.degrees(1.57)
\end{lstlisting}


\section{Graus sexagesimais para graus decimais}

É instrutivo escrever dois pequenos programas que convertam graus sexagesimais (isto é, na notação $45^{\circ}\,30'\,36''$) para graus decimais (na notação $45.51^{\circ}$) --- e vice versa. A sugestão é que o leitor escreva esses dois programas por conta própria, antes de consultar a resolução apresentada a seguir.

Partindo desse exemplo, colocamos cada componente em uma variável:

\begin{lstlisting}[language=Python]
graus = 45
minutos = 30
segundos = 36
\end{lstlisting}

\noindent Primeiro, passamos os segundos para minutos:

\begin{lstlisting}[language=Python]
segundos/60
\end{lstlisting}
\noindent\texttt{0.6}\\

\noindent Agora, passando esses minutos para graus:

\begin{lstlisting}[language=Python]
(minutos + segundos/60)/60
\end{lstlisting}
\noindent\texttt{0.51}\\

\noindent Somando esses graus aos graus inteiros que já estavam lá:

\begin{lstlisting}[language=Python]
graus + (minutos + segundos/60)/60
\end{lstlisting}
\noindent\texttt{45.51}\\

\noindent Passando a limpo, poderíamos escrever uma única linha:

\begin{lstlisting}[language=Python]
graus_decimais = graus + minutos/60 + segundos/3600
\end{lstlisting}

Quando um mesmo procedimento vai ser usado diversas vezes dentro de um programa maior, é conveniente definir uma função, ou seja, algo que recebe um ou mais argumentos e retorna algum resultado. Assim fica mais simples de chamar a função toda vez que for necessária, deixando o código mais legível. Nossa função ficaria assim:

\begin{lstlisting}[language=Python]
def sexagesimal_to_decimal(graus, minutos, segundos):
    graus_decimais = graus + minutos/60 + segundos/3600
    return graus_decimais
\end{lstlisting}

\noindent Ela recebe 3 argumentos, efetua os cálculos e retorna 1 valor. É uma boa prática colocar comentários explicando qual o objetivo da função, o que ela espera receber e o que ela devolve. Por exemplo, algo mais ou menos desse estilo:

\begin{lstlisting}[language=Python]
def sexagesimal_to_decimal(graus, minutos, segundos):
    '''
    Funcao que converte graus sexagesimais para graus decimais
    Input: graus, minutos e segundos
    Returns: graus sexagesimais
    '''
    graus_decimais = graus + minutos/60 + segundos/3600
    return graus_decimais
\end{lstlisting}

\noindent Nesse exemplo específico, pode parecer redundante colocar comentários, pois a função é muito simples. Mas documentação é sempre valiosa.

\section{Graus decimais para graus sexagesimais}

O próximo exemplo é a operação inversa. Partimos de um ângulo dado em graus decimais. A variável de entrada é apenas uma:

\begin{lstlisting}[language=Python]
theta = 45.51
\end{lstlisting}

\noindent O primeiro passo é obter os graus inteiros. Para separar os graus inteiros da parte decimal, vamos tomar o \textit{quociente} da divisão inteira por 1. Em python3, isso se faz com o símbolo \texttt{//}.

\begin{lstlisting}[language=Python]
graus_inteiros = theta // 1
print(graus_inteiros, 'deg')
\end{lstlisting}
\noindent\texttt{45.0 deg}\\

\noindent Precisamos extrair minutos inteiros daquilo que sobrou depois do ponto decimal. Para separar a parte decimal, vamos tomar o \textit{resto} da divisão inteira. Em python3, isso se faz com
o símbolo~\texttt{\%}.

\begin{lstlisting}[language=Python]
graus_decimais = theta % 1
print(graus_decimais, 'deg')
\end{lstlisting}
\noindent\texttt{0.509999999999998 deg}\\

\noindent Para converter esses graus decimais para minutos, basta multiplicar por 60:

\begin{lstlisting}[language=Python]
minutos = graus_decimais * 60
print(minutos, 'arcmin')
\end{lstlisting}
\noindent\texttt{30.59999999999988 arcmin}\\

\noindent Extraindo os minutos inteiros com o quociente da divisão inteira:

\begin{lstlisting}[language=Python]
minutos_inteiros = minutos // 1
print(minutos_inteiros, 'arcmin')
\end{lstlisting}
\noindent\texttt{30.0 arcmin}\\

\noindent Agora, precisamos extrair segundos daquilo que sobrou depois do ponto decimal. Extraindo os minutos decimais com o resto da divisão inteira:

\begin{lstlisting}[language=Python]
minutos_decimais = minutos % 1
print(minutos_decimais, 'arcmin')
\end{lstlisting}
\noindent\texttt{0.5999999999998806 arcmin}\\

\noindent Esses minutos decimais são convertidos para segundos multiplicando por 60:

\begin{lstlisting}[language=Python]
segundos = minutos_decimais * 60
print(segundos, 'arcsec')
\end{lstlisting}
\noindent\texttt{35.99999999999284 arcsec}\\

Passando a limpo, a função definitiva recebe 1 argumento (graus decimais) e retorna 3 valores (graus, minutos e segundos):

\begin{lstlisting}[language=Python]
def decimal_to_sexagesimal(theta):
    graus = theta // 1
    minutos = ((theta % 1) * 60) // 1
    segundos = (((theta % 1) * 60) % 1) * 60
    return graus, minutos, segundos
\end{lstlisting}

\noindent Se o ângulo de entrada puder ser negativo, essa função teria problemas. (Por exemplo, teste o comportamento de \texttt{-45.9 // 1}). Uma maneira simples de contornar esse problema seria fazer
todos os cálculos usando o módulo do ângulo. E depois colocar o sinal de volta na resposta:

\begin{lstlisting}[language=Python]
import numpy as np

def decimal_to_sexagesimal(theta):
    '''
    Funcao que converte graus decimais para graus sexagesimais
    Input: graus, minutos, segundos
    Returns: graus decimais
    '''
    S = np.sign(theta)
    theta = abs(theta)
    graus = theta // 1
    minutos = ((theta % 1) * 60) // 1
    segundos = (((theta % 1) * 60) % 1) * 60
    return S*graus, S*minutos, S*segundos
\end{lstlisting}

\noindent Testando o uso da função com um input qualquer:

\begin{lstlisting}[language=Python]
graus, minutos, segundos = decimal_to_sexagesimal(-10.0997)
print(graus, 'deg', minutos, 'arcmin', segundos, 'arcsec')
\end{lstlisting}
\noindent\texttt{-10.0 deg -5.0 arcmin -58.92000000000124 arcsec}\\

Haveria outras maneiras de lidar com inteiros e restos, por exemplo empregando a função \texttt{int()}.

\newpage
\section{Transformação de coordenadas}
\label{sec:trans}

O objetivo agora é escrever uma função que receba as coordenadas equatoriais ($\alpha, \delta$) e retorne como resposta as coordenadas horizontais ($h, A$). É preciso conhecer a latitude e o tempo sideral. Os dados de entrada seriam da seguinte forma (usando os mesmos valores do exemplo \ref{ex:9}):

\begin{lstlisting}[language=Python]
ra  =   4.0 # h 
dec =  20.0 # deg
TS  =   7.0 # h
phi = -30.0 # deg
\end{lstlisting}

\noindent Trata-se essencialmente de implementar as equações \ref{conv1} a \ref{conv3}. Essa tarefa é intencionalmente deixada como exercício. Usando como ponto de partida o esqueleto da função abaixo, complete o resto do código e teste para o exemplo já solucionado anteriormente.

\begin{lstlisting}[language=Python]
def radec_to_Ah(ra, dec, TS, phi):
    '''
    Funcao que converte coordenadas equatoriais para horizontais
    input
    ra : ascensao reta (horas decimais) [0,24]
    dec : declinacao (graus decimais) [-90,90]
    TS : tempo sideral (horas decimais) [0,24]
    phi : latitude geografica (graus decimais) [-90,90]
    output
    A : azimute (graus decimais) [0,360]
    h : altura (graus decimais) [-90,90]
    '''
    
    
    
    
    
    
    
    
    
    
    # complete o resto do codigo aqui...

    
    
    
    
    
    
    
    
    
    return A, h
\end{lstlisting}


