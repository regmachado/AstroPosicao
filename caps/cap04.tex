\chapter{Sistemas de coordenadas}

Neste capítulo, vamos introduzir os sistemas de coordenadas necessários para este curso. Discutiremos o conceito de tempo sideral detalhadamente.

\section{Coordenadas horizontais}

\begin{wrapfigure}[20]{r}{0pt}
\includestandalone{figs/planovertical}
\caption{Plano vertical de um astro.}
\label{fig:planovertical}
\end{wrapfigure}

O sistema de coordenadas horizontais tem esse nome pois o plano de referência é o horizonte do observador. São também chamadas de coordenadas altazimutais, pois os nomes das coordenadas são altura e azimute, definidos a seguir.

O plano vertical de um astro (Fig.~\ref{fig:planovertical}) é um plano perpendicular ao horizonte e que passa pelo zênite e pelo astro. Isso nos permite definir a primeira coordenada, que é a altura (Fig.~\ref{fig:coord-horizontais}).
 
\textbf{Altura} ($h$) é o ângulo desde o horizonte até o astro, medido ao longo do seu plano vertical. Alturas estão definidas no intervalo:
%
\begin{equation}
-90^{\circ} \leqslant h \leqslant +90^{\circ}.
\end{equation}
%
A altura do horizonte é $h=0^{\circ}$ e a altura do zênite é $h=90^{\circ}$. Alturas negativas significam que o astro está abaixo do horizonte. Alternativamente, podemos utilizar o complemento da altura, que é a distância zenital.

\newpage

\begin{wrapfigure}[22]{R}{0pt}
\includestandalone{figs/coord-horizontais}
\caption{Coordenadas horizontais: altura ($h$) e azimute ($A$).}
\label{fig:coord-horizontais}
\end{wrapfigure}

\textbf{Distância zenital} ($z$) é o ângulo desde o zênite até o astro, medido ao longo do seu plano vertical. A relação entre altura e zênite é simplesmente:
%
\begin{equation}
h + z  =  90^{\circ}.
\end{equation}
%
Portanto, a distância zenital varia no intervalo:
%
\begin{equation}
0^{\circ} \leqslant z \leqslant 180^{\circ},
\end{equation}
%
sendo que astros abaixo do horizonte têm distância zenital maior que $90^{\circ}$.

A segunda coordenada do sistema horizontal é chamada de azimute. \textbf{Azimute} ($A$) é o ângulo entre o ponto cardeal Norte e o vertical do astro, medido sobre o plano do horizonte, de Norte para Leste. O azimute está definido no intervalo:
%
\begin{equation}
0^{\circ} \leqslant A \leqslant 360^{\circ}.
\end{equation}
%
Nessa definição, os pontos cardeais têm os seguintes azimutes: $A=0^{\circ}$ (Norte), $A=90^{\circ}$ (Leste), $A=180^{\circ}$ (Sul) e $A=270^{\circ}$ (Oeste). Essa convenção não é única, sendo possível encontrar referências que adotam outras convenções do sentido de contagem do azimute, então convém reparar com atenção em cada caso.

Resumindo, há duas possibilidades para expressar as coordenadas do sistema horizontal:

\begin{equation*}
\left\{
\begin{aligned}
& \text{altura}~h\\
& \text{azimute}~A
\end{aligned}
\right.
\qquad {\rm ou} \qquad
\left\{
\begin{aligned}
& \text{distância zenital}~z\\
& \text{azimute}~A
\end{aligned}
\right.
\end{equation*}

As coordenadas equatoriais são \textit{locais}, isto é, dependem da latitude do observador. Mais do que isso, $A$ e $z$ de uma dada estrela variam o tempo todo, conforme a esfera celeste gira ao redor do observador.

\section{Coordenadas equatoriais}

No sistema de coordenadas equatoriais, as coordenadas de uma dada estrela são essencialmente `fixas', se ignorarmos efeitos de longo prazo como a precessão e outras pequenas perturbações. O sistema equatorial tem esse nome pois o plano de referência é o equador celeste. Os dois ângulos do sistema equatorial são chamados de ascensão reta e declinação, definidos a seguir (Fig.~\ref{fig:coord-equatoriais}).

\newpage

\begin{wrapfigure}[20]{r}{0pt}
\includestandalone{figs/coord-equatoriais}
\caption{Coordenadas equatoriais: ascensão reta ($\alpha$) e declinação ($\delta$).}
\label{fig:coord-equatoriais}
\end{wrapfigure}

\textbf{Declinação} ($\delta$) é o ângulo desde o equador até a estrela, medido sobre o meridiano que passa pela estrela. A declinação está definida no intervalo:
%
\begin{equation}
-90^{\circ} \leqslant \delta \leqslant +90^{\circ},
\end{equation}
%
sendo positiva no hemisfércio Norte e negativa no hemisfério Sul. A declinação é uma coordenada análoga à latitude geográfica para pontos na superfície terrestre.

Da mesma forma, precisaremos de uma coordenada análoga ao que seria a longitude geográfica. Para definir longitude geográfica, foi necessário escolher um meridiano de referência que sirva como o zero da contagem. Adotou-se como meridiano principal aquele que passa pelo observatório de Greenwich. Analogamente, nas coordenadas equatoriais adota-se como referência o meridiano que passa pelo Ponto Vernal. Assim podemos definir a ascensão reta.

\textbf{Ascensão reta} ($\alpha$) é o ângulo medido sobre o equador celeste, desde o ponto \Aries\ até o meridiano que passa pela estrela. A ascensão reta cresce de Oeste para Leste, isto é, no sentido oposto ao do movimento da esfera celeste. Ou ainda: no sentido anti-horário, se a esfera celeste for vista a partir do pólo Norte. É o sentido da seta amarela na Fig.~\ref{fig:coord-equatoriais}. A ascensão reta é um ângulo que pode variar no intervalo
%
\begin{equation}
0^{\circ} \leqslant \alpha \leqslant 360^{\circ},
\end{equation}
%
É mais comum que a ascensão reta seja expressa não em graus, mas sim em \textit{horas}.\footnote{Entendendo que trata-se da unidade angular hora. Já que 24$^{\rm h}$ = 360$^{\circ}$, cada hora corresponde a 15$^{\circ}$. A hora angular também tem as subdivisões minutos e segundos. A notação usual é no formato: $12^{\rm h}\,34^{\rm m}\,56^{\rm s}$.} Nesse caso, o intervalo de variação da ascensão reta é:
%
\begin{equation}
0^{\rm h} \leqslant \alpha \leqslant 24^{\rm h},
\end{equation}
%
Resumindo, as coordenadas equatoriais são:
%
\begin{equation*}
\left\{
\begin{aligned}
& \text{ascensão reta}~\alpha\\
& \text{declinação}~\delta
\end{aligned}
\right.
\end{equation*}

\newpage
\section{Coordenadas horárias}

No sistema equatorial, ambas as coordenadas ($\alpha, \delta$) são fixas. Vamos introduzir agora um outro sistema onde uma coordenada é fixa e a outra varia com o tempo. Pode parecer uma confusão desnecessária, mas a conveniência prática desse sistema ficará clara mais adiante. Trata-se do sistema equatorial horário, ou apenas sistema horário. No sistema horário, a declinação continua sendo a mesma. Já a ascensão reta será substituída por uma nova coordenada, chamada de ângulo horário.

\begin{wrapfigure}{R}{0pt}
\includestandalone{figs/coord-horarias}
\caption{Coordenadas horarias: ângulo horário ($H$) e declinação ($\delta$).}
\label{fig:coord-horarias}
\end{wrapfigure}

\textbf{Ângulo horário} ($H$) é o ângulo medido sobre o equador, desde o meridiano local até o meridiano que passa pela estrela. O ângulo horário poderia ser expresso em graus (de 0$^{\circ}$ a 360$^{\circ}$), mas como o nome já sugere, ele é mais comumente expresso em horas no intervalo:
%
\begin{equation}
0^{\rm h} \leqslant H \leqslant 24^{\rm h},
\end{equation}
%
O ângulo horário cresce no sentido de Leste para Oeste, ou seja, no mesmo sentido da rotação da esfera celeste. Em outras palavras, o valor de $H$ aumenta conforme o tempo passa. No momento em que uma estrela tem $H = 0^{\rm h}$, ela está cruzando o meridiano local --- é a chamada \textbf{passagem meridiana}. É também quando ela atinge a sua altura máxima --- a chamada culminação. Um certo tempo mais tarde, quando ela já estiver a $15^{\circ}$ a Oeste do meridiano local, seu ângulo horário será $H = 1^{\rm h}$, e assim por diante. Por isso, o ângulo horário funciona como uma espécie de medida da passagem do tempo.

Resumindo, as coordenadas horárias são:
%
\begin{equation*}
\left\{
\begin{aligned}
& \text{ângulo horário}~H\\
& \text{declinação}~\delta
\end{aligned}
\right.
\end{equation*}

A ascensão reta ($\alpha$) e o ângulo horário ($H$) estão conectados entre si através de uma grandeza chamada tempo sideral, que será definida a seguir.

\newpage
\section{Dia sideral}

Nosso objetivo aqui é entender como calcular o tempo sideral em um momento qualquer do ano. Mas, antes disso, vamos apresentar brevemente o conceito de dia sideral. Do ponto de vista físico, podemos pensar na Terra como uma esfera em rotação. Ela tem uma dada velocidade angular e portanto um período, que nada mais é do que o tempo que a Terra leva para dar uma volta completa ao redor de seu eixo. Esse intervalo de tempo é o \textbf{dia sideral}, que dura aproximadamente 23 horas e 56 minutos. A esfera celeste aparenta girar ao nosso redor com esse período. Cada estrela (e em particular o ponto \Aries) leva 23\,h\,56\,min para retornar à mesma posição na esfera celeste. É isso que o céu faz, continuamente.

O que complica as coisas é o Sol e a nossa contagem de tempo atrelada a ele. Ocorre que, ao mesmo tempo em que a Terra dá uma volta ao redor de seu próprio eixo, ela também avança um pouco ao longo da sua órbita ao redor do Sol. O sentido da rotação é o mesmo da translação. Compare o comportamento de uma estrela com o comportamento do Sol. Uma estrela cruza o meridiano local; 23\,h\,56\,min depois, ela cruza o meridiano local novamente. Já o Sol demora um pouco mais: o Sol cruza o meridiano local; 23\,h\,56\,min depois, ele ainda não retornou ao meridiano local. Como a Terra avançou, o Sol ficou um pouco `para trás', e são necessários 4 minutos adicionais para que ele volte a se alinhar com o meridiano local. Esse é o \textbf{dia solar}: é o intervalo de tempo entre duas passagens do Sol pelo meridiano local. O dial solar dura 24 horas. De fato, é esse o intervalo de tempo que decidimos dividir em 24 partes iguais chamadas de `horas'. São essas as horas comuns do tempo civil, isto é, o horário usado nos relógios da vida cotidiana. Por outro lado, se o dia sideral fosse dividido em 24 partes iguais, teríamos as horas siderais.

Considere agora como o céu noturno se comporta com relação ao horário de relógios comuns. Digamos que uma dada estrela cruze o meridiano local hoje à noite às 21\,h\,30\,min, por exemplo. Amanhã à noite, ela cruzará o meridiano local 4 minutos mais cedo, às 21\,h\,26\,min. Depois de amanhã, às 21\,h\,22\,min, e assim por diante.

Pensando de outra forma, imagine observar o céu todas as noites depois do pôr do Sol, num dado horário fixo, por exemplo às 20\,h. Digamos que hoje à noite seja possível ver, acima do horizonte Oeste, a constelação de Áries às 20\,h. Conforme a noite passa, o céu gira e essa constelação se põe, naturalmente. Na noite seguinte, no horário fixo 20\,h, o céu inteiro já estará um pouquinho deslocado para o Oeste. Afinal, as estrelas tinham retornado às mesmas posições já às 19\,h\,56\,min e nesses 4 minutos, avançaram mais um pouco rumo ao Oeste. Como o céu gira 360$^{\circ}$ em 23\,h\,56\,min, em 4 minutos o avanço angular é de aproximadamente 1$^{\circ}$. Ou seja, a cada noite no mesmo horário, o céu já está ${\sim}1^{\circ}$ mais para o Oeste que na noite anterior. Depois de 1 mês (${\sim}30^{\circ}$) a constelação de Áries já estará inteiramente abaixo do horizonte, e será a constelação de Touro que veremos acima do horizonte Oeste depois do pôr do Sol. No mês seguinte, será a vez de Gêmeos, e assim por diante, retornando a Áries depois de 12 meses. É por isso que, mesmo estando em uma latidide fixa, as constelações visíveis no céu noturno mudam ao longo dos meses. O céu funciona como um calendário.
% Os povos da antigüidade tinham familiaridade com diversos fenômenos cíclicos da natureza, como migrações de animais, mudança da vegetação, ciclos de chuvas, cheias dos rios etc. Os ciclos biológicos e climáticos podem ser periódicos em média, mas com alguma variação. Nenhum deles era tão perfeitamente confiável quanto a repetição anual dos movimentos dos astros.

\section{Tempo sideral}
\label{sec:ts}

Voltando às coordenadas horárias, precisamos saber converter de ascensão reta ($\alpha$) para ângulo horário ($H$) e vice-versa. Esses dois ângulos estão conectados entre si através do Tempo Sideral.

\textbf{Tempo sideral} ($TS$) é o ângulo horário do ponto \Aries. Na Fig.~\ref{fig:coord-horarias}, é possível constatar que o ângulo horário do ponto \Aries\ é a soma:
%
\begin{equation}
TS = H + \alpha.
\end{equation}
%
Ou, dito de outra forma, para transformar entre as coordenadas $\alpha$ e $H$, é necessário conhecer o $TS$ da observação.

O ponto \Aries\ naturalmente gira com a esfera celeste como um todo, então ao longo das horas, seu ângulo horário estará variando. Portanto, o $TS$ depende do horário. Mas, por causa dos movimentos descritos na seção anterior, também há uma variação ao longo do ano. A seguir, vamos apresentar uma maneira de \textit{estimar aproximadamente} o $TS$ de qualquer data. Não se trata de um cálculo exato, mas serve para esclarecer alguns conceitos.

Vamos antes relembrar a noção de fuso horário, que vale para o tempo civil. O tempo civil de Greenwich era historicamente chamado de GMT (Greenwich Mean Time), mas a nomenclatura atual é Tempo Universal (sigla UT). Na sigla UTC, freqüentemente usada, o C é de Coordenado; a correção de tempo atômico não vem ao caso. Para passar de tempo universal para o tempo de um dado fuso, basta somar ou subtrair 1 hora inteira a cada $15^{\circ}$ de longitude. Por exemplo, se são 12\,h em Greenwich, são 11\,h para alguém que esteja na longitude $-15^{\circ}$ (Oeste), e são 13\,h para alguém que esteja na longitude $+15^{\circ}$ (Leste). Cada fuso horário teria idealmente a mesma largura de $15^{\circ}$ e todos naquele intervalo adotariam a hora civil do centro do fuso. Na prática, os fusos são arbitrariamente irregulares, para seguir fronteiras de países etc. Nós estamos a Oeste de Greenwich. A hora legal de Brasília é UTC$-3$. Quando são 12\,h em Greenwich, são 9\,h em Brasília.

Quando se trata de tempo sideral, não podemos usar horas inteiras dos fusos. Precisamos da variação contínua das longitudes. A relação entre o TS de um local e o TS de Greenwich é:
%
\begin{equation}\label{TSlocal}
TS_{\rm local} = TS_{\rm Greenwich} + \text{longitude},
\end{equation}
%
sendo a longitude negativa para o Oeste. Por exemplo, a longitude de Curitiba é:
%
\[
-49^{\circ} \, 16' \, 15'' = -49.27^{\circ}
\]
ou expressa em horas (dividir por 15):
\[
-3\,{\rm h}\,17{\rm min}\,5{\rm s} = -3.285\,{\rm h}
\]
%
Então a informação da longitude exata do local (não meramente o fuso) é necessária para o cálculo do $TS$ local. Agora falta entender de onde vem o cálculo do $TS$ de Greenwich.

Lembramos que tempo sideral ($TS$) é o ângulo horário ($H$) do ponto \Aries. E ângulo horário é o ângulo entre o astro e o meridiano local. Precisamos enxergar onde está o ponto \Aries\ em cada momento. O Sol está no ponto \Aries\ no início da primavera do hemisfério Norte (21 de março). Então, ao meio-dia verdadeiro de 21 de março, um observador em Greenwich verá o Sol cruzar o meridiano local. Nesse momento, o ponto \Aries\ terá ângulo horário $H=0^{\rm h}$.

\begin{exemplo}{0}
\textit{Qual o tempo sideral em Greenwich ao meio-dia de 21 de março?}\\

Resposta: $TS = 0$\,h. No equinócio de primavera boreal, o Sol está no ponto \Aries. Ao meio-dia, o ângulo horário do ponto \Aries\ é zero, portanto o tempo sideral é zero.
\end{exemplo}

\begin{exemplo}{1}
\textit{Qual o tempo sideral em Greenwich às 18\,h\,30\,min de 21 de março?}\\

Resposta: $TS = 6$\,h\,30\,min. São 6.5\,h depois do tempo sideral zero.
\end{exemplo}

\begin{exemplo}{2}
\textit{Qual o tempo sideral em Greenwich às 8\,h de 21 de março?}\\

Resposta: $TS = 20$\,h. São 4\,h antes do tempo sideral zero (TS varia entre 0\,h e 24\,h).
\end{exemplo}

Essas estimativas aproximadas de $TS$ no dia do equinócio são toleráveis dentro de um certo erro. Elas não são exatas, entre outros motivos, pois estamos misturando tempo solar com tempo civil, sem levar em conta a Equação do Tempo. Mas o propósito, no momento, é elucidar o funcionamento do $TS$, mesmo que o valor numérico da estimativa não seja tão acurado.

Vejamos agora como o ponto \Aries\ se comporta ao longo dos dias do ano. O ponto \Aries\ (do meio-dia) percorrerá um ângulo de 360$^{\circ}$ (ou 24$^{\rm h}$) ao longo do ano (cerca de 365 dias). Na prática, o $TS$ do meio-dia avança aproximadamente 2\,h por mês. \\

\begin{exemplo}{3}
\textit{Qual o tempo sideral em Greenwich ao meio-dia de 20 de abril?}\\

Resposta: $TS = 2$\,h. Um mês depois do equinócio, o $TS$ do meio-dia terá avançado 2 horas.
\end{exemplo}

\begin{exemplo}{5}
\textit{Qual o tempo sideral em Greenwich às 13\,h de 20 de abril?}\\

Resposta: $TS = 3$\,h. Uma hora depois do exemplo anterior.
\end{exemplo}

No caso geral, é necessário contar o número de dias desde o equinócio de março (de meio-dia a meio-dia). Em seguida, somar ou subtrair as horas para além ou para antes do meio-dia.\\



\begin{exemplo}{6}
\textit{Qual o tempo sideral em Greenwich ao meio-dia de 2 de junho?}\\

Resposta: $TS = 4$\,h\,48\,min. De 21 de março até 2 de junho, passaram-se 73 dias. Com um avanço na taxa de 24\,h/365\,dias = 0.0657 h/dia, tivemos cerca de 4.8\,h = 4\,h\,48\,min.
\end{exemplo}

\begin{exemplo}{7}
\textit{Qual o tempo sideral em Greenwich às 10\,h\,30\,min de 2 de junho?}\\

Resposta: $TS = 3$\,h\,18\,min. Uma hora e meia antes do exemplo anterior.
\end{exemplo}

Todos esses exemplos diziam respeito ao $TS$ de Greenwich. Para passar para o $TS$ de um local qualquer, basta somar a longitude (equação~\ref{TSlocal}).\\

\begin{exemplo}{10ago}
\textit{Qual o tempo sideral em Curitiba às 20\,h\,30\,min de 10 de agosto?}\\

Resposta:\\

Primeiro calcularemos o tempo sideral em Greenwich. Às 20\,h\,30\,min no fuso de Brasília são 23\,h\,30\,min em UT.\\

Em 10 de agosto, já se passaram 142 dias desde 21 de março. Portanto
\[
142\,\text{dias} \times \frac{24\,{\rm h}}{365\,{\rm dias}} = 9.34\,{\rm h} \simeq 9\,{\rm h}\,20\,{\rm min}  
\]
No meio-dia de 10 de agosto em Greenwich, $TS=9$\,h\,20\,min. Onze horas e meia mais tarde, $TS=20$\,h\,50\,min.\\

Finalmente, para passar para o $TS$ de Curitiba, somar a longitude:
\begin{eqnarray*}
TS_{\rm local} &=& TS_{\rm Greenwich} + \text{longitude}\\
TS_{\rm local} &=& 20\,{\rm h}\,50\,{\rm min} - 3\,{\rm h}\,17\,{\rm min} \\
TS_{\rm local} &=&  17\,{\rm h}\,33\,{\rm min}
\end{eqnarray*}

\end{exemplo}

Com isso, temos um procedimento aproximado que serve para compreender o conceito de tempo sideral e também serve para fazer uma estimativa, porém tendo em mente que os valores obtidos podem dar erros de até ${\sim}15$ minutos.

Uma maneira de conferir a resposta correta é usando o Stellarium. É sempre possível escolher a localização, bem como data e horário. Clicando em alguma estrela, aparecem todos os dados; um deles é o tempo sideral.
 
Outra maneira de conferir o cálculo do tempo sideral é através do link \url{https://aa.usno.navy.mil/data/siderealtime}.
