\chapter{Observação noturna}

Na primeira noite de observação, cabe aos alunos propor a observação a ser feita. Com os cálculos e programas desenvolvidos ao longo do curso, é possível prever as coordenadas de uma estrela na data e horário da observação. Um desafio aos alunos seria tentar confeccionar um instrumento rudimentar que permita medir ângulos (altura e azimute). Qual seria a precisão da leitura? Ao invés de medir azimute, seria possível também medir o instante da passagem meridiana de uma estrela.

Na segunda noite, finalmente faremos observações com um telescópio em que o apontamento é computadorizado.
