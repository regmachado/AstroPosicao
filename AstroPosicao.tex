\documentclass[11pt,a4paper]{report}

\usepackage{pres/preamble}

\begin{document}

% cover
{
\sffamily
\pagecolor{mygreen}
\thispagestyle{empty}
\begin{tikzpicture}[remember picture, overlay, shift=(current page.south west)]
\begin{scope}[x={(current page.south east)},y={(current page.north west)}]
\filldraw [fill=black, draw=none, opacity=0.6] (0.2,0.7) rectangle (1,1);
\filldraw [fill=darkgray, draw=none, opacity=0.6] (0,0.0) rectangle (0.2,1);
\node at (0.28,0.9) [anchor=west] {\fontsize{20}{20}\selectfont\textcolor{mybeige}{Curso de Extensão}};
\node at (0.28,0.8) [anchor=west] {\fontsize{40}{45}\selectfont\textcolor{white}{Astronomia de Posição}};
\node at (0.92,0.60) [anchor=east] {\fontsize{20}{20}\selectfont\textcolor{mybeige}{Rubens E.~G.~Machado}};
\node at (0.92,0.57) [anchor=east] {\fontsize{20}{20}\selectfont\textcolor{mybeige}{UTFPR}};
\node at (0.92,0.54) [anchor=east] {\fontsize{20}{20}\selectfont\textcolor{mybeige}{2023}};
\node at (1.01,-0.01) [anchor=south east, xshift=3cm, yshift=-3cm] {\includestandalone[scale=1.6]{figs/fig00}};
\end{scope}
\end{tikzpicture}
\clearpage
\pagecolor{white}
}


\chapter*{Prefácio}
\addcontentsline{toc}{chapter}{Prefácio}

Esta apostila foi preparada para um curso de extensão a ser realizado na UTFPR Campus Curitiba em março de 2023. Não se trata de um curso completo de Astronomia de Posição: aqui estão presentes apenas alguns tópicos selecionados. A finalidade da apostila é servir como um resumo ou guia de estudos para os estudantes que forem acompanhar o curso. O público-alvo principal são estudantes de Física, em qualquer semestre da graduação.

O planejamento do curso prevê 8 aulas teóricas e, paralelamente, 4 atividades práticas. As aulas teóricas estão divididas em 3 partes. Na primeira parte (Capítulos 1 a 4), serão introduzidos os conceitos relativos à esfera celeste e os sistemas de coordenadas. Na segunda parte (Capítulos 5 e 6), estudaremos transformações de sistemas de coordenadas usando trigonometria esférica ou matrizes de rotação. Na terceira parte (Capítulos 7 e 8), veremos como escrever um programa computacional que calcula as transformações, usando a linguagem de programação Python. As atividades observacionais (Capítulos 9 a 12) consistem em fazer medidas --- do Sol durante o dia, e de estrelas durante a noite.

A principal referência usada para preparar este curso foi o livro do professor Roberto Boczko, \textit{Conceitos de Astronomia}, que recentemente foi disponibilizado na página do IAG/USP.\footnote{\url{https://www.iag.usp.br/astronomia/sites/default/files/conceitos_astronomia.pdf}} Outra referência muito útil e amplamente utilizada nesse tema são as notas de aula de \textit{Astronomia de Posição} do professor Gastão B.~Lima~Neto, que são atualizadas periodicamente e cuja versão mais recente pode ser encontrada em sua página.\footnote{\url{http://www.astro.iag.usp.br/~gastao/astroposicao.html}}

Mesmo quem se dedica à astronomia teórica ou computacional não pode ignorar o vocabulário das observações. Por isso, a astronomia esférica, com sua terminologia arcana, ainda é indispensável para qualquer ramo do nosso ofício, desde o sistema solar até a cosmologia. E estudar os movimentos na esfera celeste continua sendo uma boa porta de entrada para a astronomia como um todo.

Esta apostila está disponível no seguinte link, onde também podem ser encontrados os códigos-fonte em \LaTeX\ usados para gerar o texto e as figuras vetoriais aqui presentes:

\begin{center}
\url{https://github.com/regmachado/AstroPosicao/}
\end{center}

\begin{flushright}
\noindent Rubens E. G. Machado\\
Universidade Tecnológica Federal do Paraná\\
\texttt{rubensmachado@utfpr.edu.br}
\end{flushright}


\setcounter{tocdepth}{1}
{
\hypersetup{hidelinks}
\tableofcontents
}

% format chapter titles
\titleformat{\chapter}[block]
{\flushright\bfseries\Huge}{\color{mygreen}\fontsize{80}{120}\selectfont\thechapter}{0pt~\linebreak[1cm]}{\color{mygreen}\Huge}

\part{Conceitos astronômicos}

\chapter{Introdução}
\label{cap01}

\section{Astronomia de Posição}

% que é Astronomia
A Astronomia de Posição, também chamada de Astronomia Fundamental ou ainda Astronomia Esférica, trata do estudo dos movimentos dos astros no céu, sem entrar no mérito da natureza física\footnote{Alguns sustentam que essa seria a suposta distinção entre os termos \textit{Astronomia} e \textit{Astrofísica}, sendo o segundo usado para indicar o estudo dos processos físicos. Em alguns contextos específicos, pode até ser útil usar o termo mais restritivo \textit{Astrofísica} com o propósito de separar a Astronomia de Posição dos demais ramos de estudo. Ou seja, \textit{Astronomia} inclui tudo, enquanto que \textit{Astrofísica} exclui Astronomia de Posição. No entanto, na maioria das situações, a distinção não é muito relevante.} desses corpos celestes. Em outras palavras, estamos interessados em meramente descrever as posições dos astros. Essas posições são \textit{angulares}: precisamos de 2 ângulos para especificar a localização de uma estrela na esfera celeste, da mesma forma que precisamos de 2 ângulos (latidude e longitude) para especificar a localização de uma cidade na superfície do globo terrestre.

% sistemas de coordenadas
Existem diferentes sistemas de coordenadas. Em um deles, damos os ângulos em relação ao `chão' de um dado observador --- seu horizonte. Nesse sistema, é claro que os valores dos ângulos mudam dependendo de onde o observador está na Terra. Além disso, o céu gira (de Leste para Oeste). Por esses motivos, as coordenadas angulares de uma estrela estão mudando o tempo todo nesse sistema. Existe um outro sistema que usa como referência algo mais absoluto: o equador, que é o mesmo para todos os observadores. Já nesse sistema, os valores dos ângulos não mudam ao longo da noite: são uma propriedade `fixa' de cada estrela. Os detalhes desses conceitos virão nos próximos capítulos.

\section{Objetivos do curso}

% objetivo: conversão
Neste curso aprenderemos a fazer a conversão entre os dois sistemas de coordenadas que foram esboçados acima: um relativo ao observador, e um `fixo'. Ocorre que além da variação temporal, esses sistemas de eixos são inclinados entre si; essas são as origens da dificuldade. Essa conversão de coordenadas pode parecer um árido detalhe de procedimentos matemáticos, mas é uma habilidade poderosa: isso nos permite prever observações astronômicas. Ou seja, saber previamente quais estrelas serão observáveis e quando. Já fica destacada uma propriedade importante do céu: cada estrela não é visível o tempo todo. Em certas épocas do ano, em certos horários da noite, algumas estrelas estarão acima do horizonte e outras estarão abaixo. E também depende de onde a pessoa está na superfície da Terra.

% objetivo: perguntas
O objetivo geral do curso pode ser exergado do seguinte modo. Pretendemos ser capazes de responder uma pergunta como esta:
%
\begin{quote}
\itshape Se estivermos em Curitiba no dia 11 de abril de 2023, às 20h30min, a estrela Sírius estará visível no céu?
\end{quote}
%
Como é possível saber algo assim? Mesmo quem ainda não está familiarizado com estudos de astronomia estará correto ao supor que existem programas de computador que seguramente dão a resposta pronta. Sim, mas a questão é entender como, sem tratar o computador como uma caixa-preta. O que exatamente está sendo calculado? Tem que ser possível compreender o procedimento. Afinal, os astrônomos da antigüidade também eram capazes de obter esses resultados. A formulação da pergunta sugere que são informações necessárias: a localização geográfica do observador; a data; o horário. Como exatamente essas grandezas entram na conta? Na realidade, podemos deixar a pergunta ainda mais quantitativa. Supondo que a resposta anterior seja afirmativa, uma pergunta mais interessante seria:
%
\begin{quote}
\itshape Se estivermos em Curitiba no dia 11 de abril de 2023, às 20h30min, para onde exatamente deveríamos apontar um telescópio para observar Sírius?
\end{quote}
%
Nessa cidade, nessa data, nesse horário, gire o telescópio tantos graus a partir do Norte e eleve tantos graus a partir do chão: lá estará Sírius. Tudo isso pode ser calculado. Vamos aprender como exatamente esses dois ângulos são obtidos. Sem supor nada previamente, vamos deduzir todas as contas, passo a passo, a partir de princípios primeiros.

% atividades práticas
Depois de ter calculado diretamente as coordenadas de uma estrela, vamos fazer uma observação noturna ao fim do curso, para medir os dois ângulos na prática. Mas para isso precisaremos conhecer a direção do ponto cardeal Norte. Nós já teremos previamente determinado a direção da linha Norte-Sul com boa precisão. Usando uma bússola? Pelo contrário: usando astronomia, através de observações da sombra do Sol ao redor do meio-dia (no momento de sombra mínima). Além disso, para fazer as contas, precisaremos conhecer a latidude geográfica de onde estamos. Teremos determinado diretamente nossa latitude com um método engenhoso: usando a sombra do Sol em um momento especial do ano: o dia do equinócio de outono do hemisfério Sul (21 de março), quando a sombra estiver alinhada com a linha Norte-Sul determinada anteriormente.

% python
Este curso também tem uma componente computacional. As equações que transformam coordenadas poderiam ser usadas normalmente, multiplicando diversos senos e cossenos à mão com lápis e papel. Nós vamos implementar essas equações em um programa escrito em Python. Quem estiver começando a aprender programação também deve se beneficiar da experiência, que não vai exigir conhecimentos avançados.

% suma
Em suma, o objetivo é que os participantes do curso aprendam os conceitos de esfera celeste necessários para entender os sistemas de coordenadas e, em seguida, aprendam as transformações que permitem prever observações. Teremos a oportunidade de construir tudo com as próprias mãos.

Resumindo, as atividades serão:
\begin{itemize}
\setlength\itemsep{0.2em}
\item Estudar os elementos da esfera celeste
\item Aprender as definições dos sistemas de coordenadas
\item Determinar a linha Norte-Sul na prática
\item Determinar a latitude geográfica na prática
\item Fazer as transformações dos sistemas de coordenadas (com lápis e papel)
\item Implementar essas equações em Python
\item Prever as coordenadas de uma estrela
\item Medir suas coordenadas em uma observação na prática
\end{itemize}


\section{Cronograma de atividades}

\begin{table}[h]
\begin{center}
\begin{tabular}{lccc}
\rowcolor{mybeige!80!gray}  
dia & data & Aula Teórica & Atividade Observacional \\
&&&\\[-0.2em]
\rowcolor{mybeige}
terça              & 07/03/2023 & Aula 1 & \\
\rowcolor{mybeige}
quarta             & 08/03/2023 & Aula 2 & \\
&&&\\[-0.2em]
\rowcolor{mybeige} 
terça              & 14/03/2023 & Aula 3 & Observação 1 (meio-dia)\\
\rowcolor{mybeige} 
quarta             & 15/03/2023 & Aula 4 & \\
&&&\\[-0.2em]
\rowcolor{mybeige}
terça              & 21/03/2023 & Aula 5 & Observação 2 (meio-dia)\\
\rowcolor{mybeige}
quarta             & 22/03/2023 & Aula 6 & \\
&&&\\[-0.2em]
\rowcolor{mybeige} 
terça              & 28/03/2023 & Aula 7 & \\
\rowcolor{mybeige} 
quarta             & 29/03/2023 & Aula 8 & \\
&&&\\[-0.2em]
\rowcolor{mybeige} 
terça              & 04/04/2023 & & Observação 3 (noite)\\
\rowcolor{mybeige} 
quarta             & 05/04/2023 & & Observação 4 (noite) \\        
\end{tabular}
\end{center}
\end{table}

\newpage
\section{Stellarium}

Existe um programa de mapa do céu muito usado, chamado \textit{Stellarium}\footnote{\url{https://stellarium.org}}. É um sotfware versátil, que traz uma imensa quantidade de informações sobre objetos astronômicos. De certa forma, ele responde imediatamente nossa pergunta sobre a observabilidade de determinada estrela em determinada data. Esse é um dos usos bastante difundidos do Stellarium: consultar visualmente a aparência do céu em qualquer data, tanto qualitativamente quanto quantitativamente, pois além de mostrar as estrelas e constelações, ele exibe as diversas coordenadas, e também informações como magnitudes etc. De fato, uma maneira rápida e eficaz de conferir contas de conversão de coordenadas é abrir o Stellarium e passar o mouse na coordenada desejada.

Nós vamos empregar o Stellarium para ilustrar alguns conceitos da esfera celeste nos próximos capítulos. Por enquanto, é suficiente instalar o software e se familiarizar com a sua operação, que é bastante intuitiva. Para quem está estudando esfera celeste e coordenadas, uma das funcionalidades mais úteis do Stellarium é a passagem do tempo. Com o atalho \texttt{F5}, é possível escolher data e horário. Estando numa determinada noite, é possível avançar as horas continuamente com a roda do mouse e ver o céu girando de Leste para Oeste. Similarmente, é possível fixar um horário e avançar os dias ou meses. É muito instrutivo inspecionar esses padrões de movimentos (das estrelas e do Sol), ainda que qualitativamente num primeiro momento. Nos capítulos seguintes, vamos estudar os detalhes quantitativos e fazer bom proveito das visualizações do Stellarium.

\chapter{Esfera celeste}

Neste capítulo e no próximo estudaremos alguns conceitos da esfera celeste. Esfera celeste é o céu. Quando olhamos para cima numa noite estrelada em um campo aberto, temos a impressão de estarmos sob uma abóbada, como se fosse um teto hemisférico. A outra metade da esfera não é visível pois está abaixo do solo.

\section{Horizonte e zênite}

\begin{wrapfigure}{r}{0pt}
\includestandalone{figs/horizonte}
\caption{Esfera celeste mostrando o plano horizontal de um observador e seu zênite.}
\label{fig:horizonte}
\end{wrapfigure}

Começamos com a definição de alguns elementos locais, isto é, que são propriedades de cada observador.

O plano horizontal é essencialmente o `chão' do observador (o plano verde da Fig.~\ref{fig:horizonte}). Onde o céu aparenta tocar o solo, teríamos a linha do horizonte. Na prática, acidentes do relevo, vegetação ou construções geralmente bloqueiam a vista desse círculo que seria o horizonte. Sobre a linha do horizonte estão os quatro pontos cardeais: Norte, Sul, Leste e Oeste.

Se traçarmos uma linha vertical partindo do observador, ela interceptará o céu no ponto chamado de zênite. Em outras palavras, o zênite é simplesmente o ponto acima da cabeça do observador.

\section{Meridiano local}

\begin{wrapfigure}{r}{0pt}
\includestandalone{figs/planomeridiano}
\caption{O plano meridiano de um observador contém a linha Norte-Sul e o zênite.}
\label{fig:planomeridiano}
\end{wrapfigure}

Uma maneira de definir o plano meridiano é dizer que ele contém estes três pontos: o Norte, o Sul e o zênite. Naturalmente contém o observador também. A intersecção desse plano com a esfera celeste é um grande círculo chamado de meridiano local. Veremos que o meridiano local tem muita utilidade na descrição dos movimentos dos astros.

A primeira delas diz respeito ao movimento diário do Sol e sua relação com os pontos cardeais. Veremos mais adiante que o Sol só nasce no ponto cardeal Leste em certos dias especiais do ano. Idem para o pôr do Sol, que só se dá no ponto cardeal Oeste em dias especiais. No entanto, continua sendo verdade que o Sol nasce no \textit{lado Leste} e se põe no \textit{lado Oeste} do horizonte. Então, se precisássemos determinar os pontos cardeais em um dia qualquer, não seria útil tentar usar a direção exata do nascer e do pôr do Sol.

Mas existe uma propriedade que é útil o ano todo: o instante de menor sombra do dia. Esse momento é o meio-dia verdadeiro (ou meio-dia solar). Consideremos a sombra de uma haste vertical (um gnômon): de manhã, com o Sol vindo do Leste, a sombra apontará para o lado Oeste; já à tarde, com o Sol descendo para o Oeste, a sombra apontará para o Leste. Portanto, em um instante intermediário --- o meio-dia verdadeiro --- a sombra estará ao longo da direção Norte-Sul, também chamada de linha meridiana. Esta é a mínima sombra do dia. É verdade que, ao longo do ano, o comprimento da sombra do meio-dia muda, sendo mais curta no verão e mais longa no inverno. Mas independentemente do comprimento, a direção da mínima sombra do dia sempre define a direção da linha Norte-Sul. No Capítulo~\ref{cap:meridiano}, apresenta-se uma descrição mais detalhada de um método prático para essa determinação.

\section{Norte magnético}

É pertinente ressaltar que, na astronomia, o Norte que usamos é sempre o geográfico, não o magnético. O norte magnético, lido na prática em uma bússola, pode diferir consideravelmente do norte geográfico. A diferença entre os dois se chama declinação magnética. A correção não é trivial, pois a declinação magnética depende da localização do observador na superfície da Terra. Mas não se trata de um cálculo teórico simples que possa ser feito usando meramente a latitude. Isso ocorre porque o campo magnético do planeta é instrinsecamente assimétrico. Portanto, para fazer a correção é preciso consultar um mapa específico. Além disso, o mapa precisa ser recente, pois o campo magnético muda com o tempo, mesmo na escala de anos. Tipicamente, a declinação magnética pode variar da ordem de alguns graus por década, dependedo da região.

Atualmente, para um observador em Curitiba, a declinação magnética vem a ser de aproximadamente $-20^{\circ}$. O sinal negativo, neste caso, significa que a bússola aponta mais para o Oeste. Então, estando em Curitiba, alguém que faça uma leitura de uma bússola na prática precisa lembrar que o Norte verdadeiro está cerca de 20$^{\circ}$ \textit{para o Leste} do que indica a agulha. De acordo com modelos atuais, este valor tenderá a aumentar cerca de 1$^{\circ}$ por década.\footnote{\url{https://www.ngdc.noaa.gov/geomag/calculators/magcalc.shtml}}


\section{O que é a esfera celeste}

Considere uma paisagem cotidiana composta de elementos como pessoas, árvores, prédios e montanhas. Pode ser uma fotografia ou uma paisagem vista pela janela. A montanha é muito grande, mas talvez ocupe no seu campo visual o mesmo tamanho aparente que um prédio que, apesar de ser bem menor, está bem mais perto. Ou seja, nós geralmente conseguimos ter uma interpretação bastante adequada das diferentes distâncias. Isso funciona graças a alguns fatores: temos familiaridade com os tamanhos reais desses corpos quando vistos de perto; temos uma noção de perspectiva de como os corpos se apresentam quando vistos sob diferentes ângulos e sob diferentes condições de iluminação; como esses corpos têm extensão, o tamanho angular de cada um serve de referência para julgar as distâncias dos demais que compõem a paisagem. Em suma, nossa intuição visual é treinada pela experiência prévia que temos com o mundo a nossa volta, nas escalas de tamanho humanas. Nada disso funciona no caso do céu noturno.

Quando olhamos para o céu noturno, não temos como interpretar visualmente que as estrelas estão em difetentes distâncias. Cada estrela é praticamente uma fonte puntiforme de luz, sem tamanho discernível. São vistas contra um fundo escuro e sem referências relativas. Suas distâncias são estupendas e incompreensíveis para nossa intuição visual. Resulta que enxergamos esses pontos de luz como se estivessem todos à mesma distância. Por isso, a esfera celeste aparenta ser uma esfera, ainda que de raio indeterminado.

A imagem pode parecer um pouco incômoda: é como se vivêssemos no interior de uma grande uma casca esférica, centrada na Terra. No entanto, o geocentrismo dessa descrição não é motivo para preocupação, justamente por não passar disso --- uma descrição. Não é um geocentrismo em termos da estrutura física do Universo, mas apenas da escolha de onde é a origem. É meramente uma questão de referencial. E não existem referenciais certos ou errados; todos os referenciais são igualmente legítimos. Temos a liberdade de escolher qual referencial é o mais conveniente, dependendo do que se pretende estudar. Quase todas as observações astronômicas são feitas a partir da superfície da Terra. É, portanto, perfeitamente natural descrever as posições e movimentos dos astros a partir de um referencial centrado no observador --- no caso, centrado na Terra. Por isso, continua sendo conveniente adotar a descrição geocêntrica que sempre foi usada na astronomia de posição. Essa ainda é a maneira mais apropriada de lidar com dados observacionais.

Daí em diante, é sempre possível calcular mudanças de referencial, conforme a necessidade. Por exemplo, para estudar órbitas de planetas e cometas, será apropriado passar para um referencial centrado no Sol; para estudar a dinâmica das estrelas da Via Láctea, será apropriado passar para um referencial cuja origem é o centro da Galáxia, e assim por diante. Nesses casos, seria necessário também conhecer as distâncias. Métodos de determinação de distância são um tópico importantíssimo na astronomia.

\section{Pólos celestes e equador celeste}

\begin{wrapfigure}{r}{0pt}
\includestandalone{figs/equador}
\caption{Esfera celeste mostrando o equador celeste e os pólos celestes.}
\label{fig:equador}
\end{wrapfigure}

Devemos imaginar que a Terra está no centro da esfera celeste. A Terra tem seu eixo de rotação, que passa pelos pólos geográficos Norte e Sul. Se esse eixo for prolongado, ele interceptará a esfera celeste em dois pontos: o Pólo Norte Celeste e o Pólo Sul Celeste, conforme a Fig.~\ref{fig:equador}. A Terra tem seu equador geográfico, que é perpendicular ao eixo de rotação. Da mesma forma, se expandido, o plano do Equador da Terra interceptará a esfera celeste em um grande círculo chamado de Equador Celeste.

Na astronomia de posição, a Terra é estacionária, sem rotação. É a esfera celeste que gira, de Leste para Oeste, ao redor do eixo de rotação definido pelos pólos celestes. Ignorando efeitos de longo prazo, podemos considerar que as constelações são essencialmente fixas. Ou seja, todas as estrelas giram juntas, sem mudar as distâncias relativas entre elas. É como se o desenho das constelações fosse um padrão permanentemente estampado na face interna da esfera celeste. A esfera como um todo gira, mas o padrão não se deforma. É nesse sentido que usamos a expressão `estrelas fixas'. 

\section{Latitude geográfica}

Um observador no hemisfério Norte da Terra verá o pólo celeste Norte permanentemente acima de seu horizonte; equivalentemente, um observador do hemisfério Sul verá sempre o pólo celeste Sul. Para um dado observador, o pólo celeste visível é fixo e o céu inteiro aparenta girar ao redor do pólo. No caso do hemisfério Norte, por acaso existe uma estrela bem próxima da direção do pólo celeste Norte --- é a estrela $\alpha$ da constelação da Ursa Menor, conhecida como estrela Polar, ou Polaris. Na proximidade do pólo celeste Sul, não há uma estrela suficientemente brilhante para servir de referência. É comum usar os termos \textbf{boreal}, que significa do hemisfério Norte; e \textbf{austral}, que significa do hemisfério Sul.

\newpage

\begin{wrapfigure}{r}{0pt}
\includestandalone{figs/latitude}
\caption{Latitude geográfica.}
\label{fig:latitude}
\end{wrapfigure}

A altura do pólo celeste depende da latitude geográfica do observador. Para compreender essa configuração geométrica, precisamos considerar momentaneamente a extensão da Terra. Na Fig.~\ref{fig:latitude}, temos um observador no hemisfério Norte. O plano horizontal desse observador é tangente à superfície da Terra naquele ponto. A vertical do observador, prolongada até o centro da Terra, faz um ângulo $\varphi$ com o equador terrestre. Esse ângulo é a latidude geográfica, que varia no intervalo $-90^{\circ} < \varphi < +90^{\circ}$, sendo positiva no hemisfério Norte e negativa no hemisfério Sul. É fácil notar na Fig.~\ref{fig:latitude} que $\varphi$ também é o ângulo entre o horizonte e o pólo celeste Norte. A única sutileza aqui é reparar que parece haver duas linhas paralelas que apontam para o Norte: uma passando pelo centro da Terra e uma passando pelo observador. Ocorre que o raio da Terra é desprezível diante da esfera celeste, então ambas apontam efetivamente para a mesma direção. Resulta que (em módulo):

\begin{quote}
\textit{A latitude geográfica do observador é igual à altura do pólo celeste visível.}
\end{quote}

\begin{wrapfigure}[18]{r}{0pt}
\includestandalone{figs/ambas}
\caption{Esfera celeste mostrando o horizonte e o equador celeste.}
\label{fig:ambas}
\end{wrapfigure}

Voltando para a representação da esfera celeste na Fig.~\ref{fig:ambas}, podemos ver que o ângulo entre o zênite e o pólo celeste visível é $90^{\circ}-\varphi$. O ângulo entre o horizonte e equador celeste também é $90^{\circ}-\varphi$. O pólo celeste Norte fica acima do ponto cardeal Norte. Se fosse um observador no hemisfério Sul, veríamos o pólo celeste Sul acima do ponto cardeal Sul.

\begin{figure}
\includestandalone{figs/noequador}%
\includestandalone{figs/nopolonorte}\\
\includestandalone{figs/nonorte}%
\includestandalone{figs/nosul} 
\caption{As curvas vermelhas representam o movimento de algumas estrelas, para obsevardores localizados respectivamente: no equador, no pólo Norte, no hemisfério Norte, ou no hemisfério Sul da Terra.}
\label{fig:no}
\end{figure}

Agora estamos em condições de compreender exatamente como as estrelas se movem ao longo da noite, para observadores em qualquer latitude. Na Fig.~\ref{fig:no}, vamos considerar cada um dos casos, começando por um observador localizado no equador da Terra, isto é, com latitude geográfica $\varphi=0^{\circ}$. Nesse caso, a altura do pólo seria nula, então o eixo da esfera celeste encontra-se no plano do horizonte. Os pontos cardeais Norte e Sul coincidem com os pólos celestes Norte e Sul. A~esfera celeste como um todo gira ao redor desse eixo. O primeiro painel da Fig.~\ref{fig:no} mostra a trajetória de algumas estrelas para o observador que está no equador. Todas as estrelas da esfera celeste têm a oportunidade de nascer e se pôr ao longo do ano.

No segundo painel da Fig.~\ref{fig:no}, temos o caso extremo do observador que está no pólo Norte da Terra, isto é, com latitude geográfica $\varphi=+90^{\circ}$. O zênite desse observador coincide com o pólo celeste Norte. Neste caso peculiar, as estrelas não nascem nem se põem, mas continuam girando de Leste para Oeste. O hemisfério celeste Sul fica permanentemente abaixo do horizonte e nunca é visível.

No terceiro painel da Fig.~\ref{fig:no}, temos o caso do observador que está em alguma latitude intermediária qualquer do hemisfério Norte: $0^{\circ} < \varphi < 90^{\circ}$. O pólo celeste visível é o Norte e ele é visto acima do ponto cardeal Norte. As estrelas se movem de Leste para Oeste em arcos paralelos ao equador celeste e portanto inclinados com relação ao horizonte. Algumas estrelas próximas ao pólo celeste Norte nunca descem abaixo do horizonte --- são estrelas circumpolares. Algumas estrelas próximas ao pólo celeste Sul não são visíveis nunca para esse observador, pois ficam sempre abaixo do horizonte.

Finalmente, no quarto painel da Fig.~\ref{fig:no}, temos o caso do observador que está em alguma latitude intermediária qualquer do hemisfério Sul: $-90^{\circ} < \varphi < 0^{\circ}$. A situação é análoga à anterior, com a diferença de que o pólo visível é visto acima do ponto cardeal Sul.


\chapter{Movimento anual do Sol}

Neste capítulo veremos como o movimento diário do Sol muda ao longo do ano.
 
\section{Solstícios e equinócios}

As datas de início das quatro estações do ano correspondem a momentos especiais, chamados de solstícios e equinócios. As datas aproximadas costumavam ser:

\begin{center}
\begin{tabular}{ll}
\hline
21 de março & equinócio de outono \\
21 de junho & solstício de inverno  \\
23 de setembro & equinócio de primavera \\
21 de dezembro & solstício de verão \\
\hline
\end{tabular}
\end{center}

\noindent para o hemisfério Sul. Já para as datas do hemisfério Norte, basta trocar verão$\leftrightarrow$inverno e primavera$\leftrightarrow$outono. O correto é sempre dizer explicitamente a que hemisfério se está referindo. No âmbito deste texto, quando faltar o complemento, fica subentendido que estamos falando na nossa realidade de habitantes do hemisfério Sul.

Veremos a seguir várias propriedades que variam ao longo das estações do ano, começando pela duração do dia. Para evitar confusão, às vezes é usado o termo `dia claro', significando o período em que o Sol está acima do horizonte. Os equinócio são os dois dias do ano em que a duração do dia claro é igual à duração da noite: 12 horas cada. De março a junho, as noites vão ficando progressivamente mais longas, até que no solstício de inverno temos a noite mais longa do ano. De junho até setembro, o dia claro volta a avançar, até que no equinócio de primavera temos novamente 12 horas de dia claro e 12 horas de noite. De setembro até dezembro, o dia claro continua avançando, até que no solstício de verão temos o mais longo dia claro do ano; é a noite mais curta. A partir daí, as noites passam a avançar, até chegar no equinócio de outono e recomeçar o ciclo.

Na prática, isso significa que os horários do nascer do Sol (\textbf{a aurora}) e do pôr do Sol (\textbf{o ocaso}) mudam ao longo do ano. Idealmente, poderíamos imaginar que nos dias dos equinócios, o Sol nasce às 6\,h e se põe às 18\,h. Já no verão (dias claros longos), o Sol nasce antes das 6\,h e se põe depois das 18\,h. No inverno (noites longas), nasce depois das 6\,h e se põe antes das 18\,h. Na realidade, os horários exatos da aurora e do ocaso não são simétricos como se poderia crer, mas são complicados por fatores que não abordaremos agora e que têm relação com a excentricidade da órbita da Terra. De qualquer forma, a variação na duração da noite é perceptível ao longo do ano, mesmo nas nossas latitudes moderadas de cerca de $20^{\circ}$. Em altas latitudes, a diferença é muito mais pronunciada.

É também interessante notar a relação entre as datas e o clima. Por exemplo, o dia 21 de junho é a noite mais curta do ano, ou seja, o dia com a mínima insolação. No entanto, 21 de junho não é necessariamente o dia mais \textit{frio} do ano. Da mesma forma, o dia 21 de dezembro --- máxima insolação --- não é o auge do verão, mas sim o primeiro dia do verão. No segundo dia do verão, o dia claro já começa a ceder lugar à noite, e assim por diante. Ocorre que a Terra não é simplesmente uma esfera uniforme recebendo luz solar. O clima de uma região é também regulado por diversos fatores locais como vegetação, relevo, regime de chuvas e ventos, proximidade do litoral, etc. Existe uma espécie de atraso térmico, já que as grandes massas d'água levam um tempo para conseguir se aquecer ou resfriar como um todo. Resulta que as datas de início das estações calham de funcionar bem, pois em muitos lugares esse atraso faz com que as altas temperaturas acabem aconteçendo por volta no meio do verão etc. Mas isso tudo depende bastante da região em questão. Em particular, os hemisférios da Terra são bastantes assimétricos nesse respeito, já que no hemisfério Sul a fração de área coberta por oceanos é bem maior.

É também curioso perceber que as quatro estações do ano não têm exatamente a mesma duração (conte os dias em um calendário). No hemisfério Sul, temos 186 dias de outono+inverno contra 179 dias de primavera+verão. No hemisfério Norte é o oposto. Ao longo da sua órbita elíptica ao redor do Sol, a velocidade orbital da Terra varia, sendo mais alta velocidade no periélio (janeiro), e mais baixa velocidade no afélio (julho). O periélio e o afélio não têm relação com os solstícios e equinócios.

Por fim, é sempre pertinente ressaltar que a existência das estações do ano é uma conseqüência da inclinação do eixo de rotação da Terra, e não da distância da Terra ao Sol. A excentricida da órbita da Terra é muito pequena. A distância média da Terra ao Sol (que é a Unidade Astronômica) vale aproximadamente 150 milhões de km. Ela varia de cerca de 0.98\,AU no periélio para cerca de 1.02\,AU no afélio.

\section{Trópicos de Câncer e Capricórnio}

\begin{figure}
\centering
\includestandalone{figs/dezembro}
\includestandalone{figs/marcosetembro}
\includestandalone{figs/junho}
\caption{Configuração dos raios de Sol ao meio-dia nos dias dos solstícios e equinócios.}
\label{fig:marcosetembro}
\end{figure}

Vamos considerar agora como observadores em diferentes latitudes recebem os raios de Sol nos quatro dias especiais do ano --- especificamente ao meio-dia verdadeiro. Para isso, precisamos momentaneamente sair do enquadramento da esfera celeste e enxergar a Terra como uma esfera em rotação na qual incidem raios de luz provenientes do Sol, pararelos entre si.

O primeiro painel da Fig.~\ref{fig:marcosetembro} mostra a configuração da Terra recebendo raios solares no dia do solstício de verão do hemisfério Sul. No nosso verão, naturalmente é o hemisfério Sul aquele que está mais voltado para o Sol, enquanto o hemisfério Norte recebe raios de Sol mais oblíquos. Um observador que estivesse no equador da Terra ao meio-dia não veria o Sol no seu zênite, mas sim com uma certa inclinação. Porém, no hemisfério Sul, existe uma latitude tal que a vertical do observador coincida com os raios de Sol neste momento. Essa latitude define o Trópico de Capricórnio e vale aproximadamente $-23^{\circ}$. Um observador ao Sul do Trópico de Capricórnio nunca verá o Sol no zênite.

No segundo painel da Fig.~\ref{fig:marcosetembro} está representada a configuração em qualquer um dos dois equinócios, seja primavera ou verão. Nesses dois dias de alta simetria, a Terra está igualmente iluminada nos dois hemisférios. Portanto, um observador que esteja sobre o equador terrestre, receberá os raios de Sol verticalmente ao meio-dia.

O teceiro painel da Fig.~\ref{fig:marcosetembro} representa o solstício de inverno do hemisfério Sul, quando é o Norte que está mais voltado para o Sol. Analogamente ao primeiro caso, existe uma latitude Norte onde o observador receberá o Sol ao longo do seu zênite ao meio-dia. Essa latitude vale $+23^{\circ}$ e define o Trópico de Câncer. Um observador ao Norte do Trópico de Câncer nunca verá o Sol no zênite. A região da terra contida entre os trópicos de Câncer e Capricórnio é chamda de zona tropical. É fácil se convencer geometricamente que este ângulo também é a inclinação entre o eixo de rotação da Terra e a linha perpendicular ao plano da órbita.

\section{Movimento diário do Sol}

\begin{wrapfigure}{r}{0pt}
\includestandalone{figs/sois}
\caption{Trajetórias do Sol no solstício de verão (vermelho), equinócios (amarelo) e solstício de inverno (azul).}
\label{fig:sois}
\end{wrapfigure}

Agora vamos analisar os solstícios e equinócios do ponto de vista de como um observador vê a trajetória diária do Sol. Na Fig.~\ref{fig:sois} está representada a esfera celeste de um observador localizado no Trópico de Capricórnio. O pólo celeste Sul não está mostrado, mas ele está $23^{\circ}$ acima do ponto cardeal Sul. A linha amarela corresponde à trajetória do Sol nos dias dos equinócios (primavera ou outono). Nessas datas, o Sol está no equador celeste. De fato, a linha amarela é o equador celeste. A linha vermelha é a trajetória do Sol no solstício de verão, quando o Sol está $23^{\circ}$ ao Sul do equador celeste. A linha azul é a trajetória do Sol no solstício de inverno, quando o Sol está $23^{\circ}$ ao Norte do equador celeste.

Note que os três arcos são paralelos entre si. Como o arco amarelo é o equador celeste, ele é um semi-círculo, o que significa que nos dias dos equinócios, o Sol fica 12\,h acima do horizonte. Já o arco vermelho tem mais que $180^{\circ}$, pois no verão o Sol fica mais que 12\,h acima do horizonte. O arco azul tem menos que $180^{\circ}$, pois no inverno o Sol fica menos que 12\,h acima do horizonte.

Apenas nos equinócios o Sol nasce exatamente no ponto cardeal Leste e se põe exatamente no ponto cardeal Oeste. No nosso verão, o Sol está aqui no Sul e por isso ele nasce e se põe em pontos mais ao Sul da linha Leste-Oeste. Já no nosso inverno, o Sol está ao Norte, e por isso ele nasce e se põe em pontos mais ao Norte da linha Leste-Oeste. A longo do ano as trajetórias do Sol na Fig.~\ref{fig:sois} ficam oscilando entre os extremos azul (inverno) e vermelho (verão). Indo do inverno para o verão, passa pelo equinócio de primavera; voltando do verão para o inverno, passa pelo equinócio de outono.

Consideremos agora as sombras projetadas por um gnômon ao meio-dia, lembrando que a Fig.~\ref{fig:sois} é um caso particular do observador que está na latitude $-23^{\circ}$. No solstício de verão, o Sol passa pelo zênite ao meio-dia --- é o único momento sem sombra do ano. Nos equinócios, o Sol do meio-dia passa a $23^{\circ}$ do zênite, projetando uma sombra que aponta para o Sul. No solstício de inverno, o Sol do meio-dia é o mais baixo do ano e projeta uma longa sombra em direção ao Sul.

Caso o observador estivesse em latitudes mais ao Sul do que o Trópico de Capricórnio, o pólo celeste estaria mais alto e conseqüentemente os três arcos da figura estariam menos inclinados com relação ao horizonte. Como resultado, o Sol do meio-dia nunca atingiria o zênite, nem mesmo no verão. Por outro lado, para um observador em algum lugar dentro da zona tropical, o Sol passa duas vezes pelo zênite: uma vez indo para o solstício de verão e outra voltando.

\section{Eclíptica e zodíaco}

Por fim, veremos como o Sol se comporta com relação às estrelas fixas. Para entender essa relação, vale a pena passar por um momento para a perspectiva heliocêntrica, isto é, pensar em termos da órbita da Terra ao redor do Sol. Estando na Terra, e olhando para o céu, digamos que num dado momento o Sol aparente estar na direção da constelação de Áries. Já no mês seguinte, a Terra terá avançado um pouco na sua órbita e agora o Sol aparentará estar na direção da constelação de Touro, e assim por diante ao longo dos 12 meses do ano, até retornar à constelação de Áries.

Voltando agora para a perspectiva geocêntrica da astronomia de posição, temos que o Sol percorre um caminho na esfera celeste ao longo de um ano. Esse percurso do Sol se chama \textbf{eclíptica} e é um grande círculo na esfera celeste, que está inclinado com relação ao equador por um ângulo de $\epsilon = 23^{\circ}$, chamado de \textbf{obliqüidade da eclíptica}. Fisicamente, esse ângulo pode ser entendido como a inclinação entre o equador terrestre e o plano da órbita da Terra. 

Lembramos que o desenho das constelações é fixo e gira como um todo uma vez por dia ao redor da Terra. Já o Sol não é fixo com relação às estrela; ele vai avançando gradualmente ao longo da eclíptica. Para dar uma volta completa na esfera celeste em um ano, o Sol percorre aproximadamente ${\sim}1^{\circ}$ por dia (isto é, $360^{\circ} / 365$ dias). Ao longo dos meses, esse deslocamente é notável. Mas na escala de tempo curta de um único dia, o movimento é pouco apreciável. Então é como se --- a cada dia --- o Sol se comportasse como uma dada estrela fixa.

\begin{wrapfigure}{R}{0pt}
\includestandalone{figs/ecliptica}
\caption{Relação entre a eclíptica e o equador celeste.}
\label{fig:ecliptica}
\end{wrapfigure}

Historicamente, as constelações serviram como referência para esse acompanhamento sistemático do caminho anual do Sol. Porém as constelações têm diferentes extensões angulares. Já na antigüidade, desde o tempo dos babilônios, 12 constelações acabaram sendo codificadas nos \textbf{signos do zodíaco}. Cada um dos 12 signos corresponde a um arco de $30^{\circ}$ ao longo da eclíptica. Assim, a circunferência completa fica divida em 12 partes iguais, que recebem os nomes de: Áries, Touro, Gêmeos, Câncer, Leão, Virgem, Libra, Escorpião, Sagitário, Capricórnio, Aquário e Peixes. Apesar da motivação original em termos das constelações, o fenômeno da precessão dos equinócios ao longo dos milênios faz com que os nomes dos signos estejam atualmente bastante defasados com relação às constelações. Mesmo assim, continuamos usando a nomenclatura dos signos do zodíaco na astronomia.

Podemos agora conectar as datas especiais do ano com as posições do Sol na eclíptica. Na Fig.~\ref{fig:ecliptica} vemos que a eclíptica intercepta o equador em dois pontos: estes são os equinócios. Em junho, o Sol está bem ao Norte, no que seria o signo de Câncer (\Cancer): este é o solstício de verão boreal. Em dezembro, o Sol está bem ao Sul, em Capricórnio (\Capricorn): é o solstício de inverno boreal. Quando o Sol intercepta o equador, indo do inverno boreal para o verão boreal, temos o equinócio de primaveral boreal: nesse momento, 21 de março, o Sol está no signo de Áries (\Aries), que era considerado o começo do ano em certos calendários da antigüidade. Veremos no próximo capítulo que esse ponto é uma referência importante para os sistemas de coordenadas. Esse ponto (\Aries) recebe os nomes de: \textbf{Primeiro Ponto de Áries}, ou \textbf{Ponto Vernal} (já que é a primavera do hemisfério Norte), ou ainda \textbf{Ponto Gama} (pois o símbolo do carneiro se assemelha à letra grega $\gamma$). Por fim, seis meses depois de Áries, quando o Sol cruza novamente o equador celeste, dessa vez descendo do Norte para o Sul, esse é o equinócio de outono boreal, em Libra (\Libra).

\chapter{Sistemas de coordenadas}

Neste capítulo, vamos introduzir os sistemas de coordenadas necessários para este curso. Discutiremos o conceito de tempo sideral detalhadamente.

\section{Coordenadas horizontais}

\begin{wrapfigure}[20]{r}{0pt}
\includestandalone{figs/planovertical}
\caption{Plano vertical de um astro.}
\label{fig:planovertical}
\end{wrapfigure}

O sistema de coordenadas horizontais tem esse nome pois o plano de referência é o horizonte do observador. São também chamadas de coordenadas altazimutais, pois os nomes das coordenadas são altura e azimute, definidos a seguir.

O plano vertical de um astro (Fig.~\ref{fig:planovertical}) é um plano perpendicular ao horizonte e que passa pelo zênite e pelo astro. Isso nos permite definir a primeira coordenada, que é a altura (Fig.~\ref{fig:coord-horizontais}).
 
\textbf{Altura} ($h$) é o ângulo desde o horizonte até o astro, medido ao longo do seu plano vertical. Alturas estão definidas no intervalo:
%
\begin{equation}
-90^{\circ} \leqslant h \leqslant +90^{\circ}.
\end{equation}
%
A altura do horizonte é $h=0^{\circ}$ e a altura do zênite é $h=90^{\circ}$. Alturas negativas significam que o astro está abaixo do horizonte. Alternativamente, podemos utilizar o complemento da altura, que é a distância zenital.

\newpage

\begin{wrapfigure}[22]{R}{0pt}
\includestandalone{figs/coord-horizontais}
\caption{Coordenadas horizontais: altura ($h$) e azimute ($A$).}
\label{fig:coord-horizontais}
\end{wrapfigure}

\textbf{Distância zenital} ($z$) é o ângulo desde o zênite até o astro, medido ao longo do seu plano vertical. A relação entre altura e zênite é simplesmente:
%
\begin{equation}
h + z  =  90^{\circ}.
\end{equation}
%
Portanto, a distância zenital varia no intervalo:
%
\begin{equation}
0^{\circ} \leqslant z \leqslant 180^{\circ},
\end{equation}
%
sendo que astros abaixo do horizonte têm distância zenital maior que $90^{\circ}$.

A segunda coordenada do sistema horizontal é chamada de azimute. \textbf{Azimute} ($A$) é o ângulo entre o ponto cardeal Norte e o vertical do astro, medido sobre o plano do horizonte, de Norte para Leste. O azimute está definido no intervalo:
%
\begin{equation}
0^{\circ} \leqslant A \leqslant 360^{\circ}.
\end{equation}
%
Nessa definição, os pontos cardeais têm os seguintes azimutes: $A=0^{\circ}$ (Norte), $A=90^{\circ}$ (Leste), $A=180^{\circ}$ (Sul) e $A=270^{\circ}$ (Oeste). Essa convenção não é única, sendo possível encontrar referências que adotam outras convenções do sentido de contagem do azimute, então convém reparar com atenção em cada caso.

Resumindo, há duas possibilidades para expressar as coordenadas do sistema horizontal:

\begin{equation*}
\left\{
\begin{aligned}
& \text{altura}~h\\
& \text{azimute}~A
\end{aligned}
\right.
\qquad {\rm ou} \qquad
\left\{
\begin{aligned}
& \text{distância zenital}~z\\
& \text{azimute}~A
\end{aligned}
\right.
\end{equation*}

As coordenadas equatoriais são \textit{locais}, isto é, dependem da latitude do observador. Mais do que isso, $A$ e $z$ de uma dada estrela variam o tempo todo, conforme a esfera celeste gira ao redor do observador.

\section{Coordenadas equatoriais}

No sistema de coordenadas equatoriais, as coordenadas de uma dada estrela são essencialmente `fixas', se ignorarmos efeitos de longo prazo como a precessão e outras pequenas perturbações. O sistema equatorial tem esse nome pois o plano de referência é o equador celeste. Os dois ângulos do sistema equatorial são chamados de ascensão reta e declinação, definidos a seguir (Fig.~\ref{fig:coord-equatoriais}).

\newpage

\begin{wrapfigure}[20]{r}{0pt}
\includestandalone{figs/coord-equatoriais}
\caption{Coordenadas equatoriais: ascensão reta ($\alpha$) e declinação ($\delta$).}
\label{fig:coord-equatoriais}
\end{wrapfigure}

\textbf{Declinação} ($\delta$) é o ângulo desde o equador até a estrela, medido sobre o meridiano que passa pela estrela. A declinação está definida no intervalo:
%
\begin{equation}
-90^{\circ} \leqslant \delta \leqslant +90^{\circ},
\end{equation}
%
sendo positiva no hemisfércio Norte e negativa no hemisfério Sul. A declinação é uma coordenada análoga à latitude geográfica para pontos na superfície terrestre.

Da mesma forma, precisaremos de uma coordenada análoga ao que seria a longitude geográfica. Para definir longitude geográfica, foi necessário escolher um meridiano de referência que sirva como o zero da contagem. Adotou-se como meridiano principal aquele que passa pelo observatório de Greenwich. Analogamente, nas coordenadas equatoriais adota-se como referência o meridiano que passa pelo Ponto Vernal. Assim podemos definir a ascensão reta.

\textbf{Ascensão reta} ($\alpha$) é o ângulo medido sobre o equador celeste, desde o ponto \Aries\ até o meridiano que passa pela estrela. A ascensão reta cresce de Oeste para Leste, isto é, no sentido oposto ao do movimento da esfera celeste. Ou ainda: no sentido anti-horário, se a esfera celeste for vista a partir do pólo Norte. É o sentido da seta amarela na Fig.~\ref{fig:coord-equatoriais}. A ascensão reta é um ângulo que pode variar no intervalo
%
\begin{equation}
0^{\circ} \leqslant \alpha \leqslant 360^{\circ},
\end{equation}
%
É mais comum que a ascensão reta seja expressa não em graus, mas sim em \textit{horas}.\footnote{Entendendo que trata-se da unidade angular hora. Já que 24$^{\rm h}$ = 360$^{\circ}$, cada hora corresponde a 15$^{\circ}$. A hora angular também tem as subdivisões minutos e segundos. A notação usual é no formato: $12^{\rm h}\,34^{\rm m}\,56^{\rm s}$.} Nesse caso, o intervalo de variação da ascensão reta é:
%
\begin{equation}
0^{\rm h} \leqslant \alpha \leqslant 24^{\rm h},
\end{equation}
%
Resumindo, as coordenadas equatoriais são:
%
\begin{equation*}
\left\{
\begin{aligned}
& \text{ascensão reta}~\alpha\\
& \text{declinação}~\delta
\end{aligned}
\right.
\end{equation*}

\newpage
\section{Coordenadas horárias}

No sistema equatorial, ambas as coordenadas ($\alpha, \delta$) são fixas. Vamos introduzir agora um outro sistema onde uma coordenada é fixa e a outra varia com o tempo. Pode parecer uma confusão desnecessária, mas a conveniência prática desse sistema ficará clara mais adiante. Trata-se do sistema equatorial horário, ou apenas sistema horário. No sistema horário, a declinação continua sendo a mesma. Já a ascensão reta será substituída por uma nova coordenada, chamada de ângulo horário.

\begin{wrapfigure}{R}{0pt}
\includestandalone{figs/coord-horarias}
\caption{Coordenadas horarias: ângulo horário ($H$) e declinação ($\delta$).}
\label{fig:coord-horarias}
\end{wrapfigure}

\textbf{Ângulo horário} ($H$) é o ângulo medido sobre o equador, desde o meridiano local até o meridiano que passa pela estrela. O ângulo horário poderia ser expresso em graus (de 0$^{\circ}$ a 360$^{\circ}$), mas como o nome já sugere, ele é mais comumente expresso em horas no intervalo:
%
\begin{equation}
0^{\rm h} \leqslant H \leqslant 24^{\rm h},
\end{equation}
%
O ângulo horário cresce no sentido de Leste para Oeste, ou seja, no mesmo sentido da rotação da esfera celeste. Em outras palavras, o valor de $H$ aumenta conforme o tempo passa. No momento em que uma estrela tem $H = 0^{\rm h}$, ela está cruzando o meridiano local --- é a chamada \textbf{passagem meridiana}. É também quando ela atinge a sua altura máxima --- a chamada culminação. Um certo tempo mais tarde, quando ela já estiver a $15^{\circ}$ a Oeste do meridiano local, seu ângulo horário será $H = 1^{\rm h}$, e assim por diante. Por isso, o ângulo horário funciona como uma espécie de medida da passagem do tempo.

Resumindo, as coordenadas horárias são:
%
\begin{equation*}
\left\{
\begin{aligned}
& \text{ângulo horário}~H\\
& \text{declinação}~\delta
\end{aligned}
\right.
\end{equation*}

A ascensão reta ($\alpha$) e o ângulo horário ($H$) estão conectados entre si através de uma grandeza chamada tempo sideral, que será definida a seguir.

\newpage
\section{Dia sideral}

Nosso objetivo aqui é entender como calcular o tempo sideral em um momento qualquer do ano. Mas, antes disso, vamos apresentar brevemente o conceito de dia sideral. Do ponto de vista físico, podemos pensar na Terra como uma esfera em rotação. Ela tem uma dada velocidade angular e portanto um período, que nada mais é do que o tempo que a Terra leva para dar uma volta completa ao redor de seu eixo. Esse intervalo de tempo é o \textbf{dia sideral}, que dura aproximadamente 23 horas e 56 minutos. A esfera celeste aparenta girar ao nosso redor com esse período. Cada estrela (e em particular o ponto \Aries) leva 23\,h\,56\,min para retornar à mesma posição na esfera celeste. É isso que o céu faz, continuamente.

O que complica as coisas é o Sol e a nossa contagem de tempo atrelada a ele. Ocorre que, ao mesmo tempo em que a Terra dá uma volta ao redor de seu próprio eixo, ela também avança um pouco ao longo da sua órbita ao redor do Sol. O sentido da rotação é o mesmo da translação. Compare o comportamento de uma estrela com o comportamento do Sol. Uma estrela cruza o meridiano local; 23\,h\,56\,min depois, ela cruza o meridiano local novamente. Já o Sol demora um pouco mais: o Sol cruza o meridiano local; 23\,h\,56\,min depois, ele ainda não retornou ao meridiano local. Como a Terra avançou, o Sol ficou um pouco `para trás', e são necessários 4 minutos adicionais para que ele volte a se alinhar com o meridiano local. Esse é o \textbf{dia solar}: é o intervalo de tempo entre duas passagens do Sol pelo meridiano local. O dial solar dura 24 horas. De fato, é esse o intervalo de tempo que decidimos dividir em 24 partes iguais chamadas de `horas'. São essas as horas comuns do tempo civil, isto é, o horário usado nos relógios da vida cotidiana. Por outro lado, se o dia sideral fosse dividido em 24 partes iguais, teríamos as horas siderais.

Considere agora como o céu noturno se comporta com relação ao horário de relógios comuns. Digamos que uma dada estrela cruze o meridiano local hoje à noite às 21\,h\,30\,min, por exemplo. Amanhã à noite, ela cruzará o meridiano local 4 minutos mais cedo, às 21\,h\,26\,min. Depois de amanhã, às 21\,h\,22\,min, e assim por diante.

Pensando de outra forma, imagine observar o céu todas as noites depois do pôr do Sol, num dado horário fixo, por exemplo às 20\,h. Digamos que hoje à noite seja possível ver, acima do horizonte Oeste, a constelação de Áries às 20\,h. Conforme a noite passa, o céu gira e essa constelação se põe, naturalmente. Na noite seguinte, no horário fixo 20\,h, o céu inteiro já estará um pouquinho deslocado para o Oeste. Afinal, as estrelas tinham retornado às mesmas posições já às 19\,h\,56\,min e nesses 4 minutos, avançaram mais um pouco rumo ao Oeste. Como o céu gira 360$^{\circ}$ em 23\,h\,56\,min, em 4 minutos o avanço angular é de aproximadamente 1$^{\circ}$. Ou seja, a cada noite no mesmo horário, o céu já está ${\sim}1^{\circ}$ mais para o Oeste que na noite anterior. Depois de 1 mês (${\sim}30^{\circ}$) a constelação de Áries já estará inteiramente abaixo do horizonte, e será a constelação de Touro que veremos acima do horizonte Oeste depois do pôr do Sol. No mês seguinte, será a vez de Gêmeos, e assim por diante, retornando a Áries depois de 12 meses. É por isso que, mesmo estando em uma latidide fixa, as constelações visíveis no céu noturno mudam ao longo dos meses. O céu funciona como um calendário.
% Os povos da antigüidade tinham familiaridade com diversos fenômenos cíclicos da natureza, como migrações de animais, mudança da vegetação, ciclos de chuvas, cheias dos rios etc. Os ciclos biológicos e climáticos podem ser periódicos em média, mas com alguma variação. Nenhum deles era tão perfeitamente confiável quanto a repetição anual dos movimentos dos astros.

\section{Tempo sideral}
\label{sec:ts}

Voltando às coordenadas horárias, precisamos saber converter de ascensão reta ($\alpha$) para ângulo horário ($H$) e vice-versa. Esses dois ângulos estão conectados entre si através do Tempo Sideral.

\textbf{Tempo sideral} ($TS$) é o ângulo horário do ponto \Aries. Na Fig.~\ref{fig:coord-horarias}, é possível constatar que o ângulo horário do ponto \Aries\ é a soma:
%
\begin{equation}
TS = H + \alpha.
\end{equation}
%
Ou, dito de outra forma, para transformar entre as coordenadas $\alpha$ e $H$, é necessário conhecer o $TS$ da observação.

O ponto \Aries\ naturalmente gira com a esfera celeste como um todo, então ao longo das horas, seu ângulo horário estará variando. Portanto, o $TS$ depende do horário. Mas, por causa dos movimentos descritos na seção anterior, também há uma variação ao longo do ano. A seguir, vamos apresentar uma maneira de \textit{estimar aproximadamente} o $TS$ de qualquer data. Não se trata de um cálculo exato, mas serve para esclarecer alguns conceitos.

Vamos antes relembrar a noção de fuso horário, que vale para o tempo civil. O tempo civil de Greenwich era historicamente chamado de GMT (Greenwich Mean Time), mas a nomenclatura atual é Tempo Universal (sigla UT). Na sigla UTC, freqüentemente usada, o C é de Coordenado; a correção de tempo atômico não vem ao caso. Para passar de tempo universal para o tempo de um dado fuso, basta somar ou subtrair 1 hora inteira a cada $15^{\circ}$ de longitude. Por exemplo, se são 12\,h em Greenwich, são 11\,h para alguém que esteja na longitude $-15^{\circ}$ (Oeste), e são 13\,h para alguém que esteja na longitude $+15^{\circ}$ (Leste). Cada fuso horário teria idealmente a mesma largura de $15^{\circ}$ e todos naquele intervalo adotariam a hora civil do centro do fuso. Na prática, os fusos são arbitrariamente irregulares, para seguir fronteiras de países etc. Nós estamos a Oeste de Greenwich. A hora legal de Brasília é UTC$-3$. Quando são 12\,h em Greenwich, são 9\,h em Brasília.

Quando se trata de tempo sideral, não podemos usar horas inteiras dos fusos. Precisamos da variação contínua das longitudes. A relação entre o TS de um local e o TS de Greenwich é:
%
\begin{equation}\label{TSlocal}
TS_{\rm local} = TS_{\rm Greenwich} + \text{longitude},
\end{equation}
%
sendo a longitude negativa para o Oeste. Por exemplo, a longitude de Curitiba é:
%
\[
-49^{\circ} \, 16' \, 15'' = -49.27^{\circ}
\]
ou expressa em horas (dividir por 15):
\[
-3\,{\rm h}\,17{\rm min}\,5{\rm s} = -3.285\,{\rm h}
\]
%
Então a informação da longitude exata do local (não meramente o fuso) é necessária para o cálculo do $TS$ local. Agora falta entender de onde vem o cálculo do $TS$ de Greenwich.

Lembramos que tempo sideral ($TS$) é o ângulo horário ($H$) do ponto \Aries. E ângulo horário é o ângulo entre o astro e o meridiano local. Precisamos enxergar onde está o ponto \Aries\ em cada momento. O Sol está no ponto \Aries\ no início da primavera do hemisfério Norte (21 de março). Então, ao meio-dia verdadeiro de 21 de março, um observador em Greenwich verá o Sol cruzar o meridiano local. Nesse momento, o ponto \Aries\ terá ângulo horário $H=0^{\rm h}$.

\begin{exemplo}{0}
\textit{Qual o tempo sideral em Greenwich ao meio-dia de 21 de março?}\\

Resposta: $TS = 0$\,h. No equinócio de primavera boreal, o Sol está no ponto \Aries. Ao meio-dia, o ângulo horário do ponto \Aries\ é zero, portanto o tempo sideral é zero.
\end{exemplo}

\begin{exemplo}{1}
\textit{Qual o tempo sideral em Greenwich às 18\,h\,30\,min de 21 de março?}\\

Resposta: $TS = 6$\,h\,30\,min. São 6.5\,h depois do tempo sideral zero.
\end{exemplo}

\begin{exemplo}{2}
\textit{Qual o tempo sideral em Greenwich às 8\,h de 21 de março?}\\

Resposta: $TS = 20$\,h. São 4\,h antes do tempo sideral zero (TS varia entre 0\,h e 24\,h).
\end{exemplo}

Essas estimativas aproximadas de $TS$ no dia do equinócio são toleráveis dentro de um certo erro. Elas não são exatas, entre outros motivos, pois estamos misturando tempo solar com tempo civil, sem levar em conta a Equação do Tempo. Mas o propósito, no momento, é elucidar o funcionamento do $TS$, mesmo que o valor numérico da estimativa não seja tão acurado.

Vejamos agora como o ponto \Aries\ se comporta ao longo dos dias do ano. O ponto \Aries\ (do meio-dia) percorrerá um ângulo de 360$^{\circ}$ (ou 24$^{\rm h}$) ao longo do ano (cerca de 365 dias). Na prática, o $TS$ do meio-dia avança aproximadamente 2\,h por mês. \\

\begin{exemplo}{3}
\textit{Qual o tempo sideral em Greenwich ao meio-dia de 20 de abril?}\\

Resposta: $TS = 2$\,h. Um mês depois do equinócio, o $TS$ do meio-dia terá avançado 2 horas.
\end{exemplo}

\begin{exemplo}{5}
\textit{Qual o tempo sideral em Greenwich às 13\,h de 20 de abril?}\\

Resposta: $TS = 3$\,h. Uma hora depois do exemplo anterior.
\end{exemplo}

No caso geral, é necessário contar o número de dias desde o equinócio de março (de meio-dia a meio-dia). Em seguida, somar ou subtrair as horas para além ou para antes do meio-dia.\\



\begin{exemplo}{6}
\textit{Qual o tempo sideral em Greenwich ao meio-dia de 2 de junho?}\\

Resposta: $TS = 4$\,h\,48\,min. De 21 de março até 2 de junho, passaram-se 73 dias. Com um avanço na taxa de 24\,h/365\,dias = 0.0657 h/dia, tivemos cerca de 4.8\,h = 4\,h\,48\,min.
\end{exemplo}

\begin{exemplo}{7}
\textit{Qual o tempo sideral em Greenwich às 10\,h\,30\,min de 2 de junho?}\\

Resposta: $TS = 3$\,h\,18\,min. Uma hora e meia antes do exemplo anterior.
\end{exemplo}

Todos esses exemplos diziam respeito ao $TS$ de Greenwich. Para passar para o $TS$ de um local qualquer, basta somar a longitude (equação~\ref{TSlocal}).\\

\begin{exemplo}{10ago}
\textit{Qual o tempo sideral em Curitiba às 20\,h\,30\,min de 10 de agosto?}\\

Resposta:\\

Primeiro calcularemos o tempo sideral em Greenwich. Às 20\,h\,30\,min no fuso de Brasília são 23\,h\,30\,min em UT.\\

Em 10 de agosto, já se passaram 142 dias desde 21 de março. Portanto
\[
142\,\text{dias} \times \frac{24\,{\rm h}}{365\,{\rm dias}} = 9.34\,{\rm h} \simeq 9\,{\rm h}\,20\,{\rm min}  
\]
No meio-dia de 10 de agosto em Greenwich, $TS=9$\,h\,20\,min. Onze horas e meia mais tarde, $TS=20$\,h\,50\,min.\\

Finalmente, para passar para o $TS$ de Curitiba, somar a longitude:
\begin{eqnarray*}
TS_{\rm local} &=& TS_{\rm Greenwich} + \text{longitude}\\
TS_{\rm local} &=& 20\,{\rm h}\,50\,{\rm min} - 3\,{\rm h}\,17\,{\rm min} \\
TS_{\rm local} &=&  17\,{\rm h}\,33\,{\rm min}
\end{eqnarray*}

\end{exemplo}

Com isso, temos um procedimento aproximado que serve para compreender o conceito de tempo sideral e também serve para fazer uma estimativa, porém tendo em mente que os valores obtidos podem dar erros de até ${\sim}15$ minutos.

Uma maneira de conferir a resposta correta é usando o Stellarium. É sempre possível escolher a localização, bem como data e horário. Clicando em alguma estrela, aparecem todos os dados; um deles é o tempo sideral.
 
Outra maneira de conferir o cálculo do tempo sideral é através do link \url{https://aa.usno.navy.mil/data/siderealtime}.


\part{Transformações de coordenadas}

\chapter{Trigonometria esférica}

Neste capítulo, vamos usar trigonometria esférica para obter transformações de coordenadas horárias para coordenadas horizontais. 

\section{Triângulo esférico.}

\begin{wrapfigure}{r}{0.4\textwidth}
\centering
\includestandalone{figs/trianguloabc}
\caption{Um triângulo esférico}
\label{fig:trianguloabc}
\end{wrapfigure}
%
Consideremos 3 pontos na superfície de uma esfera, definindo um triângulo esférico. Os lados desse triângulo esférico não são segmentos de reta, mas sim arcos de grandes círculos. Diferentemente da geometria plana (euclidiana), em um triângulo esférico, a soma dos ângulos internos é maior que $180^{\circ}$. O triângulo esférico da Fig.~\ref{fig:trianguloabc} tem lados $a$, $b$ e $c$ e ângulos $A$, $B$ e $C$.

Para nossos fins, o raio da esfera é suposto unitário. Se o centro da esfera for o ponto $O$, então $OB$ e $OC$ são raios. A abertura entre $OB$ e $OC$ vem a ser o ângulo $a$. Já o ângulo $A$ é o ângulo com que os lados $b$ e $c$ se encontram. Está claro que fixados os dois pontos que definem $a$, seria possível ter diferentes ângulos $A$ dependendo da posição do terceiro ponto. Ou seja, $A$ e $a$ não são ângulos iguais.

É possível obter as seguintes relações (ver a dedução detalhada a seguir):
%
\begin{eqnarray}
\cos a &=& \cos b \cos c + \sin b \sin c \cos A        \label{tri1} \\[0.7em]
\frac{\sin a}{\sin A} &=& \frac{\sin b}{\sin B}        \label{tri2} \\[0.7em] 
\sin a \cos B &=& \cos b \sin c - \sin b \cos c \cos A \label{tri3}
\end{eqnarray}

\section*{Demonstração}

\begin{wrapfigure}{r}{0pt}
\centering
\includestandalone{figs/triangulo}
\caption{Um triângulo esférico}
\label{fig:triangulo}
\end{wrapfigure}

Aqui, vamos fazer a demonstração das equações \ref{tri1} a \ref{tri3}, seguindo o livro do Boczko. Na Fig.~\ref{fig:triangulo} está desenhado um triângulo esférico de lados $a$, $b$ e $c$, e de ângulos internos são $A$, $B$ e $C$. Os pontos $A$, $B$ e $C$ estão sobre a esfera e $O$ é a origem. Portanto os segmentos $OA=OB=OC$ são o raio, suposto unitário. Pelo ponto $A$ traça-se o segmento $AK$, que é tangente ao arco $\widearc{AB}$. Da mesma forma, o segmento $AL$ é tangente ao arco $\widearc{AC}$. Portanto são retos os ângulos $O\hat{A}K=O\hat{A}L$.

Considerando o triângulo $LAK$, escrevemos a lei dos cossenos:
%
\begin{equation}
KL^2 = KA^2 + LA^2 - 2~KA~LA~\cos A \label{LAK}
\end{equation}
%
Analogamente, escrevemos a lei dos cossenos para o triângulo $LOK$, notando que o ângulo $L\hat{O}K$ é $a$:
%
\begin{equation}
KL^2 = KO^2 + LO^2 - 2~KO~LO~\cos a \label{LOK}
\end{equation}
%
Igualando as equações \ref{LAK} e \ref{LOK} e rearranjando, ficamos com:
%
\begin{eqnarray}
KO^2 + LO^2 - 2~KO~LO~\cos a &=& KA^2 + LA^2 - 2~KA~LA~\cos A \\
LO^2 - LA^2 + KO^2 - KA^2 &=& 2~KO~LO~\cos a - 2~KA~LA~\cos A
\end{eqnarray}
%
Como o triângulo $LAO$ é reto em $A$:
%
\begin{equation}
LO^2 - LA^2 = AO^2
\end{equation}
%
Similarmente, como o triângulo $KAO$ é reto em $A$:
%
\begin{equation}
KO^2 - KA^2 = AO^2
\end{equation}
%
Substituindo e isolando $\cos a$:
%
\begin{eqnarray}
AO^2 + AO^2 &=& 2~KO~LO~\cos a - 2~KA~LA~\cos A \\
2~KO~LO~\cos a &=& 2 AO^2 + 2~KA~LA~\cos A \\
\cos a &=& \frac{AO^2}{KO~LO} + \frac{KA~LA}{KO~LO}~\cos A
\end{eqnarray}
%
Com o triângulo $LAO$, que é reto em $A$, obtemos o seno e o cosseno de $b$:
%
\begin{equation}
\sin b = \frac{LA}{LO} \qquad \text{e} \qquad \cos b = \frac{AO}{LO}
\end{equation}
%
Com o triângulo $KAO$, que é reto em $A$, obtemos o seno e o cosseno de $c$:
%
\begin{equation}
\sin c = \frac{KA}{KO} \qquad \text{e} \qquad \cos c = \frac{AO}{KO}
\end{equation}
%
Substituindo, chegamos na equação \ref{tri1}, como queríamos demonstrar:
\[
\cos a = \cos b \cos c + \sin b \sin c \cos A \qquad \blacksquare
\]

Com a permutação cíclica dos nomes das variáveis, é imediato escrever as outras duas equações:
\begin{eqnarray}
\cos b &=& \cos a \cos c + \sin a \sin c \cos B \label{cosb} \\
\cos c &=& \cos a \cos b + \sin a \sin b \cos C \label{cosc}
\end{eqnarray}
%
Agora é preciso rearranjar os termos da equação \ref{tri1} e elevar ao quadrado:
\begin{eqnarray}
-\cos A \sin b \sin c &=&  \cos b \cos c - \cos a \\
\cos^2 A \sin^2 b \sin^2 c &=&  \cos^2 b \cos^2 c - 2 \cos a \cos b \cos c + \cos^2 a \label{cos2A}
\end{eqnarray}
%
Fazendo o mesmo procedimento com a equação \ref{cosc}:
\begin{eqnarray}
-\cos B \sin a \sin c &=&  \cos a \cos c - \cos b \\
\cos^2 B \sin^2 a \sin^2 c &=&  \cos^2 a \cos^2 c - 2 \cos a \cos b \cos c + \cos^2 b \label{cos2B}
\end{eqnarray}
%
Agora, na equação \ref{cos2A}, é preciso substituir todos os cossenos quadrados por $\cos^2 x = 1 - \sin^2 x$:
%
\begin{eqnarray}
(1 - \sin^2 A) \sin^2 b \sin^2 c &=&  (1-\sin^2 b) (1-\sin^2 c) + \\ && - 2 \cos a \cos b \cos c + 1 - \sin^2 a \\
\sin^2 b \sin^2 c - \sin^2 A \sin^2 b \sin^2 c &=& 1-\sin^2 c - \sin^2 b + \sin^2 b \sin^2 c + \\ && - 2 \cos a \cos b \cos c + 1 - \sin^2 a \\
\sin^2 b \sin^2 c - \sin^2 A \sin^2 b \sin^2 c - \sin^2 a &=& 2 + 2 \cos a \cos b \cos c \label{abcA}
\end{eqnarray}
%
Da mesma forma, na equação \ref{cos2B}, substituir todos os $\cos^2 x = 1 - \sin^2 x$:
%
\begin{eqnarray}
(1 - \sin^2 B) \sin^2 a \sin^2 c &=&  (1-\sin^2 a) (1-\sin^2 c) + \\ && - 2 \cos a \cos b \cos c + 1 - \sin^2 b \\
\sin^2 a \sin^2 c - \sin^2 B \sin^2 a \sin^2 c &=& 1-\sin^2 c - \sin^2 a + \sin^2 a \sin^2 c + \\ && - 2 \cos a \cos b \cos c + 1 - \sin^2 b \\
\sin^2 a \sin^2 c - \sin^2 B \sin^2 a \sin^2 c - \sin^2 b &=& 2 + 2 \cos a \cos b \cos c \label{abcB}
\end{eqnarray}
%
Como nas equações \ref{abcA} e \ref{abcB} os termos da direita são os mesmos, podemos igualá-las:
%
\begin{eqnarray}
2 + 2 \cos a \cos b \cos c &=& 2 + 2 \cos a \cos b \cos c  \\
\sin^2 b \sin^2 c - \sin^2 A \sin^2 b \sin^2 c - \sin^2 a &=& \sin^2 a \sin^2 c - \sin^2 B \sin^2 a \sin^2 c - \sin^2 b \phantom{2em} \\
\sin^2 A \sin^2 b \sin^2 c &=& \sin^2 B \sin^2 a \sin^2 c \\
\sin^2 A \sin^2 b &=& \sin^2 B \sin^2 a \\
\frac{\sin^2 A}{\sin^2 a} &=& \frac{\sin^2 B}{\sin^2 b} \\
\frac{\sin A}{\sin a} &=& \frac{\sin B}{\sin b}
\end{eqnarray}
%
que é a equação \ref{tri2} que queríamos demonstrar. $\blacksquare$

Para concluir as demonstrações, vamos isolar $\cos a$ na equação \ref{tri1} e substituir na equação \ref{cosb}:
%
\begin{eqnarray}
\cos b &=& (\cos b \cos c + \sin b \sin c \cos A) \cos c + \sin a \sin c \cos B \\
\cos b &=& \cos b \cos^2 c + \sin b \sin c \cos A \cos c + \sin a \sin c \cos B \\
\cos b &=& \cos b (1-\sin^2 c) + \sin b \sin c \cos A \cos c + \sin a \sin c \cos B \\
\cos b &=& \cos b - \cos b \sin^2 c + \sin b \sin c \cos A \cos c + \sin a \sin c \cos B \\
0 &=& - \cos b \sin^2 c + \sin b \sin c \cos A \cos c + \sin a \sin c \cos B
\end{eqnarray}
%
Dividindo tudo por $\sin c$ resulta:
\begin{eqnarray}
0 &=& - \cos b \sin c + \sin b \cos A \cos c + \sin a \cos B \\
\sin a \cos B &=& \cos b \sin c - \sin b \cos c \cos A
\end{eqnarray}
que é a equação \ref{tri3} que queríamos demonstrar. $\blacksquare$

\newpage

\section{De coordenadas horárias para horizontais}

Digamos que esteja-se partindo inicialmente de coordenadas equatoriais ($\alpha, \delta$). Para passar de $\alpha$ para $H$, basta usar o tempo sideral, como visto no capítulo anterior:
%
\[
 H = TS - \alpha
\]
%
Agora estamos no sistema horário e desejamos passar para o sistema horizontal. Para isso, é necessário conhecer a latitude geográfica $\varphi$ do observador. Esquematicamente, dados ($H, \delta$), obter ($z, A$):
%
\begin{equation*}
\left\{
\begin{aligned}
& \text{ângulo horário}~H\\
& \text{declinação}~\delta
\end{aligned}
\right.
\qquad \xrightarrow[\hspace{2cm}]{\varphi} \qquad
\left\{
\begin{aligned}
& \text{distância zenital}~z\\
& \text{azimute}~A
\end{aligned}
\right.
\end{equation*}

Para isso, precisamos desenhar na esfera celeste todos os ângulos em questão, para identificar um triângulo esférico semelhante ao da Fig.~\ref{fig:trianguloabc}. Esse desenho está apresentado na Fig.~\ref{fig:relacao}. Na esfera celeste, repare no triângulo esférico constituído pelos 3 vértices: zênite, pólo Norte celeste e estrela. Lá estão marcados os ângulos que correspondem às coordenadas horizontais e coordenadas horárias, bem como a latitude geográfica.

Vale a pena nos determos para verificar cada elemento da Fig.~\ref{fig:relacao}. Consideremos os lados do triângulo: do pólo Norte ao zênite, temos o complemento da latitude; da estrela ao pólo Norte, temos o complemento da declinação; do zênite à estrela, temos a distância zenital. Consideremos agora dois dos ângulos. Estando no vértice pólo Norte, a abertura entre o meridiano local e o círculo horário que passa pela estrela é o ângulo horário (crescendo para o oeste). Estando no zênite, a abertura entre o Norte e o meridiano que passa pela estrela é o replemento do azimute (já que o azimute cresce para Leste, o azimute seria o ângulo externo do vértice zênite).

No topo da Fig.~\ref{fig:relacao}, o triângulo esférico em questão está passado a limpo para facilitar a visualização. Ele deve ser comparado com o triângulo da Fig.~\ref{fig:trianguloabc}. Dessa comparação, fica clara a equivalência dos lados e ângulos:
%
\begin{equation} \label{equivalencia}
\left\{
\begin{aligned}
a &=& z \\
b &=& 90^{\circ} - \delta \\
c &=& 90^{\circ} - \varphi
\end{aligned}
\right.
\qquad \text{e} \qquad
\left\{
\begin{aligned}
A' &=& H \\
B  &=& 360^{\circ} - A
\end{aligned}
\right.
\end{equation}
%
Como cuidado de notação, repare que aqui o lado do triângulo genérico foi chamado de $A'$ para evitar confusão com $A$, que é o azimute.

Agora, basta substituir as equivalências das equações~\ref{equivalencia} nas equações~\ref{tri1} a \ref{tri3}

\begin{figure}
\centering
\includestandalone{figs/relacao2}
\includestandalone{figs/relacao1}
\caption{Relação entre coordenadas horárias e horizontais.}
\label{fig:relacao}
\end{figure}

\newpage

Finalmente, a transformação de ($H, \delta$) para ($z, A$) resulta ser:\\

\noindent\fbox{\parbox{\textwidth}{
\begin{eqnarray}
\cos z &=& \sin \varphi  \sin \delta  + \cos \varphi  \cos \delta  \cos H \label{conv1} \\
\sin z \cos A &=& \cos \varphi \sin \delta - \sin \varphi \cos \delta \cos H \label{conv2} \\
\sin z \sin A &=& -\sin H \cos \delta \label{conv3}
\end{eqnarray}}}
~\\

\noindent Se for desejada a altura $h$, ao invés da distância zenital, a conversão é trivial, como visto anteriormente:
%
\[
 h = 90^{\circ} - z
\]
%
Pode parecer desnecessário ter 3 equações para isolar 2 incógnitas. Ocorre que conhecer $\cos A$ (equação \ref{conv2}) não é suficiente para saber em qual quadrante está o ângulo $A$. A equação \ref{conv3} é necessária para descobrir o sinal de $\sin A$ e determinar o quadrante de $A$ sem ambigüidade.

Em última análise, agora sabemos calcular a altura e azimute de qualquer estrela, partindo da ascensão reta e da declinação. É preciso saber a latitude geográfica e a data e horário da observação.

\begin{exemplo}{9}
\textit{Uma estrela tem ascensão reta $\alpha = 4^{\rm h}$ e declinação $\delta = 20^{\circ}$. Se a observação for feita num local de latitude $\varphi = -30^{\circ}$ e no tempo sideral $TS = 7\,{\rm h}$, qual será a altura~$h$ e o azimute $A$?}\\

Resposta: Primeiro, usamos $TS$ e $\alpha$ para calcular $H$:
%
\begin{eqnarray*}
TS &=& H + \alpha \\
H &=& TS - \alpha \\
H &=& 7 - 4 \\
H &=& 3\,{\rm h} \\
H &=& 3 \times 15 \\
H &=& 45^{\circ} \\
\end{eqnarray*}

Com a equação \ref{conv1}, isolamos $z$:
%
\begin{eqnarray*}
\cos z &=& \sin \varphi  \sin \delta  + \cos \varphi  \cos \delta  \cos H \\
\cos z &=& \sin(-30^{\circ})  \sin(20^{\circ})  + \cos(-30^{\circ})  \cos(20^{\circ})  \cos(45^{\circ}) \\
\cos z &=& 0.4044 \\
z &=& \arccos(0.4044) \\
z &=& 66.1445^{\circ} \\
z &=& 66^{\circ}\,08'\,40''
\end{eqnarray*}


A altura é
\begin{eqnarray*}
h &=& 90^{\circ} - z \\
h &=& 23^{\circ}\,51'\,20''
\end{eqnarray*}

Com a equação \ref{conv2}, calculamos $\cos A$:
%
\begin{eqnarray*}
\sin z \cos A &=& \cos \varphi \sin \delta - \sin \varphi \cos \delta \cos H \\
\cos A &=& \frac{\cos \varphi \sin \delta - \sin \varphi \cos \delta \cos H}{\sin z} \\
\cos A &=& \frac{\cos(-30^{\circ}) \sin(20^{\circ}) - \sin(-30^{\circ}) \cos(20^{\circ}) \cos(45^{\circ})}{\sin(66.1445^{\circ})} \\
\cos A &=&  0.6871 \\
\bar{A} &=& 46.5964 \\
\bar{A} &=& 46^{\circ}\,35'\,47''
\end{eqnarray*}

$\bar{A}$ é um valor preliminar, pois $A$ de fato pode estar no primeiro quadrante ou no quarto quadrante. Isto é:
%
\begin{eqnarray*}
\text{Se~} \sin A > 0 &:& A = \bar{A} \\
\text{Se~} \sin A < 0 &:& A = 360^{\circ} - \bar{A}
\end{eqnarray*}

Com a equação \ref{conv3}, calculamos o sinal de  $\sin A$:
%
\begin{eqnarray*} 
\sin z \sin A &=& -\sin H \cos \delta \\
\sin A &=& -\frac{\sin H \cos \delta}{\sin z}\\
\sin A &=& -\frac{\sin(45^{\circ}) \cos(20^{\circ})}{\sin(66.1445^{\circ})}\\
\sin A &<& 0
\end{eqnarray*} 
%
Portanto,
\begin{eqnarray*} 
A &=& 360^{\circ} - \bar{A} \\
A &=& 360^{\circ} - 46.5964 \\
A &=& 313.4036^{\circ} \\
A &=& 313^{\circ}\,24'\,13''
\end{eqnarray*} 

\end{exemplo}


\section{De coordenadas horizontais para horárias}

Seria possível aplicar um procedimento análogo para obter a transformação inversa, isto é, dados ($z, A$), obter ($H, \delta$):
%
\begin{equation*}
\left\{
\begin{aligned}
& \text{distância zenital}~z\\
& \text{azimute}~A
\end{aligned}
\right.
\qquad \xrightarrow[\hspace{2cm}]{\varphi} \qquad
\left\{
\begin{aligned}
& \text{ângulo horário}~H\\
& \text{declinação}~\delta
\end{aligned}
\right.
\end{equation*}
%
Sem escrever os passos intermediários, apresentamos as equações que transformam do sistema horizontal para o sistema horário. 
%
\begin{eqnarray}
\sin \delta &=& \cos z \sin \varphi + \sin z \cos \varphi \cos A \label{inv1} \\
\cos \delta \cos H &=& \cos z \cos \varphi - \sin z \sin \varphi \cos A \label{inv2} \\
\sin H \cos \delta &=& -\sin z \sin A \label{inv3}
\end{eqnarray}
%
Naturalmente, o $H$ ao final pode ser convertido para $\alpha$, se soubermos o $TS$.



\chapter{Matrizes de rotação}

Neste capítulo, vamos obter novamente as mesmas transformações do capítulo anterior (de horárias para horizontais), mas desta vez empregando matrizes de rotação para converter entre os dois sistemas.

\section{Rotação dos eixos}

\begin{wrapfigure}[14]{r}{0pt}
\includestandalone{figs/roteixos}
\caption{Rotação por um ângulo $\theta$ ao redor do eixo $z$.}
\label{fig:roteixos}
\end{wrapfigure}

Começamos com o caso mais simples de uma única rotação ao redor do eixo $z$. Consideramos (como na Fig.~\ref{fig:roteixos}) um sistema de eixos $S$ onde um ponto $P$ tem as coordenadas cartesianas $(x, y)$. O sistema $S'$ é obtido a partir de uma rotação anti-horária por um ângulo $\theta$ ao redor do eixo $z$. Nesse sistema de eixos rotacionados, as coordenadas do mesmo ponto $P$ serão $(x', y')$. Na próxima seção, é apresentada uma simples demonstração geométrica da relação entre essas coordenadas. A relação entre as coordenadas resulta ser:
%
\begin{eqnarray}
x' &=& \phantom{+} x \cos \theta + y \sin \theta \label{rot1}\\
y' &=& -x \sin \theta + y \cos \theta \label{rot2}
\end{eqnarray}

\section*{Demonstração}

Aqui vamos fazer uma rápida demostração geométrica das equações \ref{rot1} e \ref{rot2}. Para obter $x'$, a Fig.~\ref{fig:roteixosx} introduz um segmento $a$ e exibe dois triângulos separados para facilitar a visualização.
%
\begin{figure}[h]
\centering
\includestandalone{figs/roteixosx}\qquad
\includestandalone{figs/roteixosx1}\qquad
\includestandalone{figs/roteixosx2}
\caption{Para obter $x'$.}
\label{fig:roteixosx}
\end{figure}

\noindent Temos que:
\begin{eqnarray*}
\cos \theta &=& \frac{x'}{x+a} \\
x' &=& x \cos \theta + a \cos \theta
\end{eqnarray*}
%
Já que $\tan \theta = a / y$, resulta:
\begin{eqnarray*}
x' &=& x \cos \theta + y \tan \theta \cos \theta \\
x' &=& x \cos \theta + y \sin \theta \\
\end{eqnarray*}
%
Assim fica demonstrada a equação \ref{rot1}.

Similarmente, para obter $y'$, a Fig.~\ref{fig:roteixosy} introduz outros segmentos $b$ e $c$ e também exibe triângulos separados para facilitar a visualização.
%
\begin{figure}
\centering
\includestandalone{figs/roteixosy}\qquad
\includestandalone{figs/roteixosy1}\qquad
\includestandalone{figs/roteixosy2}
\caption{Para obter $y'$.}
\label{fig:roteixosy}
\end{figure}
%
Temos que $y = b+c$. Mas $\tan \theta = c/x$, e $\cos \theta = y'/b$. Substituindo $b$ e $c$ e isolando $y'$, resulta:
\begin{eqnarray*}
y &=& b+c \\
y &=& \frac{y'}{\cos \theta} + x \tan \theta \\
y \cos \theta &=& y' + x \tan \theta \cos \theta \\
y \cos \theta &=& y' + x \sin \theta \\
y' &=& -x \sin \theta + y \cos \theta
\end{eqnarray*}
%
Assim fica demonstrada a equação \ref{rot2} $\blacksquare$.

\vspace{1cm}

Voltando agora para as transformações, vamos colocar as coordenadas na notação de vetores:
%
\begin{equation}
\begin{bmatrix}
x'\\
y'
\end{bmatrix}
=
\begin{bmatrix}
\cos \theta & \sin \theta \\
-\sin \theta & \cos \theta 
\end{bmatrix}
\begin{bmatrix}
x\\
y
\end{bmatrix}
\end{equation}
%
\noindent Dessa forma, a matriz $R_z(\theta)$:
%
\begin{equation}
R_z(\theta) =
\begin{bmatrix}
\cos \theta & \sin \theta \\
-\sin \theta & \cos \theta 
\end{bmatrix}
\end{equation}
%
é chamada de \textbf{matriz de rotação} ao redor do eixo $z$, por um ângulo $\theta$.

Um esclarecimento importante: essa matriz serve para rotacionar \textit{o sistema de eixos} por um $\theta$ anti-horário. Note que o ponto $P$ é mantido fixo e são os eixos que rotacionam, passando do sistema $S$ para o sistema $S'$. Uma outra operação, diferente da que estamos tratando, seria a seguinte: mantendo fixo um único dado sistema de eixos, aplicar uma rotação por um $\theta$ anti-horário no vetor posição de $P$; ou seja, mudar $P$ de lugar. Nesse caso, o sinal de $\theta$ seria o oposto. Por isso, a matriz de rotação dessa outra operação tem os sinais dos senos trocados.

No caso de uma rotação ao redor do eixo $z$, está claro que mudam as coordenadas $x$ e $y$, mas a coordenada $z$ permanece inalterada. O mesmo vale para rotações ao redor dos outros eixos, naturalmente. Por completeza, podemos escrever como seriam as matrizes de rotação ao redor de cada eixo, por um ângulo genérico $\theta$:
%
\begin{eqnarray}
R_x(\theta) &=&
\begin{bmatrix}
1 & 0 & 0 \\
0 & \cos \theta & \sin \theta \\
0 & -\sin \theta & \cos \theta
\end{bmatrix} \\[1em]
R_y(\theta) &=&
\begin{bmatrix}
\cos \theta & 0 & -\sin \theta \\
0 & 1 & 0 \\
\sin \theta & 0 & \cos \theta \\ 
\end{bmatrix} \\[1em]
R_z(\theta) &=&
\begin{bmatrix}
\cos \theta & \sin \theta & 0 \\
-\sin \theta & \cos \theta & 0 \\ 
0 & 0 & 1 
\end{bmatrix}
\end{eqnarray}
%
Observando $R_x(\theta)$, vemos que o papel que era desempenhado por $z$ no nosso exemplo original, é agora desempenhado por $x$. Da mesma forma, o papel que era desempenhado por $(x,y)$, é agora desempenhado por $(y,z)$ --- nessa ordem. Uma maneira de se convencer dos sinais em $R_y(\theta)$ é notar que a ordem precisaria ser $(z,x)$ para manter a mesma convenção. 

É possível aplicar sucessivas rotações através de multiplicações de matrizes, mas lembrando que a ordem importa.

\newpage
\section{Coordenadas esféricas}

\begin{wrapfigure}[20]{R}{0pt}
\includestandalone{figs/esfericas}
\caption{Coordenadas esféricas.}
\label{fig:esfericas}
\end{wrapfigure}
%
Para colocar nossas coordenadas angulares em vetores, precisaremos passá-las para coordenadas cartesianas. Por isso, antes de seguir adiante, vamos relembrar a relação entre coordenadas cartesianas e coordenadas esféricas. Conforme a Fig.~\ref{fig:esfericas}, um ponto $P$ está a uma distância $r$ da origem e seu vetor posição faz um ângulo polar $\theta$ com o eixo $z$. Portanto, a projeção de $r$ ao longo do eixo $z$ vale $r \cos \theta$. Já a projeção de $r$ no plano $xy$ vale $r \sin \theta$. Essa projeção no plano, por sua vez, projeta-se com o $\cos \phi$ no eixo $x$ e com o $\sin \phi$ no eixo $y$. Desse modo, resulta que a conexão entre as coordenadas esféricas $(r, \theta, \phi)$ com as coordenadas cartesianas $(x, y, z)$ é dada por:
%
\begin{eqnarray} 
x &=& r \sin \theta \cos \phi \\
y &=& r \sin \theta \sin \phi \\
z &=& r \cos \theta 
\end{eqnarray}
%
Colocando em termos de coordenadas de um vetor \textbf{V}:
%
\begin{equation}
\textbf{V} = 
\begin{bmatrix}
x \\
y \\
z
\end{bmatrix} = 
\begin{bmatrix}
r \sin \theta \cos \phi \\
r \sin \theta \sin \phi \\
r \cos \theta
\end{bmatrix}
\end{equation}


\section{De coordenadas horárias para horizontais}

A transformação de coordenadas horárias para horizontais pode ser compreendida em termos de rotações dos sistemas de eixos. Mas, para isso, precisamos entender cuidadosamente que eixos são esses.

No caso das coordenadas horárias (esfera esquerda da Fig.~\ref{fig:eixosh}), o eixo $z$ corresponde ao eixo que passa pelo pólo Norte. Já o eixo $x$ precisa ser aquele a partir do qual se contam os ângulos da coordenada no plano. E os ângulos precisam crescer na direção de $x$ para $y$. Diante dessas exigências, concluímos que o eixo $x$ do sistema horário aponta na direção do meridiano local. Como o ângulo horário cresce com a passagem do tempo, o eixo $y$ é o que aponta para o Oeste. Isso faz com que o sistema de eixos seja sinistrógiro (que segue a regra da mão \textit{esquerda}).

Já no caso das coordenadas horizontais, o eixo $z$ corresponde ao eixo que passa pelo zênite. O ângulo no plano, azimute, é contado de Norte para Leste, então essas serão as direções dos eixos $x$ e $y$, respectivamente. Na esfera direita da Fig.~\ref{fig:eixosh}, é possível constatar que também trata-se de um sistema sinistrógiro, embora virado para o outro lado.

\begin{figure}
\centering
\includestandalone{figs/eixos-horarios}~
\includestandalone{figs/eixos-horizontais}
\caption{Eixos no sistema de coordenadas horárias (esquerda); e eixos no sistema de coordenadas horizontais (direita).}
\label{fig:eixosh}
\end{figure}

Tendo entendido os eixos cartesianos, agora podemos colocar as coordenadas de ambos os sistemas em vetores. No caso das coordenadas horárias: o ângulo no plano é o ângulo horário; o ângulo polar é o complemento da declinação; o raio é simplesmente tomado como sendo unitário:
%
\begin{eqnarray}
r &=& 1 \\
\theta &=& H \\
\phi &=& 90^{\circ} - \delta
\end{eqnarray}
%
Portanto, as coordenadas cartesianas para o sistema horário ficam:
%
\begin{eqnarray}
x &=& \sin(90^{\circ} - \delta) \cos H \\
y &=& \sin(90^{\circ} - \delta) \sin H \\
z &=& \cos(90^{\circ} - \delta)
\end{eqnarray}
%
Colocando em um vetor:
%
\begin{equation}
\textbf{V}_\text{hor\'arias} =
\begin{bmatrix}
\cos \delta \cos H \\
\cos \delta \sin H \\
\sin \delta
\end{bmatrix}
\end{equation}

Já para as coordenadas horizontais: o ângulo no plano é o azimute; o ângulo polar é a distância zenital; o raio é novamente unitário:
%
\begin{eqnarray}
r &=& 1 \\
\theta &=& A \\
\phi &=& z \text{\quad(este $z$ é distância zenital)}
\end{eqnarray}
%
Portanto, as coordenadas cartesianas para o sistema horizontal ficam, já na notação de vetor:
%
\begin{equation}
\textbf{V}_\text{horiz.} =
\begin{bmatrix}
\sin z \cos A \\
\sin z \sin A \\
\cos z
\end{bmatrix}
\end{equation}

\begin{wrapfigure}{r}{0pt}
\centering
\includestandalone{figs/rot1}
\includestandalone{figs/rot2}
\includestandalone{figs/rot3}
\caption{Rotações para alinhar os eixos da Fig.~\ref{fig:eixosh}.}
\label{fig:rot}
\end{wrapfigure}
%
Agora estamos em condição de aplicar as rotações necessárias para fazer os eixos na esfera esquerda da Fig.~\ref{fig:eixosh} se alinharem com os eixos da esfera direita. A seqüência de operações está ilustrada esquematicamente na Fig.~\ref{fig:rot}. Precisamos primeiro alinhar os eixos $z$ e depois girar $180^{\circ}$. Mais especificamente, a inclinação entre os dois eixos $z$ (Fig.~\ref{fig:eixosh}) vem a ser o ângulo entre o pólo Norte e o zênite. Como a altura do pólo é sempre a latitude, essa inclinação em questão é o complemento da latitude. Portanto, as rotações necessárias são, nessa ordem:
%
\begin{itemize}
\item[(i)] Rotação de $90^{\circ} - \varphi$ ao redor do eixo $y$
\item[(ii)] Rotação de $180^{\circ}$ ao redor do eixo $z$
\end{itemize}
%
A primeira rotação será dada pela matrix $\textbf{R}_y(90^{\circ} - \varphi)$:
%
\begin{eqnarray}
\textbf{R}_y(90^{\circ} - \varphi) &=& 
\begin{bmatrix}
\cos(90^{\circ} - \varphi) & 0 & -\sin(90^{\circ} - \varphi) \\
0 & 1 & 0 \\
\sin(90^{\circ} - \varphi) & 0 & \cos(90^{\circ} - \varphi) \\ 
\end{bmatrix}\\
\textbf{R}_y(90^{\circ} - \varphi) &=& 
\begin{bmatrix}
\sin \varphi & 0 & -\cos \varphi \\
0 & 1 & 0 \\
\cos \varphi & 0 & \sin \varphi \\ 
\end{bmatrix}
\end{eqnarray}

Já a segunda rotação será $\textbf{R}_z(180^{\circ})$:
%
\begin{eqnarray}
\textbf{R}_z(180^{\circ}) &=& 
\begin{bmatrix}
\cos 180^{\circ} & \sin 180^{\circ} & 0 \\
-\sin 180^{\circ} & \cos 180^{\circ} & 0 \\ 
0 & 0 & 1
\end{bmatrix} \\
\textbf{R}_z(180^{\circ}) &=& 
\begin{bmatrix}
-1 & 0 & 0 \\
0 & -1 & 0 \\ 
0 & 0 & 1
\end{bmatrix} \\
\end{eqnarray}

Lembrando que as multiplicações de matrizes são feitas da direita para a esquerda, podemos escrever de maneira compacta:
%
\begin{equation}
\textbf{V}_\text{horiz.} = \textbf{R}_z(180^{\circ}) ~ \textbf{R}_y(90^{\circ} - \varphi) ~ \textbf{V}_\text{hor\'arias}
\end{equation}
%
Escrevendo explicitamente todos os elementos:
%
\begin{equation}
\begin{bmatrix}
\sin z \cos A \\
\sin z \sin A \\
\cos z
\end{bmatrix} = 
\begin{bmatrix}
-1 & 0 & 0 \\
0 & -1 & 0 \\ 
0 & 0 & 1
\end{bmatrix} ~
\begin{bmatrix}
\sin \varphi & 0 & -\cos \varphi \\
0 & 1 & 0 \\
\cos \varphi & 0 & \sin \varphi \\ 
\end{bmatrix} ~
\begin{bmatrix}
\cos \delta \cos H \\
\cos \delta \sin H \\
\sin \delta
\end{bmatrix}
\end{equation}

Efetuando a multiplicação, o resultado é:
%
\begin{equation}
\begin{bmatrix}
\sin z \cos A \\
\sin z \sin A \\
\cos z
\end{bmatrix} = 
\begin{bmatrix}
\cos \varphi \sin \delta - \sin \varphi \cos \delta \cos H \\
-\cos \delta \sin H \\
\cos \varphi \cos \delta \cos H + \sin \varphi \sin \delta
\end{bmatrix}
\end{equation}
%
Este é o mesmo resultado que tinha sido obtido nas equações \ref{conv1} a \ref{conv3} usando trigonometria esférica.

Neste curso não vamos abordar todas as outras transformações. Seria possível, com procedimentos análogos, obter as transformações inversas, de coordenadas horizontais para horárias. Além disso, não vamos estudar aqui os outros sistemas, como coordenadas eclípticas ou galácticas. De qualquer modo, as transformações desses outros sistemas por meio de matrizes de rotação segue os mesmos princípios que foram ilustrados aqui.


\part{Programação em Python}

\chapter{Primeiros passos}

Pretendemos escrever programas em Python para automatizar alguns dos cálculos feitos anteriormente. Neste capítulo, o objetivo é essencialmente desenvolver um programa que calcule a solução do Exemplo \ref{ex:9}. Isto é, partindo de ascensão reta e declinação, obter altura e azimute da estrela. Para isso, é necessário saber a latitude geográfica e tempo sideral. Por enquanto, vamos supor que tempo sideral seja um dado. Com esse programa, ficará simples de re-executar as operações com dados de entrada diferentes.

\section{Graus e radianos}

Vamos começar com alguns exemplos simples de uso de Python. Um pequeno programa para converter um ângulo de graus para radianos seria assim:

\begin{lstlisting}[language=Python]
import numpy as np

angulo_em_graus = 90
angulo_em_radianos = angulo_em_graus * np.pi / 180
print(angulo_em_radianos)
\end{lstlisting}

\noindent\texttt{1.5707963267948966}\\

\noindent A biblioteca numpy foi importada no início apenas para podermos usar o valor de $\pi$. É trivial escrever um programa que faça a operação inversa, convertendo radianos para graus.

Em Python, existem muitas funções já prontas para operações que são freqüentemente necessárias. Por exemplo, já existe a função \texttt{np.radians()}, que converte de graus para radianos e pode ser usada assim, por exemplo:

\begin{lstlisting}[language=Python]
import numpy as np

x = np.radians(90)
\end{lstlisting}

\noindent Analogamente, a função \texttt{np.degrees()} converte de radianos para graus:

\begin{lstlisting}[language=Python]
import numpy as np

y = np.degrees(1.57)
\end{lstlisting}


\section{Graus sexagesimais para graus decimais}

É instrutivo escrever dois pequenos programas que convertam graus sexagesimais (isto é, na notação $45^{\circ}\,30'\,36''$) para graus decimais (na notação $45.51^{\circ}$) --- e vice versa. A sugestão é que o leitor escreva esses dois programas por conta própria, antes de consultar a resolução apresentada a seguir.

Partindo desse exemplo, colocamos cada componente em uma variável:

\begin{lstlisting}[language=Python]
graus = 45
minutos = 30
segundos = 36
\end{lstlisting}

\noindent Primeiro, passamos os segundos para minutos:

\begin{lstlisting}[language=Python]
segundos/60
\end{lstlisting}
\noindent\texttt{0.6}\\

\noindent Agora, passando esses minutos para graus:

\begin{lstlisting}[language=Python]
(minutos + segundos/60)/60
\end{lstlisting}
\noindent\texttt{0.51}\\

\noindent Somando esses graus aos graus inteiros que já estavam lá:

\begin{lstlisting}[language=Python]
graus + (minutos + segundos/60)/60
\end{lstlisting}
\noindent\texttt{45.51}\\

\noindent Passando a limpo, poderíamos escrever uma única linha:

\begin{lstlisting}[language=Python]
graus_decimais = graus + minutos/60 + segundos/3600
\end{lstlisting}

Quando um mesmo procedimento vai ser usado diversas vezes dentro de um programa maior, é conveniente definir uma função, ou seja, algo que recebe um ou mais argumentos e retorna algum resultado. Assim fica mais simples de chamar a função toda vez que for necessária, deixando o código mais legível. Nossa função ficaria assim:

\begin{lstlisting}[language=Python]
def sexagesimal_to_decimal(graus, minutos, segundos):
    graus_decimais = graus + minutos/60 + segundos/3600
    return graus_decimais
\end{lstlisting}

\noindent Ela recebe 3 argumentos, efetua os cálculos e retorna 1 valor. É uma boa prática colocar comentários explicando qual o objetivo da função, o que ela espera receber e o que ela devolve. Por exemplo, algo mais ou menos desse estilo:

\begin{lstlisting}[language=Python]
def sexagesimal_to_decimal(graus, minutos, segundos):
    '''
    Funcao que converte graus sexagesimais para graus decimais
    Input: graus, minutos e segundos
    Returns: graus sexagesimais
    '''
    graus_decimais = graus + minutos/60 + segundos/3600
    return graus_decimais
\end{lstlisting}

\noindent Nesse exemplo específico, pode parecer redundante colocar comentários, pois a função é muito simples. Mas documentação é sempre valiosa.

\section{Graus decimais para graus sexagesimais}

O próximo exemplo é a operação inversa. Partimos de um ângulo dado em graus decimais. A variável de entrada é apenas uma:

\begin{lstlisting}[language=Python]
theta = 45.51
\end{lstlisting}

\noindent O primeiro passo é obter os graus inteiros. Para separar os graus inteiros da parte decimal, vamos tomar o \textit{quociente} da divisão inteira por 1. Em python3, isso se faz com o símbolo \texttt{//}.

\begin{lstlisting}[language=Python]
graus_inteiros = theta // 1
print(graus_inteiros, 'deg')
\end{lstlisting}
\noindent\texttt{45.0 deg}\\

\noindent Precisamos extrair minutos inteiros daquilo que sobrou depois do ponto decimal. Para separar a parte decimal, vamos tomar o \textit{resto} da divisão inteira. Em python3, isso se faz com
o símbolo~\texttt{\%}.

\begin{lstlisting}[language=Python]
graus_decimais = theta % 1
print(graus_decimais, 'deg')
\end{lstlisting}
\noindent\texttt{0.509999999999998 deg}\\

\noindent Para converter esses graus decimais para minutos, basta multiplicar por 60:

\begin{lstlisting}[language=Python]
minutos = graus_decimais * 60
print(minutos, 'arcmin')
\end{lstlisting}
\noindent\texttt{30.59999999999988 arcmin}\\

\noindent Extraindo os minutos inteiros com o quociente da divisão inteira:

\begin{lstlisting}[language=Python]
minutos_inteiros = minutos // 1
print(minutos_inteiros, 'arcmin')
\end{lstlisting}
\noindent\texttt{30.0 arcmin}\\

\noindent Agora, precisamos extrair segundos daquilo que sobrou depois do ponto decimal. Extraindo os minutos decimais com o resto da divisão inteira:

\begin{lstlisting}[language=Python]
minutos_decimais = minutos % 1
print(minutos_decimais, 'arcmin')
\end{lstlisting}
\noindent\texttt{0.5999999999998806 arcmin}\\

\noindent Esses minutos decimais são convertidos para segundos multiplicando por 60:

\begin{lstlisting}[language=Python]
segundos = minutos_decimais * 60
print(segundos, 'arcsec')
\end{lstlisting}
\noindent\texttt{35.99999999999284 arcsec}\\

Passando a limpo, a função definitiva recebe 1 argumento (graus decimais) e retorna 3 valores (graus, minutos e segundos):

\begin{lstlisting}[language=Python]
def decimal_to_sexagesimal(theta):
    graus = theta // 1
    minutos = ((theta % 1) * 60) // 1
    segundos = (((theta % 1) * 60) % 1) * 60
    return graus, minutos, segundos
\end{lstlisting}

\noindent Se o ângulo de entrada puder ser negativo, essa função teria problemas. (Por exemplo, teste o comportamento de \texttt{-45.9 // 1}). Uma maneira simples de contornar esse problema seria fazer
todos os cálculos usando o módulo do ângulo. E depois colocar o sinal de volta na resposta:

\begin{lstlisting}[language=Python]
import numpy as np

def decimal_to_sexagesimal(theta):
    '''
    Funcao que converte graus decimais para graus sexagesimais
    Input: graus, minutos, segundos
    Returns: graus decimais
    '''
    S = np.sign(theta)
    theta = abs(theta)
    graus = theta // 1
    minutos = ((theta % 1) * 60) // 1
    segundos = (((theta % 1) * 60) % 1) * 60
    return S*graus, S*minutos, S*segundos
\end{lstlisting}

\noindent Testando o uso da função com um input qualquer:

\begin{lstlisting}[language=Python]
graus, minutos, segundos = decimal_to_sexagesimal(-10.0997)
print(graus, 'deg', minutos, 'arcmin', segundos, 'arcsec')
\end{lstlisting}
\noindent\texttt{-10.0 deg -5.0 arcmin -58.92000000000124 arcsec}\\

Haveria outras maneiras de lidar com inteiros e restos, por exemplo empregando a função \texttt{int()}.

\newpage
\section{Transformação de coordenadas}
\label{sec:trans}

O objetivo agora é escrever uma função que receba as coordenadas equatoriais ($\alpha, \delta$) e retorne como resposta as coordenadas horizontais ($h, A$). É preciso conhecer a latitude e o tempo sideral. Os dados de entrada seriam da seguinte forma (usando os mesmos valores do exemplo \ref{ex:9}):

\begin{lstlisting}[language=Python]
ra  =   4.0 # h 
dec =  20.0 # deg
TS  =   7.0 # h
phi = -30.0 # deg
\end{lstlisting}

\noindent Trata-se essencialmente de implementar as equações \ref{conv1} a \ref{conv3}. Essa tarefa é intencionalmente deixada como exercício. Usando como ponto de partida o esqueleto da função abaixo, complete o resto do código e teste para o exemplo já solucionado anteriormente.

\begin{lstlisting}[language=Python]
def radec_to_Ah(ra, dec, TS, phi):
    '''
    Funcao que converte coordenadas equatoriais para horizontais
    input
    ra : ascensao reta (horas decimais) [0,24]
    dec : declinacao (graus decimais) [-90,90]
    TS : tempo sideral (horas decimais) [0,24]
    phi : latitude geografica (graus decimais) [-90,90]
    output
    A : azimute (graus decimais) [0,360]
    h : altura (graus decimais) [-90,90]
    '''
    
    
    
    
    
    
    
    
    
    
    # complete o resto do codigo aqui...

    
    
    
    
    
    
    
    
    
    return A, h
\end{lstlisting}



\chapter{Um programa completo}

O programa escrito no capítulo anterior supunha que o tempo sideral já era conhecido. Neste capítulo, vamos escrever uma função que nos permita calcular o $TS$ de qualquer data e horário. Assimilando essa função do tempo sideral às anteriores, teremos um programa completo que prevê qualquer observação.

Serão apresentadas 4 abordagens diferentes para se calcular o tempo sideral: contagem explícita de dias; contagem de dias lançando mão de uma biblioteca de Python; uso da Data Juliana; e tempo sideral diretamente com biblioteca de Python.

\section{Contagem explícita de dias}
\label{sec:ts1}

Partindo de uma data e horário quaisquer, uma maneira de estimar aproximadamente o tempo sideral é adotar o procedimento que já usamos nos exemplos da Seção~\ref{sec:ts}. Para isso, é necessário contar o número de dias decorridos desde o equinócio de março. Automatizar essa contagem é um pouquinho mais inconveniente do que seria de se supor. Também é um bom exercício de programação. O enunciado do problema consiste em partir de uma data qualquer (dia $D$, mês $M$) e também do horário (hh:mm:ss) e se perguntar qual o número de dias decorridos desde o meio-dia de 21 de março.

Por exemplo, se o dia for 10 de agosto e o horário for às 23\,h\,30 UT (como no exemplo \ref{ex:10ago}), os dados de entrada seriam valores da seguinte forma:

\begin{lstlisting}[language=Python]
dia = 10
mes = 8
hora = 23
minuto = 30
segundo = 0
\end{lstlisting}

\noindent Uma função de contagem de dias que receba esses valores, deveria retornar como resposta: 142 dias, 11 horas e 30 minutos, ou 142.4792 dias. Implementar esse programa fica como exercício.

\section{Contagem de dias com datetime}
\label{sec:ts2}

Em Python, existe uma biblioteca chamada datetime, que serve justamente para manipular datas e horários. Entre outras coisas, é possível criar objetos do tipo:

\begin{lstlisting}[language=Python]
datetime.datetime(year, month, day, hour, minute, second)
\end{lstlisting}

\noindent e em seguida calcular diferenças, somar intervalos de tempo etc.

O programa a seguir exemplifica o uso de algumas funções do datetime. Especificamente, é calculado nesse exemplo o número de dias decorridos desde 21/03/2023 ao meio-dia de Greewich até o dia 10/08/2023 às 20\,h\,30\,min no fuso de Brasília.

\begin{lstlisting}[language=Python]
import datetime
from datetime import timedelta

dia = 10
mes = 8
ano = 2023

hora    = 20 
minuto  = 30
segundo =  0

fuso = -3

# Local Time
# colocar o horario local no formato datetime
LT = datetime.datetime(ano, mes, dia, hora, minuto, segundo)

# Universal Time
# subtrair o fuso para obter UT
UT = LT - timedelta(hours=fuso)

# colocar 21/mar no formato datetime
Equinocio = datetime.datetime(ano, 3, 21, 12, 0, 0)

# intervalo de tempo entre as duas datas
dias_desde_21mar = (UT - Equinocio).total_seconds() / 86400

print('UT = ',  UT.isoformat(' '))
print('Dias desde equinocio = ', dias_desde_21mar)

\end{lstlisting}
\noindent\texttt{UT =  2023-08-10 23:30:00}\\
\noindent\texttt{Dias desde equinocio =  142.47916666666666}\\

Repare que a biblioteca datetime está meramente sendo usada para contar o número de dias. A estimativa de tempo sideral usando esse método é a mesma do método anterior.

\section{Data Juliana}
\label{sec:ts3}

Aqui é introduzido um novo modo de calcular o tempo sideral, mais acurado do que as nossas estimativas que foram calculadas de maneira aproximada. Trata-se de uma fórmula proposta pela IAU (União Astronômica Internacional), que serve para calcular o tempo sideral em Greenwhich em qualquer instante. Mas essa fórmula envolve a chamada data juliana. 

A \textbf{data juliana} é uma contagem de dias cujo zero foi definido como sendo o meio-dia do dia 1 de janeiro de 4713 a.C. 

Para calcular o dia juliano, precisamos da data usual do calendário gregoriano (dia, mês e ano). Suponhamos que esses dados de entrada sejam da seguinte forma, para o exemplo da data 10/08/2023:

\begin{lstlisting}[language=Python]
D = 10
M = 8
A = 2023
\end{lstlisting}

\noindent Um algoritmo para o cálculo do dia juliano é apresentado na apostila do Gastão. Está implementado na função a seguir, que recebe como argumentos dia, mês e ano:

\begin{lstlisting}[language=Python]
def dia_juliano(D, M, A):
    if(M<3):
        A = A - 1
        M = M + 12
    A1 = int(A/100)
    A2 = 2 - A1 + int(A1/4)        
    if( (A<1582) & (M<10) & (D<4)):
        A2 = 0
    JD = int(365.25 * (A+4716)) + int(30.6001 * (M+1)) + D + A2 - 1524.5
    return JD
\end{lstlisting}

\noindent Para o exemplo usado, resultará o dia juliano $JD = 2460166.5$. A fração 0.5 dia ocorre porque o resultado do algoritmo corresponde a 0\,h em UT. É possível checar rapidamente os valores das datas julianas no Stellarium.

Existe uma grandeza auxiliar chamada de \textbf{século juliano}. É o número de séculos desde 1/1/2000, sendo que cada século tem 36525 dias. Seria o $T_{\rm J2000}$, mas podemos adotar a notação $T$ apenas. O século juliano $T$ é calculado a partir da data juliana $JD$ de acordo com a seguinte definição:
%
\begin{equation}
T = \frac{JD - 2451545}{36525}
\end{equation}
%
Com o dia juliano do exemplo, o século juliano vale aproximadamente $T = 0.2360438056$. Se desejarmos ter precisão de minutos no fim da conta, então o número de séculos precisará de da ordem de 8 casas decimais.

Finalmente, podemos usar o valor do século juliano $T$ na fórmula da IAU\footnote{\url{https://ui.adsabs.harvard.edu/abs/1982A\%26A...105..359A}}. Esta equação fornece o tempo sideral em Greenwhich, em segundos:
%
\begin{equation}
TS = 24110.54841 + 8640184.812866~T + 0.093104~T^2 - 0.0000062~T^3
\end{equation}
%
Dividindo o resultado por 3600, teremos o $TS$ em horas, ainda à 0\,h de UT. Para outros horários, somar as horas. Por exemplo, se são 23\,h\,30\,min em UT, somar 23.5\,h. Para um valor mais acurado, devemos multiplicar as horas pelo fator 1.0027, que é a razão entre dia solar e dia sideral (no exemplo, $1.0027 \times 23.5\,{\rm h} = 23.5634$\,h) Assim, teremos o $TS$ em Greenwhich. Quando o valor dá maior que 24\,h, tomamos o resto da divisão inteira por 24. Por exemplo, se o resultado desse 26\,h, o tempo sideral seria 2\,h; se desse 49\,h, o tempo sideral seria 1\,h. No nosso exemplo, o tempo sideral em Greenwhich resulta ser 20.7790\,h.

Finalmente, para passar para o $TS$ local, basta somar a longitude. Com a longitude de Curitiba ($-49.27^{\circ} = -3.285$\,h), o tempo sideral local do exemplo termina sendo 17.4943\,h, ou 17\,h\,29\,min\,39\,s. Constatamos que nossa estimativa no exemplo \ref{ex:10ago} tinha dado um valor próximo, com erro de apenas $\sim$4 minutos.

\section{Tempo sideral com astropy}
\label{sec:ts4}

A última maneira de calcular o tempo sideral seria usando funções de Python prontas para isso. A biblioteca astropy é muito utilizada na astronomia profissional e tem uma função que calcula o tempo sideral. O programa a seguir exemplifica esse uso.

\begin{lstlisting}[language=Python]
from astropy import units as u
from astropy.time import Time, TimezoneInfo
from astropy.coordinates import EarthLocation
from datetime import datetime

# coordenadas geograficas de Curitiba
lat = '-25d30m09s'
lon = '-49d17m30s'
location = EarthLocation(lat=lat, lon=lon)

# data e horario
dia = 10
mes = 8
ano = 2023
h = 20
m = 30
s = 0

# informa que o fuso vem a ser o UTC-3
tzinfo = TimezoneInfo(utc_offset=-3*u.hour)

# objeto datetime
time = datetime(ano, mes, dia, h, m, s, tzinfo=tzinfo)

# tempo com astropy
t = Time(time, scale='utc', location=location)
print('UT = ', t)


# tempo sideral com astropy
ts = t.sidereal_time('mean')
print('TS = ', ts.hour)
print('TS = ', ts)

\end{lstlisting}
\noindent\texttt{UT =  2023-08-10 23:30:00}\\
\noindent\texttt{TS =  17.492862062339007}\\
\noindent\texttt{TS =  17h29m34.3034s}\\


\section{O programa final}

Por fim, a culminação dos nossos esforços será reunir as ferramentas desenvolvidas anteriormente em um único programa. Na Seção \ref{sec:trans} havíamos implementado uma função que converte coordenadas, mas supondo o tempo sideral já dado. Agora o objetivo é escolher um dos métodos apresentados nas Seções \ref{sec:ts1} a \ref{sec:ts4} para calcular o $TS$ diretamente a partir de qualquer data e horário. Assim, dentro do mesmo programa, o $TS$ será calculado e em seguida será fornecido como argumento para a função que converte coordenadas. Teremos o procedimento completo, do começo ao fim.

Agora você pode ecolher alguma estrela qualquer de sua preferência e calcular suas coordenadas para a noite da observação.

\section{Um catálogo de estrelas}

Depois que o programa estiver funcionando para uma dada estrela qualquer, seria possível editar facilmente as coordenadas ascensão reta e declinação no início do programa.

Uma abordagem ainda mais automática seria fazer um programa que lê uma lista de coordenadas ($\alpha, \delta$) e imprime como saída uma lista de coordenadas ($h, A$), cabendo ao usuário editar o horário desejado da observação e o local.

Um exercício extra interessante seria o seguinte. Primeiro, componha uma tabela contendo as coordenadas equatoriais das estrelas mais brilhantes de cada uma das constelações do zodíaco. Por exemplo, a estrela mais brilhante da constelação de Touro ($\alpha$ Tauri, também chamada de Aldebaran) tem as coordenadas:

\begin{center}
\begin{tabular}{ccccccc}
\hline
nome & ascensão reta & declinação  \\
& (hh:mm:ss) & (dd:mm:ss) \\

\hline
&&\\
Aldebaran ($\alpha$ Tauri) & $04~35~55.2$ & $+16~30~33.5$ \\
&&\\
&&\\
&&\\
\hline
\end{tabular}
\end{center}

\noindent Complete a tabela, buscando as coordenadas em alguma base de dados (talvez o SIMBAD\footnote{\url{https://simbad.cds.unistra.fr/simbad/}}). Salve em um arquivo de texto. Em seguida, faça o programa ler esse arquivo de texto, colocando cada grandeza (cada coluna) na variável apropriada. Agora, ao invés de efetuar os cálculos para 1 única estrela, o programa fará o mesmo para uma lista de 12 estrelas. A saída do programa vai ser um outro arquivo de texto: uma tabela de 12 linhas com as alturas e azimutes de cada estrela. Vai ser interessante reparar em quais estarão abaixo do horizonte numa dada noite, num dado horário.

Você pode escolher uma dada noite e fazer um loop para rodar o programa de hora em hora para ver como as alturas mudam.

Alternativamente, pode fixar um horário depois do pôr do Sol e rodar o programa de mês em mês para ver quais constelações estarão visíveis.

Em última análise, é isso o que o Stellarium está fazendo por dentro. E também é isso que os astrônomos da antigüidade já tinham compreendido.

\vspace{2cm}
\begin{center}
 $\star$
\end{center}


\part{Atividades observacionais}

\chapter{Determinação da linha meridiana}
\label{cap:meridiano}

\begin{wrapfigure}{r}{0pt}
\centering
\includestandalone{figs/linhaNS}
\caption{Determinação da linha Norte-Sul.}
\label{fig:linhaNS}
\end{wrapfigure}

O objetivo desta atividade prática é determinar o meridiano local, isto é, a direção Norte-Sul. O método consiste em fazer observações da direção da sombra do Sol ao redor do meio-dia. Este método é interessante pois prescinde de qualquer instrumento de medida ou qualquer acesso a informações externas. Não precisamos usar relógio, nem régua, nem bússola. Não é preciso consultar nenhuma informação em livros ou na internet. Não é preciso saber a latitude e nem sequer o dia do ano. Basta usar uma estaca vertical (gnômon) e um barbante.

Neste contexto, estamos entendendo por meio-dia o meio-dia verdadeiro, não o da hora civil. O instante de menor sombra do dia é o momento em que o Sol passa pelo meridiano local. De manhã, a sombra aponta para o Oeste; à tarde a sombra aponta para o Leste.

O procedimento é:

\begin{enumerate}

\item Num momento qualquer \textit{antes} do meio-dia, observe a sombra projetada no solo e faça uma marcação no chão de onde está a extremidade da sombra. É o ponto $A$ na Fig.~\ref{fig:linhaNS}.

\item Trace no chão um círculo de raio $IA$. Isso poderia ser feito estendendo um barbante desde a base do gnômon até a extremidade da sombra.

\item Em algum momento \textit{depois} do meio-dia, a extremidade da sombra interceptará o círculo desenhado. Quando isso acontecer, marque o ponto $A'$.

\item A bissetriz do ângulo $A\hat{I}A'$ é a direção Norte-Sul. Para saber qual direção é o Norte, basta notar que à tarde o Sol estará no lado Oeste.

\end{enumerate}

Uma maneira de melhorar a precisão seria marcar mais de um ponto no lado da manhã ($B$, $C$, etc), bem como suas contrapartidas à tarde ($B'$, $C'$, etc). A média dessas bissetrizes tende a ser mais precisa do que uma única medida.

Conceitualmente, seria suficiente passar um único traço no momento de menor sombra. Ocorre que, na prática, é difícil ter certeza de quando exatamente esse instante acontece. Inevitavelmente acabaríamos marcando algumas sombras antes e outras depois, para ter confiança de qual foi a mínima sombra. Então vale a pena lançar mão da simetria manhã-tarde.

Por fim, convém ressaltar que os momentos $A$ e $A'$ são simétricos com relaçao ao meio-dia verdadeiro, não o meio-dia da hora civil. Por exemplo, não serão horários como 11\,h\,00\,min e 13\,h\,00\,min, justamente porque o meio-dia verdadeiro não ocorrerá às 12\,h\,00\,min. Este método é engenhoso justamente porque evita ter que lidar com horários.


\chapter{Determinação da latitude geográfica}

\begin{figure}[ht]
\centering
\includestandalone{figs/equinocios}
\caption{Raios de Sol durante o equinócio para um observador no hemisfério Sul ao meio-dia verdadeiro.}
\label{fig:equinocios}
\end{figure}

Esta atividade prática consiste em determinar a latitude geográfica. Precisa ser realizada no dia do equinócio (primavera ou outono), ao meio-dia verdadeiro. Aqui precisaremos medir o tamanho do gnômon e o comprimento de sua sombra.

A Fig.~\ref{fig:equinocios} representa a configuração da Terra e dos raios solares ao meio-dia do equinócio. Nesse momento, o Sol está no equador celeste. Então um observador localizado no equador da Terra receberia os raios solares verticalmente. Considere um observador, neste mesmo momento, localizado em um ponto $P$, com uma latitude qualquer $\varphi$. Na Fig.~\ref{fig:equinocios}, é fácil peceber que o ângulo $\varphi$ é igual ao ângulo entre o Sol e o zênite desse observador. Em outras palavras: a distância zenital do Sol é a latitude geográfica (ou melhor, seu módulo).

\begin{wrapfigure}[18]{r}{0pt}
\includestandalone{figs/zenital}
\caption{Distância zenital do Sol.}
\label{fig:zenital}
\end{wrapfigure}

O procedimento prático é tão simples quanto o cálculo em si. Conforme a Fig.~\ref{fig:zenital}, basta medir o comprimento do gnômon ($l$) e o comprimento da sombra ($s$). A distância zenital do Sol será:
%
\begin{equation}
\tan z = \frac{s}{l}
\end{equation}
%
e, naquele momento, $\varphi = z$. Assim fica determinada a latitude geográfica. Embora não haja muita dificuldade prática em medir o comprimento de uma sombra, a sutileza deste método é saber o momento correto do meio-dia verdadeiro.

Para fins desta atividade, uma maneira prática de contornar tal dificuldade consistiria em tomar uma série de medidas. Um pouco antes do presumido meio-dia, começamos a tomar medidas da sombra do gnômon, separadas entre si por alguns minutos, digamos. Se continuarmos fazendo isso para além do meio-dia verdadeiro, ficará claro nos dados que uma das medições corresponde à mínima sombra. É aquela que deve ser usada.

Além disso, se a atividade do capítulo anterior tiver sido feita previamente, já teremos uma marcação da linha meridiana no chão. O instante de sombra mínima deve acontecer quando a sombra do gnômon estiver alinhada com a linha meridiana.

Mesmo assim, é interessante entender por quais motivos a mínima sobra ocorreu com alguns minutos de defasagem do meio-dia do relógio.





\chapter{A que horas acontece o meio-dia?}

Neste capítulo, vamos entender por quais motivos o meio-dia verdadeiro (isto é, o instante de mínima sombra do dia) não coincide em geral com as 12\,h\,00\,min da hora civil. Veremos como calcular a correção. Ao fim, é possível confirmar na prática esse horário.

Um dos motivos para a diferença tem a ver com a longitude. Pense na rotação da Terra, que se dá de Oeste para Leste. A cada momento, o Sol estará sobre um determinado meridiano. Considere, por exemplo, algumas cidades e suas longitudes: São Paulo (longitude 46.6$^{\circ}$ Oeste), Curitiba ($49.3^{\circ}$ Oeste) e Foz do Iguaçu ($54.6^{\circ}$ Oeste). Digamos que agora o Sol esteja passando pelo meridiano de São Paulo; é meio-dia verdadeiro lá. Daqui alguns minutos, o Sol passará pelo meridiano de Curitiba. Daqui mais alguns minutos, será a vez de Foz do Iguaçu. No entanto, todos dentro desse fuso adotam a mesma hora civil. Não tem como o meio-dia do relógio corresponder ao meio-dia verdadeiro para todas as cidades ao mesmo tempo. As cidades que, em princípio, teriam a oportunidade de evitar essa confusão são aquelas no centro do fuso, onde a longitude realmente é $-3\,{\rm h} = -45^{\circ}$ (mas mesmo isso ainda não é a explicação completa). O centro do fuso não passa em Brasília (o Distrito Federal não foi feito para isso). O meridiano $-15^{\circ}$ acontece de passar em algum local intermediário entre Rio de Janeiro e São Paulo. Confira em um mapa. Uma das cidades que mais se aproximam de estar no centro do fuso é Ubatuba, no litoral de SP, por onde coincidentemente também passa o trópico de Capricórnio.

A conclusão com relação à longitude é a seguinte. Como nosso fuso é centrado em $45^{\circ}$ Oeste, as cidades que tiverem longitude um pouco mais a Oeste (por exemplo, $46^{\circ}$), veriam o Sol passar pelo meridiano alguns minutos depois de 12\,h. Já as cidade um pouco mais a Leste (por exemplo, $44^{\circ}$), veriam o Sol passar pelo meridiano alguns minutos antes de 12\,h. Mas ainda não acabou. Existe outro efeito, discutido a seguir, que traz uma segunda contribuição no cálculo.

O que descrevemos até agora (o Sol estar no meridiano etc) corresponde ao conceito de tempo solar. Cada longitude teria seu tempo solar local. No entanto, tudo isso trata do tempo solar \textit{verdadeiro}. Apesar de intuitivo, o tempo solar verdadeiro é inconveniente para marcação quantitativa do tempo, pois há irregularidades (físicas) no movimento aparente do Sol. Introduz-se então a noção de tempo solar \textit{médio}. Os motivos físicos dessas irregularidades são principalmente: a inclinação do eixo e a excentricidade da órbita da Terra (além de outros pequenos efeitos devidos a perturbações gravitacionais da Lua etc). Isso faz com que o Sol aparente estar mais veloz ou mais lento em diferentes épocas do ano. A correção é conhecida pelo nome de \textbf{equação do tempo}. Esse desvio varia desde $-14$\,min até $+16$\,min ao longo do ano (na verdade, varia lentamente ao longo dos séculos também). Não vamos detalhar o cálculo da equação do tempo em maior profundidade, mas seu resultado numérico pode ser consultado em um gráfico --- por exemplo, na figura 2.5 da apostila do Gastão. Aliás, essa correção é necessária para fazer a leitura de relógios de Sol em geral.



\begin{exemplo}{11}
\textit{Em Curitiba, no dia 25 de maio, o meio-dia verdadeiro acontecerá em que horário?}\\

Resposta: Há duas contribuições na correção: uma devida à longitude e outra devida à equação do tempo.\\

(i) Longitude: Considere a longitude de Curitiba, que é aproximadamente $49.3^{\circ}$ Oeste. Estamos cerca de 4.3$^{\circ}$ mais a Oeste do que o centro do nosso fuso. Então quando for meio-dia no horário civil de Brasília, o Sol ainda não terá chegado no meridiano de Curitiba. Como $15^{\circ}$ equivalem a 1\,h, esses 4.3$^{\circ}$ correspondem a cerca de 0.29\,h, ou 17 minutos. Então o Sol só chegaria no meridiano de Curitiba 17 minutos depois do meio-dia civil, às 12\,h\,17\,min. Essa correção é uma contribuição permanente para os horários na longitude de Curitiba; vale o ano todo.\\

(ii) Equação do tempo: Neste exemplo, estamos interessados no dia 25 de maio, que é o 145$^{\underline{\circ}}$ dia do ano. Consultando no gráfico da apostila do Gastão, descobrimos que nesse dia a equação do tempo vale aproximadamente $+3$ minutos. Pela convenção, o sinal positivo significa que o Sol médio está \textit{atrasado} com relação ao Sol verdadeiro. Então é preciso \textit{subtrair} 3 minutos do tempo civil para encontrar o meio-dia verdadeiro. Partindo dos 12\,h\,17\,min obtidos por causa da longitude, a nossa previsão é de que o meio-dia verdadeiro (em Curitiba, em 25 de maio) ocorra às 12\,h\,14\,min do tempo civil. Confira no Stellarium.

\end{exemplo}

Recapitulando: No caso geral, a passagem meridiana do Sol (meio-dia verdadeiro) ocorrerá quando o relógio indicar alguns minutos antes ou depois das 12\,h\,00\,min. Os motivos para esses minutos são essencialmente: (i) a cidade não está no centro do seu fuso horário; (ii) eixo de rotação da Terra é inclinado e (iii) a órbita da Terra não é circular. O primeiro efeito causa uma defasagem fixa para aquele lugar. Os dois outros efeitos geram um atraso ou avanço, que varia ao longo do ano. 

\begin{exemplo}{12}
\textit{Em Curitiba, no dia 21 de março, o meio-dia verdadeiro acontecerá em que horário?}\\

Obtenha esse horário de 3 maneiras:\\

(a) Fazendo o cálculo como no exemplo anterior. Use a mesma defasagem da longitude e tome cuidado com o sinal da equação do tempo.\\

(b) Confira a resposta no Stellarium.\\

(c) Meça diretamente as sombras no dia 21 de março! \\
 
\end{exemplo}

% Estamos interessados no dia 21 de março, que é o 80$^{\underline{\circ}}$ dia do ano. Consultando no gráfico, descobrimos que nesse dia a equação do tempo vale aproximadamente $-7$ minutos. Pela convenção, o sinal negativo significa que o Sol médio está adiantado com relação ao Sol verdadeiro. Então é preciso adicionar 7 minutos do tempo civil para encontrar o meio-dia verdadeiro. Partindo dos 12\,h\,17\,min obtidos por causa da longitude, a nossa previsão é de que o meio-dia verdadeiro (em Curitiba, em 21 de março) ocorra às 12\,h\,24\,min do tempo civil. Confira no Stellarium.  

Para concluir, uma curiosidade. No dia 3 de novembro, o gráfico da equação do tempo dá um valor de mais de $+16$ minutos, o que quase cancela a defasagem da longitude de Curitiba; nesse dia, o meio-dia verdadeiro acontece bem próximo de 12\,h\,00\,min.




\chapter{Observação noturna}

Na primeira noite de observação, cabe aos alunos propor a observação a ser feita. Com os cálculos e programas desenvolvidos ao longo do curso, é possível prever as coordenadas de uma estrela na data e horário da observação. Um desafio aos alunos seria tentar confeccionar um instrumento rudimentar que permita medir ângulos (altura e azimute). Qual seria a precisão da leitura? Ao invés de medir azimute, seria possível também medir o instante da passagem meridiana de uma estrela.

Na segunda noite, finalmente faremos observações com um telescópio em que o apontamento é computadorizado.

 
% format chapter title
\titleformat{\chapter}[display]
{\normalfont\huge\bfseries}{\color{mygreen}\chaptertitlename\ \thechapter}{20pt}{\color{mygreen}\Huge}

\begin{thebibliography}{9}
% \thispagestyle{empty}
\bibitem[Boczko(1984)]{Boczko} Boczko R., 1984, \textit{Conceitos de astronomia}, Editora Edgard Bl\"ucher
\bibitem[Gastao(2022)]{Gastao} Lima Neto G.~B., 2022, \textit{Astronomia de Posição}, IAG/USP
\end{thebibliography}

% back cover
\clearpage
{
\pagecolor{mygreen}
\thispagestyle{empty}
~
}

\end{document}
